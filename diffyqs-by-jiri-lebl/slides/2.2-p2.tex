\documentclass[10pt,aspectratio=169]{beamer}

% All the boilerplate is in deslides.sty
\usepackage{deslides}

\author{Ji\v{r}\'i Lebl}

\institute[OSU]{%
Oklahoma State University%
%Departemento pri Matematiko de Oklahoma {\^S}tata Universitato%
}

\title{15. Constant coefficient second order linear ODEs (part 2)\\(Notes on Diffy Qs, 2.2)}

\date{}

\begin{document}

\begin{frame}
\titlepage

%\bigskip

\begin{center}
The textbook: \url{https://www.jirka.org/diffyqs/}
\end{center}
\end{frame}

\begin{frame}
Review: Given
\quad
$a y'' + b y' + c y = 0$

\medskip

Try $y = e^{rx}$:
\quad
$a r^2 e^{rx} + 
b r e^{rx} + 
c e^{rx} = 0$

\medskip
\pause
Divide by $e^{rx}$ to get the
\emph{characteristic equation} of the ODE:
\[
a r^2 + 
b r + 
c = 0 .
\]
\pause
Solve for the $r$:
\qquad 
$r_1, r_2 = \dfrac{-b \pm \sqrt{b^2 - 4ac}}{2a}$.

\medskip
\pause

\wthus $e^{r_1 x}$ and $e^{r_2 x}$ are linearly independent solutions.

\medskip
\pause

If $r=r_1=r_2$, then $e^{rx}$ and $xe^{rx}$ are linearly independent
solutions.

\medskip
\pause

But, what if the characteristic equation has no real roots?
E.g., $r^2+1=0$.
\end{frame}

\begin{frame}
$r^2+1=0$ has \emph{complex} roots.  If $i$ is such that
$i^2=-1$, then $i^2+1=0$ and $(-i)^2+1 = 0$.

\pause
\medskip

\textbf{Remark:} Engineers sometimes use $j$ instead of $i$.

\medskip
\pause

More generally, \emph{complex} numbers are just a pair of real numbers
$(a,b)$, where we decree that
\[
(a,b)+(c,d)\overset{\text{def}}{=}(a+c,b+d) \qquad  \text{and} \qquad
(a,b) \times (c,d) \overset{\text{def}}{=} (ac-bd,ad+bc)
\]
\pause
For real numbers we say $a = (a,0)$, and we write $i=(0,1)$.

\medskip
\pause

Then $i^2 \pause = i \times i \pause = (0,1) \times (0,1) \pause = (-1,0)
\pause = -1$.

\medskip
\pause

We write $a+ib = (a,b)$.

\pause
$a$ is the \emph{real part}, write $\Re (a+ib) = a$.

\pause
$b$ is the \emph{imaginary part}, write $\Im (a+ib) = b$.

\medskip
\pause

\textbf{Example:}
\quad
$(2+3i)(4i) - 5i
\pause
=
(2\times 4)i + (3 \times 4) i^2 - 5i
\pause
=
8i + 12 (-1) - 5i
\pause
=
-12 + 3i$

\medskip
\pause

\textbf{Example:}
\quad
$(3+2i)(3-2i)
\pause
= 3^2-(2i)^2
\pause
= 3^2+2^2
=13$

\medskip
\pause

\textbf{Example:}
\quad
$\displaystyle \frac{1}{3-2i}
\pause
= \frac{1}{3-2i} \frac{3+2i}{3+2i}
\pause
= \frac{3+2i}{13}
\pause
= \frac{3}{13}+\frac{2}{13}i$.
\end{frame}

\begin{frame}
The exponential should satisfy $e^{x+y} = e^xe^y$.

\medskip
\pause

So $e^{a+ib} = e^a e^{ib}$.

\medskip
\pause

\textbf{Euler's formula:}
\quad
$e^{i \theta} = \cos \theta + i \sin \theta$
\quad and \quad
$e^{- i \theta} = \cos \theta - i \sin \theta$

\medskip
\pause

\thus\quad
$e^{a+ib} 
\pause
= e^a e^{ib}
\pause
=
e^a (\cos b + i \sin b)
\pause
=
e^a \cos b + i e^a \sin b$

\medskip
\pause

\textbf{Exercise:}
Check the identities:
\quad
$\cos \theta = \dfrac{e^{i \theta} + e^{-i \theta}}{2}$
\quad and \quad
$\sin \theta = \dfrac{e^{i \theta} - e^{-i \theta}}{2i}$.

\medskip
\pause

\textbf{Remark:} Most trig identities follow from $e^{z+w} = e^ze^w$ for
complex numbers.

\end{frame}

\begin{frame}
$ay'' + by' + cy = 0$ has the characteristic equation $a r^2 + b r + c = 0$.

\medskip
\pause

Suppose the roots are complex, that is, $b^2 - 4ac < 0$.

\medskip
\pause

The roots are (quadratic formula)
\quad
$r_1,r_2
=
\dfrac{-b \pm \sqrt{b^2 - 4ac}}{2a}
\pause
=
\dfrac{-b}{2a} \pm i\dfrac{\sqrt{4ac - b^2}}{2a}$.

\medskip
\pause

The roots are always a pair of the form $\alpha \pm i \beta$.

\medskip
\pause

The general solution is
\quad
$y = C_1 e^{(\alpha+i\beta)x} + C_2 e^{(\alpha-i\beta)x}$.

\medskip
\pause

The exponential is complex valued and so must be $C_1$, $C_2$.
Let's do better.

\medskip
\pause

Write
\quad
$y_1
= e^{(\alpha+i\beta)x}
\pause
=
e^{\alpha x} \cos (\beta x) + i e^{\alpha x} \sin (\beta x)$,
\pause
\quad
$y_2
= e^{(\alpha-i\beta)x}
\pause
=
e^{\alpha x} \cos (\beta x) - i e^{\alpha x} \sin (\beta x)$.

\medskip
\pause

These are also solutions:
\quad
$y_3 = \dfrac{y_1 + y_2}{2}
\pause 
= e^{\alpha x} \cos (\beta x)$
\pause
\quad
$y_4  = \dfrac{y_1 - y_2}{2i}
\pause
= e^{\alpha x} \sin (\beta x)$

\pause
\begin{theorem}
If the characteristic equation has the roots $\alpha \pm i \beta$
(i.e., $b^2 - 4ac < 0$),
then the general solution is
\[
y = C_1 e^{\alpha x} \cos (\beta x) + C_2 e^{\alpha x} \sin (\beta x) .
\]
\end{theorem}
\end{frame}

\begin{frame}

\textbf{Example:} Solve \quad $y'' + k^2 y = 0$, \quad for a constant $k > 0$.

\medskip
\pause

Characteristic equation is $r^2 + k^2 = 0$,
\pause
roots are $r = \pm ik$,
\pause
the general solution is
\[
y = C_1 \cos (kx) + C_2 \sin (kx) .
\]

\medskip
\pause

\textbf{Example:}
Solve \quad $y'' - 6 y' + 13 y = 0$, \quad $y(0) = 0$, $y'(0) = 10$.

\medskip
\pause

Characteristic equation is $r^2 - 6 r + 13 = 0$.

\pause
Complete the square ${(r-3)}^2 + 2^2 = 0$,
\pause
so $r = 3 \pm 2i$,
\pause
the general solution is
\[
y =
C_1 e^{3x} \cos (2x) + C_2 e^{3x} \sin (2x).
\]

\medskip
\pause

$0 = y(0)
\pause
= C_1 e^{0} \cos 0 + C_2 e^{0} \sin 0
\pause
= C_1$
\pause
\wthus $C_1 = 0$ and $y = C_2 e^{3x} \sin (2x)$.

\medskip
\pause
$y' = 3C_2 e^{3x} \sin (2x) + 2C_2 e^{3x} \cos (2x)$
\pause
\wthus $10 = y'(0) = 2C_2$, or $C_2 = 5$.

\medskip
\pause
The solution is
\[
y = 5 e^{3x} \sin (2x) 
\]

\end{frame}

\begin{frame}
Complex numbers are useful also for the Cauchy--Euler equations.

\medskip

\textbf{Exercise:}
Suppose ${(b-a)}^2-4ac < 0$.  Find a formula for the general solution
of $a x^2 y'' + b x y' + c y = 0$.  Hint: Try $y=x^r$ and note $x^r = e^{r \ln x}$.

\medskip
\pause

Try it with something simple like $x^2 y'' + y = 0$.

\end{frame}

\end{document}
