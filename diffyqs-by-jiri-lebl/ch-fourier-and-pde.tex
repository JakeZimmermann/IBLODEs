\chapter{Fourier series and PDEs} \label{FS:chapter}

%%%%%%%%%%%%%%%%%%%%%%%%%%%%%%%%%%%%%%%%%%%%%%%%%%%%%%%%%%%%%%%%%%%%%%%%%%%%%%

\section{Boundary value problems} \label{bvp:section}

%mbxINTROSUBSECTION

\sectionnotes{2 lectures\EPref{, similar to \S3.8 in \cite{EP}}\BDref{,
\S10.1 and \S11.1 in \cite{BD}}}

\subsection{Boundary value problems}

Before we tackle the Fourier series, we study
the so-called 
\emph{boundary value problems\index{boundary value problem}}
(or \emph{endpoint problems\index{endpoint problem}}).  Consider
\begin{equation*}
x'' + \lambda x = 0, \quad x(a) = 0, \quad x(b) = 0,
\end{equation*}
for some constant $\lambda$, where $x(t)$ is defined for $t$ in the interval
$[a,b]$.
Previously we specified the value of the solution and its derivative
at a single point.  Now we specify the value of the solution at two different
points.  As $x=0$ is a solution, existence of
solutions is not a problem.  Uniqueness of solutions is another issue.
The general solution to $x'' + \lambda x = 0$ has two
arbitrary constants\footnote{%
See \subsectionvref{subsection:fourfundamental} or \examplevref{example:expsecondorder} and
\examplevref{example:sincossecondorder}.}.
It is, therefore,
natural (but wrong) to believe that requiring two
conditions guarantees a unique solution.

\begin{example}
Take $\lambda = 1$,
$a=0$, $b=\pi$.  That is,
\begin{equation*}
x'' + x = 0, \quad x(0) = 0, \quad x(\pi) = 0.
\end{equation*}
Then $x = \sin t$ is another solution (besides $x=0$) satisfying both boundary
conditions.  There are more.  Write down the general
solution of the differential equation, which is $x= A \cos t + B \sin t$.
The condition $x(0) = 0$ forces $A=0$.  Letting $x(\pi) = 0$ does not
give us any more information as $x = B \sin t$ already satisfies both
boundary conditions.
Hence, there are infinitely many solutions of the form $x = B \sin t$,
where $B$ is an arbitrary constant.
\end{example}

\begin{example}
On the other hand, consider $\lambda = 2$.  That is,
\begin{equation*}
x'' + 2 x = 0, \quad x(0) = 0, \quad x(\pi) = 0.
\end{equation*}
Then the general solution is
$x= A \cos ( \sqrt{2}\,t) + B \sin ( \sqrt{2}\,t)$.  Letting $x(0) = 0$ still
forces $A = 0$.  We apply the second condition to find
$0=x(\pi) = B \sin ( \sqrt{2}\,\pi)$.
As $\sin ( \sqrt{2}\,\pi) \not= 0$ we obtain
$B = 0$.  Therefore $x=0$ is the unique solution to this problem.
\end{example}

What is going on?  We will be interested in finding which
constants $\lambda$ allow a nonzero solution, and we will be interested in
finding those solutions.  This problem is an analogue of finding
eigenvalues and eigenvectors of matrices.  

\subsection{Eigenvalue problems}

For basic Fourier series theory we will need
the following three eigenvalue problems.
We will consider more general equations and boundary conditions,
but we will postpone this until
\chapterref{SL:chapter}.
\begin{equation} \label{bv:eq1}
x'' + \lambda x = 0, \quad x(a) = 0, \quad x(b) = 0 ,
\end{equation}
\begin{equation} \label{bv:eq2}
x'' + \lambda x = 0, \quad x'(a) = 0, \quad x'(b) = 0 ,
\end{equation}
and
\begin{equation} \label{bv:eq3}
x'' + \lambda x = 0, \quad x(a) = x(b), \quad x'(a) = x'(b) .
\end{equation}
A number $\lambda$ is called an
\emph{eigenvalue\index{eigenvalue of a boundary value problem}}
of \eqref{bv:eq1}
(resp.\ \eqref{bv:eq2} or \eqref{bv:eq3}) if and only if
there exists a nonzero (not identically zero) solution to \eqref{bv:eq1}
(resp.\ \eqref{bv:eq2} or \eqref{bv:eq3})
given that specific $\lambda$.  A
nonzero solution is called a corresponding
\emph{\myindex{eigenfunction}}\index{corresponding eigenfunction}.

Note the similarity to eigenvalues and eigenvectors of matrices.  The
similarity is not just coincidental.  If we think of the equations as
differential operators, then we are doing the same exact thing.
Think of a function $x(t)$
as a vector with infinitely many components (one for each $t$).
Let $L = -\frac{d^2}{{dt}^2}$ be the linear operator.
Then the eigenvalue/eigenfunction pair should be $\lambda$ and
nonzero $x$ such that $Lx = \lambda x$.
In other words,
we are looking for nonzero functions $x$
satisfying certain endpoint conditions that solve
$(L- \lambda)x = 0$.  A lot of the formalism from linear algebra still
applies here, though we will not pursue this line of reasoning too far.

\begin{example} \label{bvp:eig1ex}
Let us find the eigenvalues and eigenfunctions of
\begin{equation*}
x'' + \lambda x = 0, \quad x(0) = 0, \quad x(\pi) = 0 .
\end{equation*}

%For reasons that will be clear from the computations,
We have to handle
the cases $\lambda > 0$, $\lambda = 0$, $\lambda < 0$ separately.
First suppose that $\lambda > 0$.  Then
the general solution to $x''+\lambda x = 0$ is
\begin{equation*}
x = A \cos ( \sqrt{\lambda}\, t) + B \sin ( \sqrt{\lambda}\, t).
\end{equation*}
The condition $x(0) = 0$ implies immediately $A = 0$.
Next
\begin{equation*}
0 = x(\pi) = B \sin ( \sqrt{\lambda}\, \pi ) .
\end{equation*}
If $B$ is zero, then $x$ is not a nonzero solution.  So to get a nonzero
solution we must have that $\sin ( \sqrt{\lambda}\, \pi) = 0$.  Hence,
$\sqrt{\lambda}\, \pi$ must be an integer multiple of $\pi$.  In other words,
 $\sqrt{\lambda} = k$ for a positive integer $k$.
Hence the positive eigenvalues are
$k^2$ for all integers $k \geq 1$.  Corresponding eigenfunctions
can be taken as $x=\sin (k t)$.  Just like for eigenvectors, constant
multiples of an eigenfunction are also eigenfunctions,
so we only need to pick one.

Now suppose that $\lambda = 0$.  In this case the equation is $x'' = 0$,
and its general solution is $x = At + B$.  The condition $x(0) = 0$ implies
that $B=0$, and $x(\pi) = 0$ implies that $A = 0$.  This means that $\lambda
= 0$ is \emph{not} an eigenvalue.

Finally, suppose that $\lambda < 0$.  In this case we have the general
solution\footnote{Recall that
$\cosh s = \frac{1}{2}(e^s+e^{-s})$
and
$\sinh s = \frac{1}{2}(e^s-e^{-s})$.  As an exercise
try the computation with the general solution written as
$x = A e^{\sqrt{-\lambda}\, t} + B e^{-\sqrt{-\lambda}\, t}$ (for
different $A$ and $B$ of course).}
\begin{equation*}
x = A \cosh ( \sqrt{-\lambda}\, t) + B \sinh ( \sqrt{-\lambda}\, t ) .
\end{equation*}
Letting $x(0) = 0$ implies that $A = 0$ (recall $\cosh 0 = 1$ and $\sinh 0 =
0$).  So our solution must be $x = B \sinh ( \sqrt{-\lambda}\, t )$ and satisfy
$x(\pi) = 0$.  This is only possible if $B$ is zero.  Why?  Because
$\sinh \xi$ is only zero when $\xi=0$.  You should plot sinh to see this
fact.
We can also see this from the definition of sinh.
We get $0 = \sinh \xi = \frac{e^\xi -
e^{-\xi}}{2}$.  Hence $e^\xi = e^{-\xi}$, which implies $\xi = -\xi$ and that is only
true if $\xi=0$.  So there are no negative eigenvalues.

In summary, the eigenvalues and corresponding eigenfunctions are
\begin{equation*}
\lambda_k = k^2 \qquad \text{with an eigenfunction} \qquad x_k = \sin (k t)
\qquad \text{for all integers } k \geq 1 .
\end{equation*}
\end{example}

\begin{example}
Let us compute the 
 eigenvalues and eigenfunctions of
\begin{equation*}
x'' + \lambda x = 0, \quad x'(0) = 0, \quad x'(\pi) = 0 .
\end{equation*}

Again we have to handle the cases $\lambda > 0$, $\lambda = 0$, $\lambda
< 0$ separately.
First suppose that $\lambda > 0$.
The general solution to $x''+\lambda x = 0$ is
$x = A \cos ( \sqrt{\lambda}\, t) + B \sin ( \sqrt{\lambda}\, t)$.  So
\begin{equation*}
x' = -A\sqrt{\lambda}\, \sin ( \sqrt{\lambda}\, t) + B\sqrt{\lambda}\,
\cos (\sqrt{\lambda}\, t) .
\end{equation*}
The condition $x'(0) = 0$ implies immediately $B = 0$.
Next
\begin{equation*}
0 = x'(\pi) = -A\sqrt{\lambda}\, \sin ( \sqrt{\lambda}\, \pi) .
\end{equation*}
Again $A$ cannot be zero if $\lambda$ is to be an eigenvalue,
and $\sin ( \sqrt{\lambda}\, \pi)$ is only zero
if
$\sqrt{\lambda} = k$ for a positive integer $k$.
Hence the positive eigenvalues are again
$k^2$ for all integers $k \geq 1$.  And the corresponding eigenfunctions
can be taken as $x=\cos (k t)$.

Now suppose that $\lambda = 0$.  In this case the equation is $x'' = 0$
and the general solution is $x = At + B$ so $x' = A$.  The condition
$x'(0) = 0$ implies that
$A=0$.  The condition $x'(\pi) = 0$ also implies $A=0$.
Hence $B$ could be anything (let us take it to be 1).  So $\lambda = 0$
is an eigenvalue and $x=1$ is a corresponding eigenfunction.

Finally, let $\lambda < 0$.  In this case the general solution is
$x = A \cosh ( \sqrt{-\lambda}\, t) + B \sinh ( \sqrt{-\lambda}\, t)$
and
\begin{equation*}
x' = A\sqrt{-\lambda}\, \sinh ( \sqrt{-\lambda}\, t)
+ B\sqrt{-\lambda}\, \cosh ( \sqrt{-\lambda}\, t ) .
\end{equation*}
We have already seen (with roles of $A$ and $B$ switched) that for this
expression to be zero at $t=0$ and $t=\pi$, we must have $A=B=0$.  Hence there are
no negative eigenvalues.

In summary, the eigenvalues and corresponding eigenfunctions are
\begin{equation*}
\lambda_k = k^2 \qquad \text{with an eigenfunction} \qquad x_k = \cos (k t)
\qquad \text{for all integers } k \geq 1 ,
\end{equation*}
and there is another eigenvalue
\begin{equation*}
\lambda_0 = 0 \qquad \text{with an eigenfunction} \qquad x_0 = 1.
\end{equation*}
\end{example}

The following problem is the one that leads to the general Fourier
series.

\begin{example} \label{bvp-periodic:example}
Let us compute the 
eigenvalues and eigenfunctions of
\begin{equation*}
x'' + \lambda x = 0, \quad x(-\pi) = x(\pi), \quad x'(-\pi) = x'(\pi) .
\end{equation*}
We have not specified the values or the derivatives
at the endpoints, but rather that they are the same at the beginning and
at the end of the interval.

Let us skip $\lambda < 0$.  The computations are the same as before,
and again we find
that there are no negative eigenvalues.

For $\lambda = 0$, the general solution is $x = At + B$.  The condition
$x(-\pi) = x(\pi)$ implies that $A=0$ ($A\pi + B = -A\pi +B$ implies $A=0$).
The second condition $x'(-\pi) = x'(\pi)$ says nothing about $B$ and hence
$\lambda=0$ is an eigenvalue with a corresponding eigenfunction $x=1$.

For $\lambda > 0$ we get that
$x = A \cos ( \sqrt{\lambda}\, t ) + B \sin ( \sqrt{\lambda}\, t)$.
Now
\begin{equation*}
\underbrace{A \cos (-\sqrt{\lambda}\, \pi) + B \sin (-\sqrt{\lambda}\,
\pi)}_{x(-\pi)}
=
\underbrace{A \cos (  \sqrt{\lambda}\, \pi ) + B \sin ( \sqrt{\lambda}\,
\pi)}_{x(\pi)} .
\end{equation*}
We remember that $\cos (- \theta) = \cos (\theta)$ and
$\sin (-\theta) = - \sin (\theta)$.  Therefore,
\begin{equation*}
A \cos (\sqrt{\lambda}\, \pi) - B \sin ( \sqrt{\lambda}\, \pi)
=
A \cos (\sqrt{\lambda}\, \pi) + B \sin ( \sqrt{\lambda}\, \pi).
\end{equation*}
Hence either $B=0$ or $\sin ( \sqrt{\lambda}\, \pi) = 0$.
Similarly (exercise) if we differentiate $x$ and plug in the second
condition we find that $A=0$ or $\sin ( \sqrt{\lambda}\, \pi) = 0$.
Therefore, unless we want $A$ and $B$ to both be zero (which we do not)
we must have $\sin ( \sqrt{\lambda}\, \pi ) = 0$.  Hence, $\sqrt{\lambda}$
is an integer and the eigenvalues are yet again $\lambda = k^2$ for
an integer $k \geq 1$.  In this case, however, 
$x = A \cos (k t) + B \sin (k t)$ is an eigenfunction for any $A$ and any $B$.
So we have two linearly independent eigenfunctions $\sin (kt)$ and $\cos (kt)$.
Remember that for a matrix we can also have two eigenvectors
corresponding to a single eigenvalue if the eigenvalue is repeated.

In summary, the eigenvalues and corresponding eigenfunctions are
\begin{align*}
& \lambda_k = k^2 & & \text{with eigenfunctions} & &
\cos (k t) \quad \text{and}\quad  \sin (k t)
 & & \text{for all integers } k \geq 1 , \\
& \lambda_0 = 0 & & \text{with an eigenfunction} & & x_0 = 1.
\end{align*}
\end{example}

\subsection{Orthogonality of eigenfunctions}

Something that will be very useful in the next section is the
\emph{\myindex{orthogonality}} property of the eigenfunctions. This is an analogue
of the following fact about eigenvectors of a matrix.  A matrix is
called
\emph{symmetric\index{symmetric matrix}}
if $A = A^T$ (it is equal to its transpose).
\emph{Eigenvectors for two distinct eigenvalues of a symmetric
matrix are orthogonal.}
%That symmetry is required.  
%We will not prove this fact here.
The
differential operators we are dealing with act much like a symmetric matrix.
We, therefore, get the following theorem.

%\medskip
%
%Suppose $\lambda_1$ and $\lambda_2$ are two distinct eigenvalues of $A$
%and $\vec{v}_1$ and $\vec{v}_2$ are the corresponding eigenvectors.  Then
%we of course have that $A \vec{v}_1 = \lambda_1 \vec{v}_1$ and
%$A \vec{v}_2 = \lambda_2 \vec{v}_2$.
%\begin{equation*}
%\langle A \vec{v}_1 , \vec{v}_2 \rangle = \lambda_1 \langle \vec{v}_1 , \vec{v}_2 \rangle
%\qquad
%\langle A \vec{v}_2 , \vec{v}_1 \rangle = \lambda_2 \langle \vec{v}_2 , \vec{v}_1 \rangle
%\end{equation*}
%
%\begin{equation*}
%\langle A \vec{v}_1 , \vec{v}_2 \rangle -
%\langle A \vec{v}_2 , \vec{v}_1 \rangle 
%=
%(\lambda_1 - \lambda_2 ) \langle \vec{v}_1 , \vec{v}_2 \rangle
%\end{equation*}
%
%\begin{equation*}
%\langle (A-A^T) \vec{v}_1 , \vec{v}_2 \rangle
%=
%(\lambda_1 - \lambda_2 ) \langle \vec{v}_1 , \vec{v}_2 \rangle
%\end{equation*}

\begin{theorem} \label{bvp:orthogonaleigen}
Suppose that $x_1(t)$ and $x_2(t)$ are two eigenfunctions of the problem
\eqref{bv:eq1}, \eqref{bv:eq2} or \eqref{bv:eq3}
for two different
eigenvalues $\lambda_1$ and $\lambda_2$.  Then they are
\emph{orthogonal\index{orthogonal!functions}}
in the sense that
\begin{equation*}
\int_a^b x_1(t) x_2(t) \,dt = 0 .
\end{equation*}
\end{theorem}

The terminology comes from the fact that the integral is a type of
inner product.  We will expand on this in the next section.  The theorem
has a very short, elegant, and illuminating proof so let us give it here.
First, we have the following two equations.
\begin{equation*}
x_1'' + \lambda_1 x_1 = 0
\qquad \text{and} \qquad
x_2'' + \lambda_2 x_2 = 0.
\end{equation*}
Multiply the first by $x_2$ and the second by $x_1$ and subtract to get
\begin{equation*}
(\lambda_1 - \lambda_2) x_1 x_2 = x_2'' x_1 - x_2 x_1'' .
\end{equation*}
Now integrate both sides of the equation:
\begin{equation*}
\begin{split}
(\lambda_1 - \lambda_2) \int_a^b x_1 x_2 \,dt
& =
\int_a^b x_2'' x_1 - x_2 x_1'' \,dt \\
& =
\int_a^b \frac{d}{dt} \left( x_2' x_1 - x_2 x_1' \right) \,dt \\
& =
\Bigl[ x_2' x_1 - x_2 x_1' \Bigr]_{t=a}^b
= 0 .
\end{split}
\end{equation*}
The last equality holds because of the boundary conditions.  For example, if
we consider \eqref{bv:eq1} we have $x_1(a) = x_1(b) = x_2(a) = x_2(b) = 0$
and so $x_2' x_1 - x_2 x_1'$ is zero at both $a$ and $b$.
As $\lambda_1 \not= \lambda_2$, the theorem follows.

\begin{exercise}[easy]
Finish the proof of the theorem (check the last equality in the proof) for the cases
\eqref{bv:eq2} and \eqref{bv:eq3}.
\end{exercise}

The function $\sin (n t)$ is an eigenfunction for the problem
$x''+\lambda x = 0$, $x(0) = 0$, $x(\pi) = 0$. 
Hence for positive
integers $n$ and $m$ we have the integrals
\begin{equation*}
\int_{0}^\pi \sin (mt) \sin (nt) \,dt = 0 ,
\quad
\text{when } m \not = n.
\end{equation*}
Similarly,
\begin{equation*}
\int_{0}^\pi \cos (mt) \cos (nt) \,dt = 0 ,
\quad
\text{when } m \not = n,
\qquad \text{and} \qquad
\int_{0}^\pi  \cos (nt) \,dt = 0 .
\end{equation*}
And finally we also get
\begin{equation*}
\int_{-\pi}^\pi \sin (mt) \sin (nt) \,dt = 0 ,
\quad
\text{when } m \not = n, 
\qquad \text{and} \qquad
\int_{-\pi}^\pi  \sin (nt) \,dt = 0 ,
\end{equation*}
\begin{equation*}
\int_{-\pi}^\pi \cos (mt) \cos (nt) \,dt = 0 ,
\quad
\text{when } m \not = n,
\qquad \text{and} \qquad
\int_{-\pi}^\pi  \cos (nt) \,dt = 0 ,
\end{equation*}
and
\begin{equation*}
\int_{-\pi}^\pi \cos (mt) \sin (nt) \,dt = 0 
\qquad \text{(even if $m=n$).}
\end{equation*}

%\medskip
%
%The theorem is also true when different boundary conditions are applied as
%well.  For example, if we require $x'(a) = x'(b) = 0$, or
%$x(a) = x'(b) = 0$, or
%$x'(a) = x(b) = 0$.  See the proof.


%By what we have seen previously we apply the theorem to find the integrals
%\begin{equation*}
%\int_{-\pi}^\pi \sin (mt) \sin (nt) \,dt = 0 \qquad \text{and} \qquad
%\int_{-\pi}^\pi \cos (mt) \cos (nt) \,dt = 0 ,
%\end{equation*}
%when $m \not = n$, and 
%\begin{equation*}
%\int_{-\pi}^\pi \sin (mt) \cos (nt) \,dt = 0 ,
%\end{equation*}
%for all $m$ and $n$.

\subsection{Fredholm alternative}

We now touch on a very useful theorem in the theory of differential
equations.  The theorem holds in a more general setting than we are
going to state it, but for our purposes the following statement is
sufficient.  We will give a slightly more general version in
\chapterref{SL:chapter}.

\begin{theorem}[Fredholm alternative%
\footnote{Named after the Swedish mathematician
\href{https://en.wikipedia.org/wiki/Fredholm}{Erik Ivar Fredholm}
(1866--1927).}]\index{Fredholm alternative!simple case}
\label{thm:fredholmsimple}
Exactly one of the following statements holds.
Either
\begin{equation} \label{simpfredhomeq}
x'' + \lambda x = 0, \quad x(a) = 0, \quad x(b) = 0
\end{equation}
has a nonzero solution, or
\begin{equation} \label{simpfrednonhomeq}
x'' + \lambda x = f(t), \quad x(a) = 0, \quad x(b) = 0
\end{equation}
has a unique solution for every function $f$ continuous on $[a,b]$.
\end{theorem}

The theorem is also true for the other types of
boundary conditions we considered.
The theorem means that if $\lambda$ is not an eigenvalue, the nonhomogeneous
equation \eqref{simpfrednonhomeq} has a unique solution for every right-hand
side.  On the other hand if $\lambda$ is an eigenvalue, then 
\eqref{simpfrednonhomeq} need not have a solution for every $f$,
and furthermore,
even if it happens to have a solution, the solution is not
unique.

We also want to reinforce the idea here that linear differential operators have
much in common with matrices.  So it is no surprise that
there is a finite-dimensional version of Fredholm alternative for matrices as
well.  Let $A$ be an $n \times n$ matrix.  The Fredholm alternative then
states that either $(A-\lambda I) \vec{x}
= \vec{0}$ has a nontrivial solution, or $(A-\lambda I) \vec{x} = \vec{b}$
has a unique solution for every $\vec{b}$.

A lot of intuition from linear algebra can be applied to linear differential
operators, but one must be careful of course.  For example, one 
difference we have already seen is that in general a differential operator
will have infinitely many eigenvalues, while a matrix has only finitely many.

\subsection{Application}

Let us consider a physical application of an endpoint problem.
Suppose we have a tightly stretched quickly spinning elastic
string or rope of uniform linear density $\rho$, for example in
$\unitfrac{kg}{m}$.
Let us put this problem into the $xy$-plane and both $x$ and $y$
are in meters.  The $x$-axis represents the
position on the string.  The string rotates at angular velocity $\omega$,
in $\unitfrac{radians}{s}$.
Imagine that the whole $xy$-plane rotates at angular velocity $\omega$.
This way, the string stays in this $xy$-plane and $y$ 
measures its deflection from the equilibrium position, $y=0$, on the $x$-axis.
Hence the graph of $y$ gives the shape of the string.
We consider an ideal string with
no volume, just a mathematical curve.
We suppose the tension on the string is a constant $T$ in Newtons.
%If we take a small segment and we look at the tension at the endpoints, we
%see that this force is tangential and we will assume that the magnitude is
%the same at both end points.  Hence the magnitude
%is constant everywhere and we will
%call its magnitude $T$.
Assuming that the deflection is small,
we can use Newton's second law (let us skip the derivation) to get the equation
\begin{equation*}
T y'' + \rho \omega^2 y = 0 .
\end{equation*}
To check the units notice that the units of $y''$ are $\unitfrac{m}{m^2}$, as the derivative is
in terms of $x$.

Let $L$ be the length of the string (in meters) and the string
is fixed at the beginning and end
points.  Hence, $y(0) = 0$ and $y(L) = 0$.  See
\figurevref{bvp:whirstringfig}.

\begin{myfig}
\capstart
\inputpdft{bvp-whirstring}
\caption{Whirling string.\label{bvp:whirstringfig}}
\end{myfig}

We rewrite the equation as
$y'' + \frac{\rho \omega^2}{T} y = 0$.
The setup is similar to \examplevref{bvp:eig1ex}, except for the
interval length being $L$ instead of $\pi$.  We are looking for eigenvalues
of $y'' + \lambda y = 0, y(0) = 0, y(L) = 0$ where
$\lambda = \frac{\rho \omega^2}{T}$.  As before
there are no nonpositive eigenvalues.  With $\lambda > 0$,
the general solution to the equation is $y = A \cos (  \sqrt{\lambda} \,x ) + B
\sin ( \sqrt{\lambda} \,x )$.  The condition $y(0) = 0$ implies that $A = 0$ as
before.  The condition $y(L) = 0$ implies that
$\sin ( \sqrt{\lambda} \, L) = 0$ and hence
$\sqrt{\lambda} \, L = k \pi$  for some integer $k > 0$, so
\begin{equation*}
\frac{\rho \omega^2}{T} = \lambda = \frac{k^2 \pi^2}{L^2} .
\end{equation*}

What does this say about the shape of the string?  It says that for
all parameters $\rho$, $\omega$, $T$ not satisfying the equation above, the
string is in the equilibrium position, $y=0$.  When 
$\frac{\rho \omega^2}{T} = \frac{k^2 \pi^2}{L^2}$, then the string will
\myquote{pop out} some distance $B$.  We cannot compute $B$
with the information we have.

Let us assume that $\rho$ and $T$ are fixed and we are changing $\omega$.
For most values of $\omega$, the string is in the equilibrium state.  When 
the angular velocity $\omega$ hits a value
$\omega = \frac{k \pi \sqrt{T}}{L\sqrt{\rho}}$, then the string 
pops out and has the shape of a sin wave crossing the
$x$-axis $k-1$ times between the end points.
For example, at $k=1$, the string does not cross the $x$-axis
and the shape looks like in \figurevref{bvp:whirstringfig}.
On the other hand, when $k=3$ the string crosses the $x$-axis
2 times, see \figurevref{bvp:whirstring2fig}.
When $\omega$ changes again, the string returns to
the equilibrium position.  The higher the angular velocity,
the more times it crosses the $x$-axis when it is popped out.

\begin{myfig}
\capstart
\inputpdft{bvp-whirstring2}
\caption{Whirling string at the third eigenvalue ($k=3$).\label{bvp:whirstring2fig}}
\end{myfig}

For another example, if you have a spinning jump rope (then $k=1$ as it is
completely \myquote{popped out}) and you
pull on the ends to increase the tension, then the velocity also increases
for the rope to stay \myquote{popped out}.


\subsection{Exercises}

Hint for the following exercises:  Note that
if $\lambda > 0$, then
$\cos \bigl( \sqrt{\lambda}\, (t - a) \bigr)$
and $\sin  \bigl( \sqrt{\lambda}\, (t - a) \bigr)$
are also solutions of the homogeneous
equation.

\begin{exercise}
Compute all
eigenvalues and eigenfunctions of
$x'' + \lambda x = 0$, $x(a) = 0$, $x(b) = 0$ (assume $a < b$).
\end{exercise}

\begin{exercise}
Compute all
eigenvalues and eigenfunctions of
$x'' + \lambda x = 0$, $x'(a) = 0$, $x'(b) = 0$ (assume $a < b$).
\end{exercise}

\begin{exercise}
Compute all
eigenvalues and eigenfunctions of
$x'' + \lambda x = 0$, $x'(a) = 0$, $x(b) = 0$ (assume $a < b$).
\end{exercise}

\begin{exercise}
Compute all 
eigenvalues and eigenfunctions of
$x'' + \lambda x = 0$, $x(a) = x(b)$, $x'(a) = x'(b)$ (assume $a < b$).
\end{exercise}

\begin{exercise}
We skipped the case of $\lambda < 0$ for
the boundary value problem
$x'' + \lambda x = 0$, $x(-\pi) = x(\pi)$, $x'(-\pi) = x'(\pi)$.
Finish the calculation and show that there are no negative eigenvalues.
\end{exercise}

\setcounter{exercise}{100}

\begin{exercise}
Consider a spinning string of length 2 and linear density 0.1 and tension 3.
Find smallest angular velocity when the string pops out.
\end{exercise}
\exsol{%
$\omega = \pi \sqrt{\frac{15}{2}}$
}

\begin{exercise}
Suppose $x'' + \lambda x = 0$ and $x(0)=1$, $x(1) = 1$.
Find all $\lambda$ for which there is more
than one solution.  Also find the corresponding solutions (only for the
eigenvalues).
\end{exercise}
\exsol{%
$\lambda_k = 4 k^2 \pi^2$ for $k = 1,2,3,\ldots$
\quad
$x_k =  \cos (2k\pi t) + B \sin (2k\pi t)$ \quad (for any $B$)
}

\begin{exercise}
Suppose $x'' + x = 0$ and $x(0)=0$, $x'(\pi) = 1$.
Find all the solution(s) if any exist.
\end{exercise}
\exsol{%
$x(t) = - \sin(t)$
}

\begin{exercise}
Consider
$x' + \lambda x = 0$ and $x(0)=0$, $x(1) = 0$.  Why does it not
have any eigenvalues?  Why does any first order equation with two endpoint
conditions such as above have no eigenvalues?
\end{exercise}
\exsol{%
General solution is $x = C e^{-\lambda t}$.  Since $x(0) = 0$ then $C=0$, and so $x(t) = 0$.
Therefore,
the solution is always identically zero.  One condition is always
enough to guarantee a unique solution for a first order equation.
}

\begin{exercise}[challenging]
Suppose $x''' + \lambda x = 0$ and $x(0)=0$, $x'(0) = 0$, $x(1) = 0$.
Suppose that $\lambda > 0$.  Find an equation that all such
eigenvalues must satisfy.
Hint: Note that $-\sqrt[3]{\lambda}$ is a root
of $r^3+\lambda = 0$.
\end{exercise}
\exsol{%
$\frac{\sqrt{3}}{3} e^{\frac{-3}{2}\sqrt[3]{\lambda}}
- \frac{\sqrt{3}}{3} \cos \bigl( \frac{\sqrt{3}\, \sqrt[3]{\lambda}}{2} \bigr)
+ \sin \bigl( \frac{\sqrt{3}\, \sqrt[3]{\lambda}}{2}\bigr) = 0$
}

%%%%%%%%%%%%%%%%%%%%%%%%%%%%%%%%%%%%%%%%%%%%%%%%%%%%%%%%%%%%%%%%%%%%%%%%%%%%%%

\sectionnewpage
\section{The trigonometric series} \label{ts:section}

%mbxINTROSUBSECTION

\sectionnotes{2 lectures\EPref{, \S9.1 in \cite{EP}}\BDref{,
\S10.2 in \cite{BD}}}

\subsection{Periodic functions and motivation}

As motivation for studying Fourier series, suppose we have the problem
\begin{equation} \label{ts:deq}
x'' + \omega_0^2 x = f(t) ,
\end{equation}
for some periodic function $f(t)$.
We already solved
\begin{equation} \label{ts:deqcos}
x'' + \omega_0^2 x = F_0 \cos ( \omega t) .
\end{equation}
One way to solve \eqref{ts:deq} is to
decompose $f(t)$ as a sum of cosines (and sines) and then
solve many problems of the form \eqref{ts:deqcos}.  We then use
the principle of superposition, to sum up all the solutions we got
to get a solution to \eqref{ts:deq}.

Before we proceed, let us talk a little bit more in detail about
periodic functions.
A function is said to be \emph{\myindex{periodic}} with period $P$ if
$f(t) = f(t+P)$ for all $t$.  For brevity we say $f(t)$ is $P$-periodic.
Note that a $P$-periodic function is also $2P$-periodic, $3P$-periodic
and so on.
For example, $\cos (t)$ and $\sin (t)$ are
$2\pi$-periodic.  So are $\cos (kt)$ and $\sin (kt)$ for all integers $k$.  The
constant functions are an extreme example.  They are periodic for any period
(exercise).

Normally we start with a function $f(t)$ defined on some interval $[-L,L]$,
and we want to
\emph{extend $f(t)$ periodically}\index{extend periodically}\index{periodic extension}
to make it
a $2L$-periodic function.  We do this extension
by defining a new function $F(t)$
such that for $t$ in $[-L,L]$, $F(t) = f(t)$.  For $t$ in $[L,3L]$,
we define $F(t) = f(t-2L)$, for $t$ in $[-3L,-L]$, $F(t) = f(t+2L)$, and
so on.
To make that work we needed $f(-L) = f(L)$.
We could have also started with $f$
defined only on the half-open interval $(-L,L]$ and then define $f(-L) = f(L)$.

\begin{example}
Define $f(t) = 1-t^2$ on $[-1,1]$.  Now extend $f(t)$ periodically to
a 2-periodic function.  See \figurevref{ts:perextofinvertedparabolafig}.
\begin{myfig}
\capstart
\diffyincludegraphics{width=3in}{width=4.5in}{ts-perextofinvertedparabola}
\caption{Periodic extension of the function
$1-t^2$.\label{ts:perextofinvertedparabolafig}}
\end{myfig}
\end{example}

You should be careful to distinguish between $f(t)$ and its extension.  A common
mistake is to assume that a formula for $f(t)$ holds for its extension.  It
can be confusing when the formula for $f(t)$ is periodic, but with perhaps
a different period.

\begin{exercise}
Define $f(t) = \cos t$ on $[\nicefrac{-\pi}{2},\nicefrac{\pi}{2}]$.  Take the $\pi$-periodic
extension and sketch its graph.  How does it compare to the graph of
$\cos t$?
\end{exercise}

\subsection{Inner product and eigenvector decomposition}

Suppose we have a \emph{\myindex{symmetric matrix}},
that is $A^T = A$.  As we remarked before, 
eigenvectors of $A$ are then orthogonal.  Here the word
\emph{orthogonal}\index{orthogonal!vectors} means
that if $\vec{v}$ and $\vec{w}$ are two 
eigenvectors of $A$ for distinct eigenvalues,
then $\langle \vec{v} , \vec{w} \rangle = 0$.
In this case the inner product $\langle \vec{v} , \vec{w} \rangle$
is the \emph{\myindex{dot product}},
which can be computed as $\vec{v}^T\vec{w}$.

To decompose a vector $\vec{v}$ in terms of mutually orthogonal
vectors $\vec{w}_1$ and $\vec{w}_2$ we write
\begin{equation*}
\vec{v} = a_1 \vec{w}_1  + a_2 \vec{w}_2 .
\end{equation*}
Let us find the formula for $a_1$ and $a_2$.  First let us compute
\begin{equation*}
\langle \vec{v} , \vec{w_1} \rangle
=
\langle a_1 \vec{w}_1  + a_2 \vec{w}_2 , \vec{w_1} \rangle
=
a_1 \langle \vec{w}_1 , \vec{w_1} \rangle
+
a_2 \underbrace{\langle \vec{w}_2 , \vec{w_1} \rangle}_{=0}
=
a_1 \langle \vec{w}_1 , \vec{w_1} \rangle .
\end{equation*}
Therefore,
\begin{equation*}
a_1 = 
\frac{\langle \vec{v} , \vec{w_1} \rangle}{
\langle \vec{w}_1 , \vec{w_1} \rangle} .
\end{equation*}
Similarly
\begin{equation*}
a_2 = 
\frac{\langle \vec{v} , \vec{w_2} \rangle}{
\langle \vec{w}_2 , \vec{w_2} \rangle} .
\end{equation*}
You probably remember this formula from vector calculus.

\begin{example}
Write
$\vec{v} = \left[ \begin{smallmatrix} 2 \\ 3 \end{smallmatrix} \right]$
as a linear combination of 
$\vec{w_1} = \left[ \begin{smallmatrix} 1 \\ -1 \end{smallmatrix} \right]$
and
$\vec{w_2} = \left[ \begin{smallmatrix} 1 \\ 1 \end{smallmatrix} \right]$.

First note that $\vec{w}_1$ and $\vec{w}_2$ are orthogonal
as $\langle \vec{w}_1 , \vec{w}_2 \rangle = 1(1) + (-1)1 = 0$.
Then
\begin{align*}
& a_1 = 
\frac{\langle \vec{v} , \vec{w_1} \rangle}{
\langle \vec{w}_1 , \vec{w_1} \rangle}
=
\frac{2(1) + 3(-1)}{1(1) + (-1)(-1)} = \frac{-1}{2} ,
\\
& a_2 = 
\frac{\langle \vec{v} , \vec{w_2} \rangle}{
\langle \vec{w}_2 , \vec{w_2} \rangle}
=
\frac{2 + 3}{1 + 1} = \frac{5}{2} .
\end{align*}
Hence
\begin{equation*}
\begin{bmatrix} 2 \\ 3 \end{bmatrix}
=
\frac{-1}{2}
\begin{bmatrix} 1 \\ -1 \end{bmatrix}
+
\frac{5}{2}
\begin{bmatrix} 1 \\ 1 \end{bmatrix} .
\end{equation*}
\end{example}

\subsection{The trigonometric series}

Instead of decomposing a vector in terms of eigenvectors of a matrix,
we decompose a function in terms of eigenfunctions of a certain
eigenvalue problem.  The eigenvalue problem we use for
the Fourier series is 
\begin{equation*}
x'' + \lambda x = 0, \quad x(-\pi) = x(\pi), \quad x'(-\pi) = x'(\pi) .
\end{equation*}
We computed that eigenfunctions are 1, $\cos (k t)$,
$\sin (k t)$.  That is, we want to find a representation of a
$2\pi$-periodic function $f(t)$ as
\begin{equation*}
\mybxbg{~~
f(t) = \frac{a_0}{2} +
\sum_{n=1}^\infty a_n \cos (n t) + b_n \sin (n t) .
~~}
\end{equation*}
This series is called the \emph{\myindex{Fourier series}}%
\footnote{Named after the French mathematician
\href{https://en.wikipedia.org/wiki/Joseph_Fourier}{Jean Baptiste Joseph Fourier}
(1768--1830).} or the
\emph{\myindex{trigonometric series}} for $f(t)$.
We write the coefficient of the eigenfunction 1 as $\frac{a_0}{2}$
for convenience.
We could also think of $1 = \cos (0t)$, so that
we only need to look at $\cos (kt)$ and $\sin (kt)$.

As for matrices we want to find a \emph{\myindex{projection}}
of $f(t)$ onto the subspaces given by the eigenfunctions.  So we want to
define an \emph{\myindex{inner product of functions}}.  For example, to
find $a_n$ we want to compute $\langle \, f(t) \, , \, \cos (nt) \, \rangle$.
We define the inner product as
\begin{equation*}
\langle \, f(t)\, , \, g(t) \, \rangle \overset{\text{def}}{=}
\int_{-\pi}^\pi f(t) \, g(t) \, dt .
\end{equation*}
With this definition of the inner product,
we saw in the previous section that the eigenfunctions $\cos (kt)$
(including the constant eigenfunction), and
$\sin (kt)$ are \emph{orthogonal\index{orthogonal!functions}} in the sense
that
\begin{align*}
\langle \, \cos (mt)\, , \, \cos (nt) \, \rangle = 0 & \qquad \text{for } m \not= n , \\
\langle \, \sin (mt)\, , \, \sin (nt) \, \rangle = 0 & \qquad \text{for } m \not= n , \\
\langle \, \sin (mt)\, , \, \cos (nt) \, \rangle = 0 & \qquad \text{for all } m \text{ and } n .
\end{align*}
For $n=1,2,3,\ldots$
we have
\begin{align*}
\langle \, \cos (nt) \, , \, \cos (nt) \, \rangle &=
\int_{-\pi}^\pi \cos(nt)\cos(nt) \, dt
=
\pi,
\\
\langle \, \sin (nt) \, , \, \sin (nt) \, \rangle &=
\int_{-\pi}^\pi \sin(nt)\sin(nt) \, dt
=
\pi,
\end{align*}
by elementary calculus.  For the constant we get
\begin{equation*}
\langle \, 1 \, , \, 1 \, \rangle
=
\int_{-\pi}^\pi 1 \cdot 1 \, dt
 = 2\pi.
\end{equation*}
The coefficients are given by
\begin{equation*}
\mybxbg{~~
\begin{aligned}
& a_n =
\frac{\langle \, f(t) \, , \, \cos (nt) \, \rangle}{\langle \, \cos (nt) \, , \,
\cos (nt) \, \rangle}
= 
\frac{1}{\pi} \int_{-\pi}^\pi f(t) \cos (nt) \, dt , \\
& b_n =
\frac{\langle \, f(t) \, , \, \sin (nt) \, \rangle}{\langle \, \sin (nt) \, , \,
\sin (nt) \, \rangle}
= 
\frac{1}{\pi} \int_{-\pi}^\pi f(t) \sin (nt) \, dt .
\end{aligned}
~~}
\end{equation*}
Compare these expressions with the finite-dimensional example.
For $a_0$ we get a similar formula
\begin{equation*}
\mybxbg{~~
a_0 = 2
\frac{\langle \, f(t) \, , \, 1 \, \rangle}{\langle \, 1 \, , \,
1 \, \rangle}
=
\frac{1}{\pi} \int_{-\pi}^\pi f(t) \, dt .
~~}
\end{equation*}

Let us check the formulas using the orthogonality properties.  Suppose for
a moment that
\begin{equation*}
f(t) = \frac{a_0}{2} + \sum_{n=1}^\infty a_n \cos (n t) + b_n
\sin (n t) .
\end{equation*}
Then for $m \geq 1$ we have
\begin{equation*}
\begin{split}
\langle \, f(t)\,,\,\cos (mt) \, \rangle
& =
\Bigl\langle \, \frac{a_0}{2} + \sum_{n=1}^\infty a_n \cos (n t) + b_n
\sin (n t) \,,\, \cos (mt) \, \Bigr\rangle \\
& =
\frac{a_0}{2}
\langle \, 1 \, , \, \cos (mt) \, \rangle
+ \sum_{n=1}^\infty
a_n \langle \, \cos (nt) \, , \, \cos (mt) \, \rangle +
b_n \langle \, \sin (n t) \, , \, \cos (mt) \, \rangle \\
& =
a_m \langle \, \cos (mt) \, , \, \cos (mt) \, \rangle .
\end{split}
\end{equation*}
And hence
$a_m =
\frac{\langle \, f(t) \, , \, \cos (mt) \, \rangle}{\langle \, \cos (mt) \, , \,
\cos (mt) \, \rangle}$.

\begin{exercise}
Carry out the calculation for $a_0$ and $b_m$.
\end{exercise}

\begin{example}
Take the function
\begin{equation*}
f(t) = t
\end{equation*}
for $t$ in $(-\pi,\pi]$.  Extend $f(t)$ periodically and write it 
as a Fourier series.  This function is called the \emph{\myindex{sawtooth}}.

\begin{myfig}
\capstart
\diffyincludegraphics{width=3in}{width=4.5in}{ts-sawtooth}
\caption{The graph of the sawtooth function.\label{ts:sawtoothfig}}
\end{myfig}

The plot of the extended periodic function is given in
\figurevref{ts:sawtoothfig}.
Let us compute the coefficients.  We start with $a_0$,
\begin{equation*}
a_0 = \frac{1}{\pi} \int_{-\pi}^\pi t \,dt = 0 .
\end{equation*}
We will often use the result from calculus that says that the integral of an odd
function over a symmetric interval is zero.  Recall that an
\emph{\myindex{odd function}} is a
function $\varphi(t)$ such that $\varphi(-t) = -\varphi(t)$.  For example
the functions $t$, $\sin t$, or (importantly for us)
$t \cos (nt)$ are all odd functions.  Thus
\begin{equation*}
a_n = \frac{1}{\pi} \int_{-\pi}^\pi t \cos (nt) \,dt = 0 .
\end{equation*}
Let us move to $b_n$.  Another useful fact from calculus
is that the integral of an even function over
a symmetric interval is
twice the integral of the same function over half the interval.  
Recall an \emph{\myindex{even function}}
is a
function $\varphi(t)$ such that $\varphi(-t) = \varphi(t)$.  For example
$t \sin (nt)$ is even.
\begin{equation*}
\begin{split}
b_n & = \frac{1}{\pi} \int_{-\pi}^\pi t \sin (nt) \,dt \\
& = \frac{2}{\pi} \int_{0}^\pi t \sin (nt) \,dt \\
& = \frac{2}{\pi} \left(
\left[ \frac{-t \cos (nt)}{n} \right]_{t=0}^{\pi}
+
\frac{1}{n}
\int_{0}^\pi \cos (nt) \,dt
\right)
\\
& = \frac{2}{\pi} \left(
\frac{-\pi \cos (n\pi)}{n}
+
0
\right) \\
& =  \frac{-2 \cos (n\pi)}{n}
=  \frac{2 \,{(-1)}^{n+1}}{n} .
\end{split}
\end{equation*}
We have used the fact that 
\begin{equation*}
\cos (n\pi) = {(-1)}^n =
\begin{cases}
1 & \text{if } n \text{ even} , \\
-1 & \text{if } n \text{ odd} .
\end{cases}
\end{equation*}
The series, therefore, is
\begin{equation*}
\sum_{n=1}^\infty
\frac{2 \,{(-1)}^{n+1}}{n} \,
\sin (n t) .
\end{equation*}

Let us write out the first 3 harmonics of the series for $f(t)$.
\begin{equation*}
2 \, \sin (t)
- \sin (2t)
+\frac{2}{3} \sin (3t)
+ \cdots
\end{equation*}
The plot of these first three terms of the series, along with a plot
of the first 20 terms is given in
\figurevref{ts:sawtoothfsfig}.

\begin{myfig}
\capstart
%original files ts-sawtooth-fs3 ts-sawtooth-fs20
\diffyincludegraphics{width=6.24in}{width=9in}{ts-sawtooth-fs3-fs20}
\caption{First 3 (left graph) and 20 (right graph) harmonics of the sawtooth
function.\label{ts:sawtoothfsfig}}
\end{myfig}
\end{example}

\begin{example}
Take the function
\begin{equation*}
f(t) =
\begin{cases}
0 & \text{if } \;{-\pi} < t \leq 0 , \\
\pi & \text{if } \;\phantom{-}0 < t \leq \pi .
\end{cases}
\end{equation*}
\nopagebreak[4]%
Extend $f(t)$ periodically and write it 
as a Fourier series.  This function or its variants appear often
in applications and the function is called the
\emph{\myindex{square wave}}.

\begin{myfig}
\capstart
\diffyincludegraphics{width=3in}{width=4.5in}{ts-squarewave}
\caption{The graph of the square wave function.\label{ts:squarewavefig}}
\end{myfig}

The plot of the extended periodic function is given in
\figurevref{ts:squarewavefig}.
Now we compute the coefficients.  We start with $a_0$
\begin{equation*}
a_0 = \frac{1}{\pi} \int_{-\pi}^\pi f(t) \,dt
= \frac{1}{\pi} \int_{0}^\pi \pi \,dt = \pi .
\end{equation*}
Next,
\begin{equation*}
a_n = \frac{1}{\pi} \int_{-\pi}^\pi f(t) \cos (nt) \,dt 
= \frac{1}{\pi} \int_{0}^\pi \pi \cos (nt) \,dt = 0 .
\end{equation*}
And finally,
\begin{equation*}
\begin{split}
b_n & = \frac{1}{\pi} \int_{-\pi}^\pi f(t) \sin (nt) \,dt \\
& = \frac{1}{\pi} \int_{0}^\pi \pi \sin (nt) \,dt \\
& = \left[ \frac{- \cos (nt)}{n} \right]_{t=0}^\pi \\
& = \frac{1 - \cos (\pi n)}{n}
= \frac{1 - {(-1)}^n}{n}
=
\begin{cases}
\frac{2}{n} & \text{if } n \text{ is odd} , \\
0 & \text{if } n \text{ is even} .
\end{cases}
\end{split}
\end{equation*}
The Fourier series is
\begin{equation*}
\frac{\pi}{2} +  \sum_{\substack{n=1\\n \text{ odd}}}^\infty
\frac{2}{n} 
\sin (n t)
=
\frac{\pi}{2} + \sum_{k=1}^\infty
\frac{2}{2k-1} 
\sin \bigl( (2k-1)\, t \bigr) .
\end{equation*}

Let us write out the first 3 harmonics of the series for $f(t)$:
\begin{equation*}
\frac{\pi}{2}
+
2 \, \sin (t)
+
\frac{2}{3}  \sin (3t)
+ \cdots
\end{equation*}
The plot of these first three and also of the first 20 terms of the series
is given in
\figurevref{ts:squarewavefsfig}.

\begin{myfig}
\capstart
%original files ts-squarewave-fs3 ts-squarewave-fs20
\diffyincludegraphics{width=6.24in}{width=9in}{ts-squarewave-fs3-fs20}
\caption{First 3 (left graph) and 20 (right graph) harmonics of the
square wave function.\label{ts:squarewavefsfig}}
\end{myfig}
\end{example}

We have so far skirted the issue of convergence.  For example,
if $f(t)$ is the square wave function,
the equation
\begin{equation*}
f(t) = 
\frac{\pi}{2} + \sum_{k=1}^\infty
\frac{2}{2k-1} 
\sin \bigl( (2k-1)\, t \bigr) .
\end{equation*}
is only an equality for such $t$ where $f(t)$ is continuous.
We do not get an equality for $t=-\pi,0,\pi$ and all the other discontinuities
of $f(t)$.  It is not hard to see that when $t$ is an integer multiple of
$\pi$ (which gives all the discontinuities), then
\begin{equation*}
\frac{\pi}{2} + \sum_{k=1}^\infty
\frac{2}{2k-1} 
\sin \bigl( (2k-1)\, t \bigr) = \frac{\pi}{2} .
\end{equation*}
We redefine $f(t)$ on $[-\pi,\pi]$ as
\begin{equation*}
f(t) =
\begin{cases}
0 & \text{if } \; {-\pi} < t < 0 , \\
\pi & \text{if } \; \phantom{-}0 < t < \pi , \\
\nicefrac{\pi}{2} & \text{if } \; \phantom{-}t = -\pi, 
t = 0,\text{ or }
t = \pi,
\end{cases}
\end{equation*}
and extend periodically.
The series equals this new extended $f(t)$ everywhere, including the
discontinuities.
We will generally not worry about changing the function values
at several (finitely many) points.

We will say more about convergence in the next section.  Let us, however,
briefly mention an effect of the discontinuity.  Zoom in near the
discontinuity in the square wave.  Further, plot the first 100
harmonics, see
\figurevref{ts:squarewavegibbsfig}.  While the
series is a very good approximation away from the discontinuities, the error
(the overshoot)
near the discontinuity at $t=\pi$ does not seem to be getting any smaller
as we take more and more harmonics.
This behavior is known as the \emph{\myindex{Gibbs phenomenon}}.
The region where the error is large does get smaller, however, the more
terms in the series we take.

\begin{myfig}
\capstart
\diffyincludegraphics{width=3in}{width=4.5in}{ts-squarewave-gibbs}
\caption{Gibbs phenomenon in action.\label{ts:squarewavegibbsfig}}
\end{myfig}

\medskip

We can think of a periodic function as a \myquote{signal} being a superposition of
many signals of pure frequency.  For example, we could think of the square
wave as a tone of certain base frequency.
This base frequency is called the
\emph{\myindex{fundamental frequency}}.
The square wave
will be
a superposition of
many different pure tones of frequencies that are multiples of the
fundamental frequency.
In music,
the higher frequencies are called the \emph{\myindex{overtones}}.
All the frequencies that appear are called the
\emph{\myindex{spectrum}} of the signal.
On the other hand a simple sine wave is only the pure tone (no overtones).  The
simplest way to make sound using a computer is the square wave, and the sound
is very different from a pure tone.  If you ever played video games
from the 1980s or so, then you heard what square waves sound like.

\subsection{Exercises}

\begin{exercise}
Suppose $f(t)$ is defined on $[-\pi,\pi]$ as $\sin (5t) + \cos (3t)$.  Extend
periodically and compute the Fourier series of $f(t)$.
\end{exercise}

\begin{exercise}
Suppose $f(t)$ is defined on $[-\pi,\pi]$ as $\lvert t \rvert$.
Extend periodically and compute the Fourier series of $f(t)$.
\end{exercise}

\begin{exercise}
Suppose $f(t)$ is defined on $[-\pi,\pi]$ as $\lvert t \rvert^3$.
Extend periodically and compute the Fourier series of $f(t)$.
\end{exercise}

\begin{exercise}
Suppose $f(t)$ is defined on $(-\pi,\pi]$ as
\begin{equation*}
f(t) =
\begin{cases}
-1 & \text{if } \; {-\pi} < t \leq 0 , \\
1 & \text{if } \; \phantom{-}0 < t \leq \pi .
\end{cases}
\end{equation*}
Extend periodically and compute the Fourier series of $f(t)$.
\end{exercise}

\begin{exercise}
Suppose $f(t)$ is defined on $(-\pi,\pi]$ as $t^3$.
Extend periodically and compute the Fourier series of $f(t)$.
\end{exercise}

\begin{exercise}
Suppose $f(t)$ is defined on $[-\pi,\pi]$ as $t^2$.
Extend periodically and compute the Fourier series of $f(t)$.
\end{exercise}

There is another form of the Fourier series using complex exponentials
$e^{nt}$ for $n=\ldots,-2,-1,0,1,2,\ldots$ instead of
$\cos(nt)$ and $\sin(nt)$ for positive $n$.  This form may be easier
to work with sometimes.  It is certainly more compact to write,
and there is only one formula for the coefficients.  On the downside,
the coefficients are complex numbers.

\begin{exercise}
\begin{samepage}
Let 
\begin{equation*}
f(t) = \frac{a_0}{2} + \sum_{n=1}^\infty a_n \cos (n t)
+ b_n \sin (n t) .
\end{equation*}
Use Euler's formula $e^{i\theta} = \cos (\theta) + i \sin (\theta)$ to
show that there exist complex numbers $c_m$ such that
\begin{equation*}
f(t) = 
\sum_{m=-\infty}^\infty c_m e^{imt} .
\end{equation*}
Note that the sum now ranges over all the integers including negative ones.
Do not worry about convergence in this calculation.
Hint: It may be better to start from the complex exponential form and write
the series as
\begin{equation*}
c_0 + \sum_{m=1}^\infty \Bigl( c_m e^{imt} + c_{-m} e^{-imt}  \Bigr).
\end{equation*}
\end{samepage}
\end{exercise}

\setcounter{exercise}{100}

\begin{exercise}
Suppose $f(t)$ is defined on $[-\pi,\pi]$ as $f(t) = \sin(t)$.  Extend
periodically and compute the Fourier series.
\end{exercise}
\exsol{%
$\sin(t)$
}

\begin{exercise}
Suppose $f(t)$ is defined on $(-\pi,\pi]$ as $f(t) = \sin(\pi t)$.  Extend
periodically and compute the Fourier series.
\end{exercise}
\exsol{%
$\sum\limits_{n=1}^\infty
\frac{(\pi-n) \sin( \pi n+{\pi}^{2})
+(\pi+n)\sin(\pi n-{\pi}^{2}) }{\pi {n}^{2}-{\pi}^{3}}
\sin(nt)$
%$a_0 = \frac{1}{\pi} \int_{-\pi}^\pi \sin(\pi t) \, dt$
%\\
%$a_n =
%\frac{1}{\pi} \int_{-\pi}^\pi \sin(\pi t) \cos (nt) \, dt$
%\\
%$b_n =
%\frac{1}{\pi} \int_{-\pi}^\pi f(t) \sin (nt) \, dt$
}

\begin{exercise}
Suppose $f(t)$ is defined on $(-\pi,\pi]$ as $f(t) = \sin^2(t)$.
Extend
periodically and compute the Fourier series.
\end{exercise}
\exsol{%
$\frac{1}{2}-\frac{1}{2}\cos(2t)$
}

\begin{exercise}
Suppose $f(t)$ is defined on $(-\pi,\pi]$ as $f(t) = t^4$.
Extend periodically and compute the Fourier series.
\end{exercise}
\exsol{%
$\frac{\pi^4}{5} + \sum\limits_{n=1}^\infty
\frac{{(-1)}^{n} (8{\pi}^{2}{n}^{2}-48) }{{n}^{4}}
\cos(nt)$
}

%%%%%%%%%%%%%%%%%%%%%%%%%%%%%%%%%%%%%%%%%%%%%%%%%%%%%%%%%%%%%%%%%%%%%%%%%%%%%%

\sectionnewpage
\section{More on the Fourier series}
\label{moreonfourier:section}

%mbxINTROSUBSECTION

\sectionnotes{2 lectures\EPref{, \S9.2--\S9.3 in \cite{EP}}\BDref{,
\S10.3 in \cite{BD}}}

%Before reading the lecture, it may be good to first try
%Project IV (Fourier series)\index{IODE software!Project IV} from the
%IODE website: \url{http://www.math.uiuc.edu/iode/}.  After reading the
%lecture it may be good to continue with 
%Project V (Fourier series again)\index{IODE software!Project V}.

\subsection{$2L$-periodic functions}

We have computed the Fourier series for a $2\pi$-periodic function, but what
about functions of different periods.  Well, fear not, the computation is a
simple case of change of variables.  We just rescale the independent
axis.  Suppose we have a $2L$-periodic function $f(t)$.  Then $L$ is called
the \emph{\myindex{half period}}.  Let $s = \frac{\pi}{L}  t$.
Then the function
\begin{equation*}
g(s) = f\left(\frac{L}{\pi} s \right)
\end{equation*}
is $2\pi$-periodic.  We must also rescale all our sines and cosines.
In the series we use $\frac{\pi}{L} t$ as the variable.  That is, we
want to write
\begin{equation*}
\mybxbg{~~
f(t) = 
\frac{a_0}{2} +
\sum_{n=1}^\infty a_n \cos \left( \frac{n \pi}{L} t \right)
+ b_n \sin \left(\frac{n \pi}{L} t \right) .
~~}
\end{equation*}
If we change variables to $s$, we see that
\begin{equation*}
g(s) = 
\frac{a_0}{2} +
\sum_{n=1}^\infty a_n \cos (n s)
+ b_n \sin (n s) .
\end{equation*}
We compute $a_n$ and $b_n$ as before.  After we write down the
integrals, we change variables from $s$ back to $t$, noting also
that $ds = \frac{\pi}{L} \, dt$.
\begin{equation*}
\mybxbg{~~
\begin{aligned}
& a_0 =
\frac{1}{\pi}
\int_{-\pi}^\pi
g(s) \, ds
=
\frac{1}{L}
\int_{-L}^L
f(t) \, dt , \\
& a_n =
\frac{1}{\pi}
\int_{-\pi}^\pi
g(s) \, \cos (n s) \, ds
=
\frac{1}{L}
\int_{-L}^L
f(t) \, \cos \left( \frac{n \pi}{L} t \right) \, dt , \\
& b_n =
\frac{1}{\pi}
\int_{-\pi}^\pi
g(s) \, \sin (n s) \, ds
=
\frac{1}{L}
\int_{-L}^L
f(t) \, \sin \left( \frac{n \pi}{L} t \right) \, dt .
\end{aligned}
~~}
\end{equation*}

The two most common half periods that show up in examples
are $\pi$ and 1 because of the simplicity of the formulas.  We should stress that we have
done no new mathematics, we have only changed variables.  If you understand 
the Fourier series for $2\pi$-periodic functions, you understand it for
$2L$-periodic functions.  You can think of it as just using
different units for time.  All that we are doing is moving some constants
around, but all the mathematics is the same.

\begin{example}
Let
\begin{equation*}
f(t) =
\lvert t \rvert
\qquad \text{for } \; {-1} < t \leq 1,
\end{equation*}
extended periodically.  The plot of the
periodic extension is given in \figurevref{gfs:sawcontfig}.
Compute the Fourier series of $f(t)$.

\begin{myfig}
\capstart
\diffyincludegraphics{width=3in}{width=4.5in}{gfs-sawcont}
\caption{Periodic extension of the function $f(t)$.\label{gfs:sawcontfig}}
\end{myfig}

We want to
write $f(t) = \frac{a_0}{2} + \sum_{n=1}^\infty a_n \cos (n \pi t) + b_n
\sin (n \pi t)$.  For $n \geq 1$ we note that $\lvert t \rvert \cos (n \pi t)$
is even and hence
\begin{equation*}
\begin{split}
a_n & = \int_{-1}^1 f(t) \cos (n \pi t) \, dt \\
& = 2 \int_{0}^1 t \cos (n \pi t) \, dt \\
 & = 2 \left[ \frac{t}{n \pi} \sin (n \pi t) \right]_{t=0}^1 -
2 \int_{0}^1 \frac{1}{n \pi} \sin (n \pi t) \, dt \\
& =  0 + \frac{1}{n^2 \pi^2} \Bigl[ \cos (n \pi t) \Bigr]_{t=0}^1
 =  \frac{2 \bigl( {(-1)}^n -1 \bigr) }{n^2 \pi^2}
=
\begin{cases}
0 & \text{if } n \text{ is even} , \\
\frac{-4 }{n^2 \pi^2} & \text{if } n \text{ is odd}  .
\end{cases}
\end{split}
\end{equation*}
Next we find $a_0$:
\begin{equation*}
a_0 = \int_{-1}^1 \lvert t \rvert \, dt 
=
1 .
\end{equation*}
You should be able to find this integral by thinking about the integral
as the area under the graph without doing any computation at all.
Finally we can find $b_n$.  Here, we notice that
$\lvert t \rvert \sin (n \pi t)$ is odd and, therefore,
\begin{equation*}
b_n = \int_{-1}^1 f(t) \sin (n \pi t) \, dt = 0 .
\end{equation*}
Hence,
the series is 
\begin{equation*}
\frac{1}{2} + 
\sum_{\substack{n=1 \\ n \text{ odd}}}^\infty \frac{-4}{n^2 \pi^2} \cos (n \pi t) .
\end{equation*}

Let us explicitly write down the first few terms of the series
up to the $3^{\text{rd}}$ harmonic.
\begin{equation*}
\frac{1}{2} -
\frac{4}{\pi^2} \cos (\pi t)
-
\frac{4}{9 \pi^2} \cos (3 \pi t)
- \cdots
\end{equation*}
The plot of these few terms and also a plot up to the ${20}^{\text{th}}$
harmonic is given in
\figurevref{gfs:sawcontfsfig}.  You should notice how close the graph is
to the real function.  You should also notice that there is no
\myquote{Gibbs phenomenon} present as there are no discontinuities.

\begin{myfig}
\capstart
%original files gfs-sawcontfs3 gfs-sawcont-fs20
\diffyincludegraphics{width=6.24in}{width=9in}{gfs-sawcont-fs3-fs20}
\caption{Fourier series of $f(t)$ up to the $3^{\text{rd}}$ harmonic (left
graph)
and up to the ${20}^{\text{th}}$ harmonic (right graph).\label{gfs:sawcontfsfig}}
\end{myfig}
\end{example}

\subsection{Convergence}

We will need the one sided limits of functions.
We will use the following notation
\begin{equation*}
f(c-) = \lim_{t \uparrow c} f(t),
\qquad \text{and} \qquad
f(c+) = \lim_{t \downarrow c} f(t).
\end{equation*}
If you are unfamiliar with this notation,
$\lim_{t \uparrow c} f(t)$ means we are taking a limit of $f(t)$
as $t$ approaches $c$ from below (i.e.\ $t < c$) and
$\lim_{t \downarrow c} f(t)$ means we are taking a limit of $f(t)$
as $t$ approaches $c$ from above (i.e.\ $t > c$).
For example, for the square wave function
\begin{equation} \label{gfs:sqwaveeq}
f(t) =
\begin{cases}
0 & \text{if } \; {-\pi} < t \leq 0 , \\
\pi & \text{if } \; \phantom{-}0 < t \leq \pi ,
\end{cases}
\end{equation}
we have $f(0-) = 0$ and $f(0+) = \pi$.

Let $f(t)$ be a function defined on an interval $[a,b]$.  Suppose
that we find finitely many points
$a=t_0$, $t_1$, $t_2$, \ldots, $t_k=b$ in
the interval, such that $f(t)$ is continuous
on the intervals
$(t_0,t_1)$, 
$(t_1,t_2)$, \ldots, 
$(t_{k-1},t_k)$.
Also suppose that all the one sided limits exist, that is,
all of
$f(t_0+)$,
$f(t_1-)$,
$f(t_1+)$,
$f(t_2-)$,
$f(t_2+)$,
\ldots,
$f(t_k-)$
exist and are finite.
Then
we say $f(t)$ is \emph{\myindex{piecewise continuous}}.

If moreover, $f(t)$ is differentiable at all but finitely many points,
and $f'(t)$ is piecewise continuous, then 
$f(t)$ is said to be \emph{\myindex{piecewise smooth}}.

\begin{example}
The square wave function \eqref{gfs:sqwaveeq}
is piecewise smooth on $[-\pi,\pi]$ or any other interval.  In such a
case we simply say that the function is piecewise smooth.
\end{example}

\begin{example}
The function $f(t) = \lvert t \lvert$
is piecewise smooth.
\end{example}

\begin{example}
The function $f(t) = \frac{1}{t}$ is not piecewise smooth on
$[-1,1]$ (or any other interval containing zero).  In fact, it is not
even piecewise continuous.
\end{example}

\begin{example}
The function $f(t) = \sqrt[3]{t}$ is not piecewise smooth on
$[-1,1]$ (or any other interval containing zero).  $f(t)$ is continuous, but
the derivative of $f(t)$ is unbounded near zero and hence not piecewise
continuous.
\end{example}

Piecewise smooth functions have an easy answer on the convergence
of the Fourier series.

\begin{theorem}
Suppose $f(t)$ is a $2L$-periodic piecewise smooth function.
Let
\begin{equation*}
\frac{a_0}{2} + \sum_{n=1}^\infty a_n \cos \left( \frac{n \pi}{L} t
\right)
+ b_n \sin \left( \frac{n \pi}{L} t \right)
\end{equation*}
be the Fourier series for $f(t)$.  Then the series converges
for all $t$.  If $f(t)$ is continuous
at $t$, then
\begin{equation*}
f(t) = \frac{a_0}{2} + \sum_{n=1}^\infty
a_n \cos \left( \frac{n \pi}{L} t \right)
+ b_n \sin \left( \frac{n \pi}{L} t \right) .
\end{equation*}
Otherwise,
\begin{equation*}
\frac{f(t-)+f(t+)}{2} =
\frac{a_0}{2} + \sum_{n=1}^\infty a_n \cos \left( \frac{n \pi}{L}  t
\right)
+ b_n \sin \left( \frac{n \pi}{L} t \right) .
\end{equation*}
\end{theorem}

If we happen to have that
$f(t) = \frac{f(t-)+f(t+)}{2}$ at all the discontinuities, the Fourier series
converges to $f(t)$ everywhere.  We can always just redefine $f(t)$
by changing the value at each discontinuity appropriately.  Then we can write
an equals sign between $f(t)$ and the series without any worry.
We mentioned this fact
briefly at the end last section.

The theorem does not say how fast the series converges.
Think back to the discussion of the Gibbs phenomenon in the last section.
The closer you get to the discontinuity, the more terms you need to take
to get an accurate approximation to the function.

\subsection{Differentiation and integration of Fourier series}

Not only does Fourier series converge nicely, but it is easy to differentiate
and integrate the series.  We can do this just by differentiating or
integrating term by term.

\begin{theorem}
Suppose
\begin{equation*}
f(t) = \frac{a_0}{2} + \sum_{n=1}^\infty a_n \cos \left( \frac{n \pi}{L} t
\right)
+ b_n \sin \left( \frac{n \pi}{L} t \right)
\end{equation*}
is a piecewise smooth continuous function and the derivative $f'(t)$ is
piecewise smooth.  Then the derivative can be
obtained by differentiating term by term,
\begin{equation*}
f'(t) = \sum_{n=1}^\infty \frac{-a_n n \pi}{L} 
\sin \left( \frac{n \pi}{L} t \right)
+ \frac{b_n n \pi}{L} \cos \left( \frac{n \pi}{L} t \right) .
\end{equation*}
\end{theorem}

It is important that the function is continuous.  It can have corners, but no
jumps.  Otherwise, the differentiated series will fail to converge.  For an
exercise, take the series obtained for the square wave and try to
differentiate the series.  Similarly, we can also integrate a Fourier series.

\begin{theorem}
Suppose
\begin{equation*}
f(t) = \frac{a_0}{2} + \sum_{n=1}^\infty
a_n \cos \left( \frac{n \pi}{L} t \right)
+ b_n \sin \left( \frac{n \pi}{L} t \right)
\end{equation*}
is a piecewise smooth function.  Then the antiderivative is
obtained by antidifferentiating term by term and so
\begin{equation*}
F(t) = \frac{a_0 t}{2} + C + \sum_{n=1}^\infty
\frac{a_n L}{n \pi} \sin \left( \frac{n \pi}{L} t \right)
+ \frac{-b_n L}{n \pi}  \cos \left( \frac{n \pi}{L} t \right) ,
\end{equation*}
where $F'(t) = f(t)$ and $C$ is an arbitrary constant.
\end{theorem}

Note that the series for $F(t)$ is no longer a Fourier series as it contains
the $\frac{a_0 t}{2}$ term.  The antiderivative
of a periodic function need no longer be periodic and so we should not
expect a Fourier series.

\subsection{Rates of convergence and smoothness}

Let us do an example of a periodic function with one derivative everywhere.

\begin{example}
Take the function
\begin{equation*}
f(t) =
\begin{cases}
(t+1)\,t & \text{if } \; {-1} < t \leq 0 , \\
(1-t)\,t & \text{if } \; \phantom{-}0 < t \leq 1 ,
\end{cases}
\end{equation*}
and extend to a
2-periodic function.  The plot is given in
\figurevref{gfs:smoothexfig}.

\begin{myfig}
\capstart
\diffyincludegraphics{width=3in}{width=4.5in}{gfs-smoothex}
\caption{Smooth 2-periodic function.\label{gfs:smoothexfig}}
\end{myfig}

This function has one derivative everywhere, but it
does not have a second derivative whenever $t$ is an integer.

\begin{exercise}
Compute $f''(0+)$ and $f''(0-)$.
\end{exercise}

Let us compute the Fourier series coefficients.  The actual computation
involves several integration by parts and is left to student.
\begin{align*}
a_0 & = 
\int_{-1}^1
f(t) \, dt = 
\int_{-1}^0
(t+1)\,t \, dt +
\int_0^1
(1-t)\,t \, dt = 0 , \\
a_n & = 
\int_{-1}^1
f(t) \, \cos (n\pi t) \, dt = 
\int_{-1}^0
(t+1)\,t
\, \cos (n \pi t) \, dt +
\int_0^1
(1-t)\,t
\, \cos (n \pi t) \, dt = 0, \\
b_n & = 
\int_{-1}^1
f(t) \, \sin (n\pi t) \, dt = 
\int_{-1}^0
(t+1)\,t
\, \sin (n \pi t) \, dt +
\int_0^1
(1-t)\,t
\, \sin (n \pi t) \, dt \\
& =
\frac{4 ( 1-{(-1)}^n)}{\pi^3 n^3} 
=
\begin{cases}
\frac{8}{\pi^3 n^3} & \text{if } n \text{ is odd} , \\
0 & \text{if } n \text{ is even} .
\end{cases}
\end{align*}
That is, the series is
\begin{equation*}
\sum_{\substack{n=1 \\ n \text{ odd}}}^\infty \frac{8}{\pi^3 n^3} \sin (n \pi t) .
\end{equation*}

This series converges very fast.
If you plot up
to the third harmonic, that is the function
\begin{equation*}
\frac{8}{\pi^3} \sin (\pi t) + 
\frac{8}{27 \pi^3} \sin (3 \pi t) ,
\end{equation*}
it is almost indistinguishable from the plot of $f(t)$ in
\figurevref{gfs:smoothexfig}.
In fact, the coefficient 
$\frac{8}{27 \pi^3}$ is already just 0.0096 (approximately).
The reason for this behavior is the $n^3$ term in the denominator.
The coefficients $b_n$ in this case go to zero as fast as
$\nicefrac{1}{n^3}$ goes to
zero.
\end{example}

For functions constructed piecewise from polynomials as above,
it is generally true that
if you have one derivative, the Fourier
coefficients will go to zero approximately like $\nicefrac{1}{n^3}$.  If you
have only a continuous function, then the Fourier coefficients will go to
zero as $\nicefrac{1}{n^2}$.  If you have discontinuities, then 
the Fourier coefficients will go to zero approximately as $\nicefrac{1}{n}$.
For more general functions the story is somewhat more complicated but the
same idea holds, the more derivatives you have, the faster the coefficients
go to zero.  Similar reasoning works in reverse.  If the coefficients go to
zero like $\nicefrac{1}{n^2}$, you always obtain a continuous function.  If
they go to zero like $\nicefrac{1}{n^3}$, you obtain an everywhere differentiable
function.
%Therefore, we can tell a lot about the smoothness of a function by looking
%at its Fourier coefficients.

To justify this behavior, take for example the function defined by
the Fourier series
\begin{equation*}
f(t) = \sum_{n=1}^\infty \frac{1}{n^3} \sin (n t) .
\end{equation*}
When we differentiate term by term we notice
\begin{equation*}
f'(t) = \sum_{n=1}^\infty \frac{1}{n^2} \cos (n t) .
\end{equation*}
Therefore, the coefficients now go down like $\nicefrac{1}{n^2}$, which 
means that we have a continuous function.
The derivative 
of $f'(t)$ is defined
at most points, but there are points where $f'(t)$ is not differentiable.
It has corners, but no jumps.
If we
differentiate again (where we can), we find that the function
$f''(t)$,
now fails to be continuous (has jumps)
\begin{equation*}
f''(t) = \sum_{n=1}^\infty \frac{-1}{n} \sin (n t) .
\end{equation*}
This function is similar to the sawtooth.  If we tried to differentiate
the series again, we would obtain
\begin{equation*}
\sum_{n=1}^\infty -\cos (n t) ,
\end{equation*}
which does not converge!

\begin{exercise}
Use a computer to plot the series we obtained for $f(t)$, $f'(t)$ and
$f''(t)$.  That is, plot say the first 5 harmonics of the functions.  At what
points does $f''(t)$ have the discontinuities?
\end{exercise}

\subsection{Exercises}

\begin{exercise}
Let
\begin{equation*}
f(t) =
\begin{cases}
0 & \text{if } \; {-1} < t \leq 0 , \\
t & \text{if } \; \phantom{-}0 < t \leq  1 ,
\end{cases}
\end{equation*}
extended periodically.
\begin{tasks}
\task Compute the Fourier series for $f(t)$.
\task Write out the series explicitly up to the $3^{\text{rd}}$ harmonic.
\end{tasks}
\end{exercise}

\begin{exercise}
Let
\begin{equation*}
f(t) =
\begin{cases}
-t & \text{if } \; {-1} < t \leq 0 , \\
t^2 & \text{if } \; \phantom{-}0 < t \leq  1 ,
\end{cases}
\end{equation*}
extended periodically.
\begin{tasks}
\task Compute the Fourier series for $f(t)$.
\task Write out the series explicitly up to the $3^{\text{rd}}$ harmonic.
\end{tasks}
\end{exercise}

\begin{exercise}
Let
\begin{equation*}
f(t) =
\begin{cases}
\frac{-t}{10} & \text{if } \; {-10} < t \leq 0 , \\
\frac{t}{10} & \text{if } \; \phantom{-1}0 < t \leq  10 ,
\end{cases}
\end{equation*}
extended periodically (period is 20).
\begin{tasks}
\task Compute the Fourier series for $f(t)$.
\task Write out the series explicitly up to the $3^{\text{rd}}$ harmonic.
\end{tasks}
\end{exercise}

\begin{exercise}
Let $f(t) = \sum_{n=1}^\infty \frac{1}{n^3} \cos (n t)$.  Is $f(t)$
continuous and differentiable everywhere?  Find the derivative (if it exists
everywhere)
or justify why $f(t)$ is not differentiable everywhere.
\end{exercise}

\begin{exercise}
Let $f(t) = \sum_{n=1}^\infty \frac{{(-1)}^n}{n} \sin (n t)$.  Is $f(t)$
differentiable everywhere?  Find the derivative (if it exists everywhere) or
justify why $f(t)$ is not differentiable everywhere.
\end{exercise}

\begin{exercise}
Let
\begin{equation*}
f(t) =
\begin{cases}
0 & \text{if } \; {-2} < t \leq 0, \\
t & \text{if } \; \phantom{-}0 < t \leq 1, \\
-t+2 & \text{if } \; \phantom{-}1 < t \leq 2,
\end{cases}
\end{equation*}
extended periodically.
\begin{tasks}
\task Compute the Fourier series for $f(t)$.
\task Write out the series explicitly up to the $3^{\text{rd}}$ harmonic.
\end{tasks}
\end{exercise}

\begin{exercise}
Let
\begin{equation*}
f(t) = e^t \qquad \text{for } \; {-1} < t \leq 1
\end{equation*}
extended periodically.
\begin{tasks}
\task Compute the Fourier series for $f(t)$.
\task Write out the series explicitly up to the $3^{\text{rd}}$ harmonic.
\task What does the series converge to at $t=1$.
\end{tasks}
\end{exercise}

\begin{exercise}
\pagebreak[2]
Let
\begin{equation*}
f(t) = t^2 \qquad \text{for } \; {-1} < t \leq 1
\end{equation*}
extended periodically.
\begin{tasks}
\task Compute the Fourier series for $f(t)$.
\task By plugging in $t=0$,
evaluate $\displaystyle \sum_{n=1}^\infty \frac{{(-1)}^n}{n^2} = 1 - \frac{1}{4} +
\frac{1}{9} - \cdots$.
\task Now evaluate $\displaystyle \sum_{n=1}^\infty \frac{1}{n^2} = 1 + \frac{1}{4} +
\frac{1}{9} + \cdots$.
\end{tasks}
\end{exercise}

\begin{exercise}
Let
\begin{equation*}
f(t) =
\begin{cases}
0 & \text{if } \; {-3} < t \leq 0, \\
t & \text{if } \; \phantom{-}0 < t \leq 3,
\end{cases}
\end{equation*}
extended periodically.  Suppose $F(t)$ is the function given
by the Fourier series of $f$.  Without computing the Fourier series
evaluate
\begin{tasks}(3)
\task $F(2)$
\task $F(-2)$
\task $F(4)$
\task $F(-4)$
\task $F(3)$
\task $F(-9)$
\end{tasks}
\end{exercise}

\setcounter{exercise}{100}

\begin{exercise}
Let
\begin{equation*}
f(t) = t^2 \qquad \text{for } \; {-2} < t \leq 2
\end{equation*}
extended periodically.
\begin{tasks}
\task Compute the Fourier series for $f(t)$.
\task Write out the series explicitly up to the $3^{\text{rd}}$ harmonic.
\end{tasks}
\end{exercise}
\exsol{%
a) $\frac{8}{6} +
\sum\limits_{n=1}^\infty
\frac{16{(-1)}^n}{\pi^2 n^2}
\cos\bigl(\frac{n\pi}{2} t\bigr)$
\quad
b) $\frac{8}{6}
-
\frac{16}{\pi^2 }
\cos\bigl(\frac{\pi}{2} t\bigr)
+
\frac{4}{\pi^2}
\cos\bigl(\pi t\bigr)
-
\frac{16}{9\pi^2}
\cos\bigl(\frac{3\pi}{2} t\bigr) + \cdots$
}

\begin{exercise}
Let
\begin{equation*}
f(t) = t \qquad \text{for } \; {-\lambda} < t \leq \lambda \; \text{ (for some } \lambda > 0 \text{)}
\end{equation*}
extended periodically.
\begin{tasks}
\task Compute the Fourier series for $f(t)$.
\task Write out the series explicitly up to the $3^{\text{rd}}$ harmonic.
\end{tasks}
\end{exercise}
\exsol{%
a)
$\sum\limits_{n=1}^\infty
\frac{{(-1)}^{n+1}2\lambda}{n \pi}
\sin\bigl(\frac{n\pi}{\lambda} t\bigr)$
\quad
b)
$\frac{2\lambda}{\pi}
\sin\bigl(\frac{\pi}{\lambda} t\bigr)
-
\frac{\lambda}{\pi}
\sin\bigl(\frac{2\pi}{\lambda} t\bigr)
+
\frac{2\lambda}{3\pi}
\sin\bigl(\frac{3\pi}{\lambda} t\bigr) - \cdots$
}

\begin{exercise}
Let
\begin{equation*}
f(t) = \frac{1}{2} + \sum_{n=1}^\infty
\frac{1}{n(n^2+1)}
\sin(n\pi t) .
\end{equation*}
Compute $f'(t)$.
\end{exercise}
\exsol{%
$f'(t) = \sum\limits_{n=1}^\infty
\frac{\pi}{n^2+1}
\cos(n\pi t)$
}

\begin{exercise}
\pagebreak[2]
Let
\begin{equation*}
f(t) = \frac{1}{2} + \sum_{n=1}^\infty
\frac{1}{n^3}
\cos(n t) .
\end{equation*}
\begin{tasks}
\task Find the antiderivative.
\task Is the antiderivative periodic?
\end{tasks}
\end{exercise}
\exsol{%
a)
$F(t) = \frac{t}{2} + C + \sum\limits_{n=1}^\infty
\frac{1}{n^4}
\sin(nt)$
\qquad
b) no
}

\begin{exercise}
Let
\begin{equation*}
f(t) = \nicefrac{t}{2} \qquad \text{for } \; {-\pi} < t < \pi
\end{equation*}
extended periodically.
\begin{tasks}
\task Compute the Fourier series for $f(t)$.
\task Plug in $t=\nicefrac{\pi}{2}$ to find a series representation
for $\nicefrac{\pi}{4}$.
\task Using the first 4 terms of the result from part b) approximate
$\nicefrac{\pi}{4}$.
\end{tasks}
\end{exercise}
\exsol{%
a)
$\sum\limits_{n=1}^\infty
\frac{{(-1)}^{n+1}}{n} \sin(nt)$
\qquad
b) $f$ is continuous at $t=\nicefrac{\pi}{2}$ so the
Fourier series converges to $f(\nicefrac{\pi}{2}) = \nicefrac{\pi}{4}$.
Obtain
$\nicefrac{\pi}{4} = \sum\limits_{n=1}^\infty
\frac{{(-1)}^{n+1}}{2n-1} = 1 - \nicefrac{1}{3} + \nicefrac{1}{5}-
\nicefrac{1}{7} + \cdots$.
\qquad
c) Using the first 4 terms get $\nicefrac{76}{105}\approx 0.72$ (quite a bad
approximation, you would have to take about 50 terms to start to get to
within $0.01$ of $\nicefrac{\pi}{4}$).
}

\begin{exercise}
Let
\begin{equation*}
f(t) = 
\begin{cases}
0 & \text{if } \; {-2} < t \leq 0, \\
2 & \text{if } \; \phantom{-}0 < t \leq 2,
\end{cases}
\end{equation*}
extended periodically.  Suppose $F(t)$ is the function given
by the Fourier series of $f$.  Without computing the Fourier series
evaluate
\begin{tasks}(3)
\task $F(0)$
\task $F(-1)$
\task $F(1)$
\task $F(-2)$
\task $F(4)$
\task $F(-9)$
\end{tasks}
\end{exercise}
\exsol{%
a) $F(0) = 1$, 
b) $F(-1) = 0$, 
c) $F(1) = 2$, 
d) $F(-2) = 1$, 
e) $F(4) = 1$, 
f) $F(-9) = 0$ 
}

%%%%%%%%%%%%%%%%%%%%%%%%%%%%%%%%%%%%%%%%%%%%%%%%%%%%%%%%%%%%%%%%%%%%%%%%%%%%%%

\sectionnewpage
\section{Sine and cosine series}
\label{sec:scs}

%mbxINTROSUBSECTION

\sectionnotes{2 lectures\EPref{, \S9.3 in \cite{EP}}\BDref{,
\S10.4 in \cite{BD}}}

\subsection{Odd and even periodic functions}

You may have noticed by now that an odd function has no cosine terms in the 
Fourier series and an even function has no sine terms in the Fourier series.
This observation is not a coincidence.  Let us look at even and odd periodic
function in more detail.

Recall that a function $f(t)$ is \emph{odd}\index{odd function} if $f(-t) =
-f(t)$.  A function $f(t)$ is \emph{even}\index{even function} if
$f(-t) = f(t)$.  For example, $\cos (n t)$ is even and $\sin (n t)$ is odd.
Similarly the function $t^k$ is even if $k$ is even and odd when $k$ is odd.

\begin{exercise}
Take two functions $f(t)$ and $g(t)$ and define their product $h(t) =
f(t)g(t)$.
\begin{tasks}
\task Suppose both $f(t)$ and $g(t)$ are odd.  Is $h(t)$ odd or even?
\task Suppose one is even and one is odd.  Is $h(t)$ odd or even?
\task Suppose both are even.  Is $h(t)$ odd or even?
\end{tasks}
\end{exercise}

If $f(t)$ and $g(t)$ are both odd, then $f(t)+g(t)$ is odd.  Similarly for
even functions.  On the other hand,
if $f(t)$ is odd and $g(t)$ even, then we cannot say anything about
the sum
$f(t) + g(t)$.  In fact, the Fourier series of any function is a sum of
an odd (the sine terms) and an even (the cosine terms) function.

In this section we consider odd and even periodic
functions.  We have previously defined the $2L$-periodic extension
of a function defined on the interval $[-L,L]$.  Sometimes we are only
interested in the function on the range $[0,L]$ and it would be convenient
to have an odd (resp.\ even) function.  If the function is odd (resp.\ even),
all the cosine (resp.\ sine) terms disappear.
What we will do is
take the
odd (resp.\ even) extension of the function to $[-L,L]$ and then 
extend periodically to a $2L$-periodic function.

Take a function $f(t)$ defined on $[0,L]$.  On $(-L,L]$ define the functions
\begin{align*}
F_{\text{odd}}(t) & \overset{\text{def}}{=}
\begin{cases}
f(t) & \text{if } \; \phantom{-}0 \leq t \leq L , \\
-f(-t) & \text{if } \; {-L} < t < 0 ,
\end{cases}
\\
F_{\text{even}}(t) & \overset{\text{def}}{=}
\begin{cases}
f(t) & \text{if } \; \phantom{-}0 \leq t \leq L , \\
f(-t) & \text{if } \; {-L} < t < 0 .
\end{cases}
\end{align*}
Extend $F_{\text{odd}}(t)$ and $F_{\text{even}}(t)$ to be $2L$-periodic.
Then
$F_{\text{odd}}(t)$ is called
the \emph{\myindex{odd periodic extension}} of $f(t)$, and
$F_{\text{even}}(t)$ is called the
\emph{\myindex{even periodic extension}} of $f(t)$.
For the odd extension we generally assume that $f(0) = f(L) = 0$.

\begin{exercise}
Check that $F_{\text{odd}}(t)$ is odd and $F_{\text{even}}(t)$ is even.
For $F_{\text{odd}}$,
assume $f(0) = f(L) = 0$.
\end{exercise}

\begin{example}
Take the function $f(t) = t\,(1-t)$ defined on $[0,1]$. 
\figurevref{scs:oddevenextfig}
shows the plots of the odd and even periodic extensions of $f(t)$.

\begin{myfig}
\capstart
%original files scs-oddext scs-evenext
\diffyincludegraphics{width=6.24in}{width=9in}{scs-ext-odd-even}
\caption{Odd and even 2-periodic extension of $f(t) =
t\,(1-t)$, $0 \leq t \leq 1$.\label{scs:oddevenextfig}}
\end{myfig}
\end{example}

\subsection{Sine and cosine series}

Let $f(t)$ be an odd $2L$-periodic function.  We write 
the Fourier series for $f(t)$.  First, we compute the coefficients $a_n$ (including
$n=0$) and get
\begin{equation*}
a_n = \frac{1}{L} \int_{-L}^L f(t) \cos \left( \frac{n \pi}{L} t \right)
\, dt = 0 .
\end{equation*}
That is, there are no cosine terms in the Fourier series of an odd function.
The integral is zero
because $f(t) \cos \left( {n \pi}{L} t \right)$
is an odd function (product of an odd and an
even function is odd) and the integral of an odd function over a symmetric
interval is always zero.
The integral of an even function over a symmetric interval
$[-L,L]$ is twice the integral of the function over the interval $[0,L]$.
The function $f(t) \sin \left( \frac{n \pi}{L} t \right)$ is the product of two odd
functions and hence is even.
\begin{equation*}
b_n = 
\frac{1}{L} \int_{-L}^L f(t) \sin \left( \frac{n \pi}{L} t \right) \, dt =
\frac{2}{L} \int_{0}^L f(t) \sin \left( \frac{n \pi}{L} t \right) \, dt .
\end{equation*}
We now write the Fourier series of $f(t)$ as
\begin{equation*}
\sum_{n=1}^\infty b_n \sin \left( \frac{n \pi}{L} t \right) .
\end{equation*}

Similarly, if $f(t)$ is an even $2L$-periodic function.  For the same exact
reasons as above, we find that $b_n = 0$ and
\begin{equation*}
a_n = 
\frac{2}{L} \int_{0}^L f(t) \cos \left( \frac{n \pi}{L} t \right) \, dt .
\end{equation*}
The formula still works for $n=0$, in which case it becomes
\begin{equation*}
a_0 = 
\frac{2}{L} \int_{0}^L f(t) \, dt .
\end{equation*}
The Fourier series is then
\begin{equation*}
\frac{a_0}{2}
+
\sum_{n=1}^\infty a_n \cos \left( \frac{n \pi}{L} t \right) .
\end{equation*}

An interesting consequence is that the coefficients of the Fourier series of
an odd (or even) function can be computed by just integrating over the half
interval $[0,L]$.  Therefore, we can compute the Fourier series of
the odd (or even) extension of a
function by computing certain integrals over the interval
where the original function is defined.

\begin{theorem}
Let $f(t)$ be a piecewise smooth function defined on $[0,L]$.
Then the odd periodic extension
of $f(t)$ has the Fourier series
\begin{equation*}
\mybxbg{~~
F_{\text{odd}}(t) = \sum_{n=1}^\infty b_n \sin \left( \frac{n \pi}{L} t
\right) ,
~~}
\end{equation*}
where
\begin{equation*}
\mybxbg{~~
b_n = 
\frac{2}{L} \int_{0}^L f(t)\, \sin \left( \frac{n \pi}{L} t \right) \, dt .
~~}
\end{equation*}
The even periodic extension of $f(t)$ has the Fourier series
\begin{equation*}
\mybxbg{~~
F_{\text{even}}(t) = \frac{a_0}{2} + \sum_{n=1}^\infty a_n \cos \left(
\frac{n \pi}{L} t \right) ,
~~}
\end{equation*}
where
\begin{equation*}
\mybxbg{~~
a_n = 
\frac{2}{L} \int_{0}^L f(t)\, \cos \left( \frac{n \pi}{L} t \right) \, dt .
~~}
\end{equation*}
\end{theorem}

We call the series $\sum_{n=1}^\infty b_n \sin \left( \frac{n \pi}{L} t\right)$ 
the \emph{\myindex{sine series}} of $f(t)$ and we call the series
$\frac{a_0}{2} + \sum_{n=1}^\infty a_n \cos \left( \frac{n \pi}{L} t
\right)$
the \emph{\myindex{cosine series}} of $f(t)$.  
We often do not actually care what happens outside of $[0,L]$.  In this case,
we pick whichever series fits our problem better.

It is not necessary to start with the full Fourier series to obtain
the sine and cosine series.
The sine series is really the eigenfunction expansion of $f(t)$ using 
eigenfunctions of the eigenvalue problem $x''+\lambda x = 0$, $x(0) = 0$,
$x(L) = L$.  The cosine series is the eigenfunction expansion of $f(t)$
using 
eigenfunctions of the eigenvalue problem $x''+\lambda x = 0$, $x'(0) = 0$,
$x'(L) = L$.  We could have, therefore, gotten the same formulas
by defining the inner product
\begin{equation*}
\langle f(t), g(t) \rangle = \int_0^L f(t) g(t) \, dt ,
\end{equation*}
and following the procedure of \sectionref{ts:section}.  This point of view is
useful, as we commonly use a specific series that arose because our underlying
question 
led to a certain eigenvalue problem.  If the eigenvalue 
problem is not one of the three we covered so far, you can still do an
eigenfunction expansion, generalizing the results of this chapter.  We will
deal with such a generalization in \chapterref{SL:chapter}.

%f(t) = \frac{a_0}{2} + \sum_{n=1}^\infty a_n \cos \left( \frac{n \pi}{L} 
%t \right)
%+ b_n \sin \left( \frac{n \pi}{L} t \right) ,

\begin{example}
Find the Fourier series of the even periodic extension of 
the function $f(t) = t^2$ for $0 \leq t \leq \pi$.

We want to write
\begin{equation*}
f(t) = \frac{a_0}{2} + \sum_{n=1}^\infty a_n \cos (n t) ,
\end{equation*}
where
\begin{equation*}
a_0 = \frac{2}{\pi}
\int_0^\pi t^2 \, dt = \frac{2 \pi^2}{3} ,
\end{equation*}
and
\begin{equation*}
\begin{split}
a_n & = \frac{2}{\pi}
\int_0^\pi t^2 \cos (n t) \, dt
= \frac{2}{\pi} \left[ t^2 \frac{1}{n} \sin (nt) \right]_0^\pi -
\frac{4}{n\pi}
\int_0^\pi t \sin (n t) \, dt \\
& = 
\frac{4}{n^2\pi}
\Bigl[ t \cos (n t) \Bigr]_0^\pi
+
\frac{4}{n^2\pi}
\int_0^\pi \cos (n t) \, dt
= 
\frac{4{(-1)}^n}{n^2} .
\end{split}
\end{equation*}
Note that we have \myquote{detected} the continuity of the extension since the
coefficients decay as $\frac{1}{n^2}$.  That is, the even periodic extension
of $t^2$ has no jump discontinuities.  It does have corners, since
the derivative, which is an odd function and a sine series, has jumps; it has
a Fourier series whose coefficients decay only as $\frac{1}{n}$.

Explicitly, the first few terms of the series are
\begin{equation*}
\frac{\pi^2}{3} - 4 \cos (t) + \cos (2t) - \frac{4}{9} \cos (3t) + \cdots
\end{equation*}
\end{example}

\begin{exercise}
\leavevmode
\begin{tasks}
\task Compute the derivative of the even periodic extension of $f(t)$ above and verify it
has jump discontinuities.  Use the actual definition of $f(t)$, not its cosine
series!
\task Why is it that the derivative of the even periodic extension of $f(t)$ is the
odd periodic extension of $f'(t)$?
\end{tasks}
\end{exercise}

\subsection{Application}

Fourier series ties in to the boundary value problems
we studied earlier.  Let us see this connection in an application.

Consider the boundary value problem for $0 < t < L$,
\begin{equation*}
x''(t) + \lambda\, x(t) = f(t) ,
\end{equation*}
for the \emph{\myindex{Dirichlet boundary conditions}}
$x(0) = 0$, $x(L) = 0$.
The Fredholm alternative (\thmvref{thm:fredholmsimple})
says that
as long as $\lambda$ is not an eigenvalue of the underlying homogeneous
problem, there exists a unique solution.
Eigenfunctions of this eigenvalue problem are the functions
$\sin \left( \frac{n \pi}{L} t \right)$.
Therefore,
to find the solution,
we first find the Fourier sine series for $f(t)$.
We write $x$ also as a sine series, but with unknown coefficients.  
We substitute the series for $x$ into the equation and solve for the unknown
coefficients.
If we have
the \emph{\myindex{Neumann boundary conditions}}
$x'(0) = 0$, $x'(L) = 0$, we do the same procedure using the cosine
series.

Let us see how this method works on examples.

\begin{example}
Take the boundary value problem for $0 < t < 1$,
\begin{equation*}
x''(t) + 2 x(t) = f(t) ,
\end{equation*}
where $f(t) = t$ on $0 < t < 1$, and 
satisfying the Dirichlet boundary conditions
$x(0) = 0$, $x(1)=0$.
We write $f(t)$ as a sine series
\begin{equation*}
f(t) = \sum_{n=1}^\infty c_n \sin (n \pi t) .
\end{equation*}
Compute
\begin{equation*}
c_n = 2 \int_0^1 t \sin (n \pi t) \,dt = \frac{2 \, {(-1)}^{n+1}}{n \pi} .
\end{equation*}
We write $x(t)$ as
\begin{equation*}
x(t) = \sum_{n=1}^\infty b_n \sin (n \pi t) .
\end{equation*}
We plug in to obtain 
\begin{equation*}
\begin{split}
x''(t) + 2 x(t) & =
\underbrace{
\sum_{n=1}^\infty - b_n n^2 \pi^2 \sin (n \pi t)
}_{x''}
\,
+
\,
2
\underbrace{
\sum_{n=1}^\infty b_n \sin (n \pi t)
}_{x}
\\
& =
\sum_{n=1}^\infty b_n (2 - n^2 \pi^2 ) \sin (n \pi t)
\\
& = f(t)
=
\sum_{n=1}^\infty  \frac{2\, {(-1)}^{n+1}}{n \pi} \sin (n \pi t) .
\end{split}
\end{equation*}
Therefore,
\begin{equation*}
b_n (2 - n^2 \pi^2)
=
\frac{2\,{(-1)}^{n+1}}{n \pi}
\end{equation*}
or
\begin{equation*}
b_n
=
\frac{2\,{(-1)}^{n+1}}{n \pi (2 - n^2 \pi^2)} .
\end{equation*}
That $2-n^2\pi^2$ is not zero for any $n$, and that we can
solve for $b_n$, is precisely because
$2$ is not an eigenvalue of the problem.
We have thus obtained a Fourier series for the solution
\begin{equation*}
x(t) = 
\sum_{n=1}^\infty
\frac{2\,{(-1)}^{n+1}}{n \pi \,(2 - n^2 \pi^2)}
\sin (n \pi t) .
\end{equation*}
See \figurevref{bnd-dirich-graph:fig} for a graph of the solution.
Notice that because the eigenfunctions satisfy the boundary conditions, 
and $x$ is written in terms of the boundary conditions, then $x$
satisfies the boundary conditions.
\begin{myfig}
\capstart
\diffyincludegraphics{width=3in}{width=4.5in}{bnd-dirich-graph}
\caption{Plot of the solution of $x''+2x=t$, $x(0)=0$, $x(1)=0$.%
\label{bnd-dirich-graph:fig}}
\end{myfig}
\end{example}

\begin{example}
Similarly we handle the Neumann conditions.
Take the boundary value problem for $0 < t < 1$,
\begin{equation*}
x''(t) + 2 x(t) = f(t) ,
\end{equation*}
where again $f(t) = t$ on $0 < t < 1$, but now satisfying
the Neumann boundary conditions
$x'(0) = 0$, $x'(1)=0$.
We write $f(t)$ as a cosine series
\begin{equation*}
f(t) = \frac{c_0}{2} + \sum_{n=1}^\infty c_n \cos (n \pi t) ,
\end{equation*}
where
\begin{equation*}
c_0 = 2 \int_0^1 t \,dt = 1 ,
\end{equation*}
and
\begin{equation*}
c_n = 2 \int_0^1 t \cos (n \pi t) \,dt =
\frac{2\bigl({(-1)}^n-1\bigr)}{\pi^2 n^2} = 
\begin{cases}
\frac{-4}{\pi^2 n^2} & \text{if } n \text{ odd} , \\
0 & \text{if } n \text{ even}.
\end{cases}
\end{equation*}
We write $x(t)$ as a cosine series
\begin{equation*}
x(t) = \frac{a_0}{2} + \sum_{n=1}^\infty a_n \cos (n \pi t) .
\end{equation*}
We plug in to obtain 
\begin{equation*}
\begin{split}
x''(t) + 2 x(t) & =
\sum_{n=1}^\infty \Bigl[ - a_n n^2 \pi^2 \cos (n \pi t) \Bigr]
+
a_0 +
2
\sum_{n=1}^\infty \Bigl[ a_n \cos (n \pi t) \Bigr]
\\
& =
a_0 +
\sum_{n=1}^\infty a_n (2 - n^2 \pi^2 ) \cos (n \pi t)
\\
& = f(t)
=
\frac{1}{2} +
\sum_{\substack{n=1\\n~\text{odd}}}^\infty
\frac{-4}{\pi^2 n^2} \cos (n \pi t) .
\end{split}
\end{equation*}
Therefore, $a_0 = \frac{1}{2}$, $a_n = 0$ for $n$ even ($n \geq 2$) and for
$n$ odd we have
\begin{equation*}
a_n (2 - n^2 \pi^2)
=
\frac{-4}{\pi^2 n^2} ,
\end{equation*}
or
\begin{equation*}
a_n
=
\frac{-4}{n^2 \pi^2 (2 - n^2 \pi^2)} .
\end{equation*}
The Fourier series for the solution $x(t)$ is
\begin{equation*}
x(t) = 
\frac{1}{4} +
\sum_{\substack{n=1\\n~\text{odd}}}^\infty
\frac{-4}{n^2 \pi^2 (2 - n^2 \pi^2)} 
\cos (n \pi t) .
\end{equation*}
\end{example}

\subsection{Exercises}

\begin{exercise}
Take $f(t) = {(t-1)}^2$ defined on $0 \leq t \leq 1$.
\begin{tasks}
\task Sketch the plot of the even periodic extension of $f$.
\task Sketch the plot of the odd periodic extension of $f$.
\end{tasks}
\end{exercise}

\begin{exercise}
Find the Fourier series of both the odd and even
periodic extension of 
the function $f(t) = {(t-1)}^2$ for $0 \leq t \leq 1$.
Can you tell which extension is continuous from the Fourier series
coefficients?
\end{exercise}

\begin{exercise}
Find the Fourier series of both the odd and even periodic extension of 
the function $f(t) = t$ for $0 \leq t \leq \pi$.
\end{exercise}

\begin{exercise}
Find the Fourier series of the even periodic extension of 
the function $f(t) = \sin t$ for $0 \leq t \leq \pi$.
\end{exercise}

\begin{exercise}
\pagebreak[2]
Consider
\begin{equation*}
x''(t) + 4 x(t) = f(t) ,
\end{equation*}
where $f(t) = 1$ on $0 < t < 1$.
\begin{tasks}
\task Solve for the Dirichlet conditions $x(0)=0, x(1) = 0$.
\task Solve for the Neumann conditions $x'(0)=0, x'(1) = 0$.
\end{tasks}
\end{exercise}

\begin{exercise}
Consider
\begin{equation*}
x''(t) + 9 x(t) = f(t) ,
\end{equation*}
for $f(t) = \sin (2\pi t)$ on $0 < t < 1$.
\begin{tasks}
\task Solve for the Dirichlet conditions $x(0)=0, x(1) = 0$.
\task Solve for the Neumann conditions $x'(0)=0, x'(1) = 0$.
\end{tasks}
\end{exercise}

\begin{exercise}
Consider
\begin{equation*}
x''(t) + 3 x(t) = f(t) , \quad x(0) = 0, \quad x(1) = 0,
\end{equation*}
where $f(t) = \sum_{n=1}^\infty b_n \sin (n \pi t)$.  Write the solution $x(t)$
as a Fourier series, where the coefficients are given in terms of $b_n$.
\end{exercise}

\begin{exercise}
Let $f(t) = t^2(2-t)$ for $0 \leq t \leq 2$.  Let $F(t)$ be the odd periodic
extension.  Compute $F(1)$, $F(2)$, $F(3)$, $F(-1)$, $F(\nicefrac{9}{2})$,
$F(101)$, $F(103)$.  Note: Do \textbf{not} compute the sine series.
\end{exercise}

\setcounter{exercise}{100}

\begin{exercise}
Let $f(t) = \nicefrac{t}{3}$ on $0 \leq t < 3$.
\begin{tasks}
\task Find the Fourier series of the even periodic extension.
\task Find the Fourier series of the odd periodic extension.
\end{tasks}
\end{exercise}
\exsol{%
a)
$\nicefrac{1}{2}
+
\sum\limits_{\substack{n=1\\n\text{ odd}}}^\infty
\frac{-4}{\pi^2 n^2}
\cos\bigl(\frac{n\pi}{3} t \bigr)$
\qquad
b) 
$\sum\limits_{n=1}^\infty
\frac{2{(-1)}^{n+1}}{\pi n}
\sin\bigl(\frac{n\pi}{3} t \bigr)$
}

\begin{exercise}
Let $f(t) = \cos(2t)$ on $0 \leq t < \pi$.
\begin{tasks}
\task Find the Fourier series of the even periodic extension.
\task Find the Fourier series of the odd periodic extension.
\end{tasks}
\end{exercise}
\exsol{%
a)
$\cos(2t)$
\qquad
b) 
$\sum\limits_{\substack{n=1 \\n \text{ odd}}}^\infty
\frac{-4n}{\pi n^2 - 4 \pi}
\sin(n t)$
}

\begin{exercise}
Let $f(t)$ be defined on $0 \leq t < 1$.  Consider
the average of the two extensions
$g(t) = \frac{F_{\text{odd}}(t)+ F_{\text{even}}(t)}{2}$.
\begin{tasks}(2)
\task What is $g(t)$ if $0 \leq t < 1$ (Justify!)
\task What is $g(t)$ if $-1 < t < 0$ (Justify!)
\end{tasks}
\end{exercise}
\exsol{%
a) $f(t)$
\qquad
b) $0$
}

\begin{exercise}
Let $f(t) = \sum_{n=1}^\infty \frac{1}{n^2} \sin(nt)$.  Solve
$x''- x = f(t)$ for the Dirichlet conditions $x(0) = 0$
and $x(\pi) = 0$.
\end{exercise}
\exsol{%
$\sum\limits_{n=1}^\infty \frac{-1}{n^2(1+n^2)} \sin(nt)$
}

\begin{exercise}[challenging]
Let $f(t) = t + \sum_{n=1}^\infty \frac{1}{2^n} \sin(nt)$.  Solve
$x'' + \pi x = f(t)$ for the Dirichlet conditions $x(0) = 0$
and $x(\pi) = 1$.  Hint:  Note that $\frac{t}{\pi}$ satisfies the
given Dirichlet conditions.
\end{exercise}
\exsol{%
$\frac{t}{\pi} + \sum\limits_{n=1}^\infty \frac{1}{2^n(\pi-n^2)} \sin(nt)$
}


%%%%%%%%%%%%%%%%%%%%%%%%%%%%%%%%%%%%%%%%%%%%%%%%%%%%%%%%%%%%%%%%%%%%%%%%%%%%%%

\sectionnewpage
\section{Applications of Fourier series}
\label{appoffourier:section}

%mbxINTROSUBSECTION

\sectionnotes{2 lectures\EPref{, \S9.4 in \cite{EP}}\BDref{,
not in \cite{BD}}}

\subsection{Periodically forced oscillation}

\begin{mywrapfigsimp}{2.0in}{2.3in}
\noindent
\inputpdft{massfigforce}
\end{mywrapfigsimp}
Let us return to the forced oscillations.  Consider a mass-spring system as
before, where we have a mass $m$
on a spring with spring constant $k$,
with damping $c$, and a force $F(t)$ applied to the mass.  Suppose 
the forcing function $F(t)$ is $2L$-periodic for some $L > 0$.
We saw
this problem in \chapterref{ho:chapter} with $F(t) = F_0 \cos (\omega t)$.  The
equation that governs this particular setup is
\begin{equation} \label{afs:eq}
mx''(t) + cx'(t) + kx(t) = F(t) .
\end{equation}

The general solution of \eqref{afs:eq} consists of the complementary solution $x_c$, which
solves the associated homogeneous equation $mx'' + cx' + kx = 0$, and
a particular solution of \eqref{afs:eq} we call $x_p$.  For $c > 0$,
the complementary solution $x_c$ will decay as time goes by.
Therefore,
we are mostly interested
in a particular solution $x_p$ that does not decay
and is periodic with the same period as $F(t)$.  We call this particular
solution
the \emph{\myindex{steady periodic solution}} and we write it as $x_{sp}$ as before.
What is new in this section is that we consider an arbitrary
forcing function $F(t)$ instead of a simple cosine.

For simplicity, suppose $c=0$.  The problem with $c > 0$ is very
similar.
The equation
\begin{equation*}
mx'' + kx = 0 
\end{equation*}
has the general solution
\begin{equation*}
x(t) = A \cos (\omega_0 t) + 
B \sin (\omega_0 t) ,
\end{equation*}
where $\omega_0 = \sqrt{\frac{k}{m}}$.
Any solution to
$mx''(t) + kx(t) = F(t)$ is of the form
$A \cos (\omega_0 t) + B \sin (\omega_0 t) + x_{sp}$.
The steady
periodic solution $x_{sp}$ has the same period as $F(t)$.

In the spirit of the last section and the idea of undetermined coefficients
we first write
\begin{equation*}
F(t) = \frac{c_0}{2} + \sum_{n=1}^\infty
c_n \cos \left( \frac{n \pi}{L} t \right) +
d_n \sin \left( \frac{n \pi}{L} t \right) .
\end{equation*}
Then we write a proposed steady periodic solution $x$ as
\begin{equation*}
x(t) = \frac{a_0}{2} + \sum_{n=1}^\infty
a_n \cos \left( \frac{n \pi}{L} t \right) +
b_n \sin \left( \frac{n \pi}{L} t \right) ,
\end{equation*}
where $a_n$ and $b_n$ are unknowns.
We plug $x$ into the differential equation and solve for $a_n$ and
$b_n$ in terms of $c_n$ and $d_n$.  This process
is perhaps best understood by example.
\pagebreak[2]

\begin{example} \label{afs:steadyex}
Suppose that $k=2$, and $m=1$.
The units are again the mks units\index{mks units}
(meters-kilograms-seconds).
There is a jetpack strapped to the mass, which fires with a force of 1
newton for 1
second and then is off for 1 second, and so on.  We want to find the steady periodic
solution.

The equation is, therefore,
\begin{equation*}
x'' + 2 x = F(t) ,
\end{equation*}
where $F(t)$ is the step function
\begin{equation*}
F(t) =
\begin{cases}
0 & \text{if } \; {-1} < t < 0 , \\
1 & \text{if } \; \phantom{-}0 < t < 1 ,
\end{cases}
\end{equation*}
extended periodically.
We write
\begin{equation*}
F(t) = \frac{c_0}{2} + \sum_{n=1}^\infty
c_n \cos (n \pi t) +
d_n \sin (n \pi t) .
\end{equation*}
We compute
\begin{align*}
c_n & = \int_{-1}^1 F(t) \cos (n \pi t) \, dt = 
\int_{0}^1 \cos (n \pi t) \, dt = 0 \qquad \text{for } \; n \geq 1,
\\
c_0 & = \int_{-1}^1 F(t) \, dt = 
\int_{0}^1 \, dt = 1 ,
\\
d_n & = \int_{-1}^1 F(t) \sin (n \pi t) \, dt
\\
& = \int_{0}^1 \sin (n \pi t) \, dt
\\
& = \left[ \frac{-\cos (n \pi t)}{n \pi} \right]_{t=0}^1
\\
& = \frac{1-{(-1)}^n}{\pi n} =
\begin{cases}
\frac{2}{\pi n} & \text{if } n \text{ odd} , \\
0 & \text{if } n \text{ even} .
\end{cases}
\end{align*}
So
\begin{equation*}
F(t) = \frac{1}{2} + \sum_{\substack{n=1 \\ n \text{ odd}}}^\infty
\frac{2}{\pi n} \sin (n \pi t) .
\end{equation*}

We want to try
\begin{equation*}
x(t) = \frac{a_0}{2} + \sum_{n=1}^\infty
a_n \cos (n \pi t) +
b_n \sin (n \pi t) .
\end{equation*}
Once we plug $x$ into the differential equation $x''+2x = F(t)$,
it is clear that $a_n = 0$ for $n \geq 1$ as there are no corresponding terms
in the series for
$F(t)$.  Similarly $b_n = 0$ for $n$ even.  Hence we try
\begin{equation*}
x(t) = \frac{a_0}{2} +
\sum_{\substack{n=1 \\ n \text{ odd}}}^\infty
b_n \sin (n \pi t) .
\end{equation*}
We plug into the differential equation and obtain
\begin{equation*}
\begin{split}
x'' + 2 x & =
\sum_{\substack{n=1 \\ n \text{ odd}}}^\infty
\Bigl[ - b_n n^2 \pi^2 \sin (n \pi t) \Bigr] + 
a_0 +
2
\sum_{\substack{n=1 \\ n \text{ odd}}}^\infty
\Bigl[ b_n \sin (n \pi t) \Bigr]
\\
& =
a_0 +
\sum_{\substack{n=1 \\ n \text{ odd}}}^\infty
b_n (2 - n^2 \pi^2 ) \sin (n \pi t)
\\
& =
F(t) = \frac{1}{2} + \sum_{\substack{n=1 \\ n \text{ odd}}}^\infty
\frac{2}{\pi n} \sin (n \pi t) .
\end{split}
\end{equation*}
So $a_0 = \frac{1}{2}$, $b_n = 0$ for even $n$, and for odd $n$ we
get
\begin{equation*}
b_n = 
\frac{2}{\pi n (2 - n^2 \pi^2 )} .
\end{equation*}

The steady periodic solution has the Fourier series
\begin{equation*}
x_{sp}(t) = \frac{1}{4} + \sum_{\substack{n=1 \\ n \text{ odd}}}^\infty
\frac{2}{\pi n (2 - n^2 \pi^2 )}
\sin (n \pi t) .
\end{equation*}
We know this is the steady periodic solution as it contains no terms 
of the complementary solution and it is periodic with the same period as
$F(t)$ itself.  See \figurevref{afs:steadyexfig} for the plot of this solution.
\begin{myfig}
\capstart
\diffyincludegraphics{width=3in}{width=4.5in}{afs-steadyex}
\caption{Plot of the steady periodic solution $x_{sp}$ of
\exampleref{afs:steadyex}.%
\label{afs:steadyexfig}}
\end{myfig}
\end{example}

\subsection{Resonance}

Just as when the forcing function was a simple cosine, we may encounter
resonance.  Assume $c=0$ and let us discuss only pure resonance.
Let $F(t)$ be $2L$-periodic and consider
\begin{equation*}
m x''(t) + k x (t) = F(t) .
\end{equation*}
When we expand $F(t)$ and find that some of its terms coincide with the
complementary solution to $mx''+kx=0$, we cannot use those terms in the
guess.  Just like before, they disappear when we plug them into the left-hand
side and we get a contradictory equation (such as $0=1$).   That is,
suppose
\begin{equation*}
x_c = A \cos (\omega_0 t) + B \sin (\omega_0 t), 
\end{equation*}
where $\omega_0 = \frac{N \pi}{L}$ for some positive integer $N$.
We have
to modify our guess and try
\begin{equation*}
x(t) = \frac{a_0}{2} +
t \left(
a_N \cos \left( \frac{N \pi}{L} t \right) +
b_N \sin \left( \frac{N \pi}{L} t \right) \right) +
\sum_{\substack{n=1\\n\not= N}}^\infty
a_n \cos \left( \frac{n \pi}{L} t \right) +
b_n \sin \left( \frac{n \pi}{L} t \right) .
\end{equation*}
In other words, we multiply the offending term by $t$.  From then on, we
proceed as before.

Of course, the solution is not a Fourier series (it is not even
periodic) since it contains these terms multiplied by $t$.  Further, the
terms
$t \left( a_N \cos \left( \frac{N \pi}{L} t \right) +
b_N \sin \left( \frac{N \pi}{L} t \right) \right)$ eventually dominate and lead to
wild oscillations.  As before, this behavior is called \emph{\myindex{pure
resonance}} or just \emph{\myindex{resonance}}.

Note that there now may be infinitely many resonance frequencies to hit.
That is, as we change the frequency of $F$ (we change $L$), different
terms from the Fourier series of $F$ may interfere with the complementary
solution and cause resonance.
However, we should note that since everything is an approximation and in
particular $c$ is never actually zero but something very close to zero,
only the first
few resonance frequencies matter in real life.

\begin{example}
We want to solve the equation
\begin{equation} \label{afs:eq-resonance}
2 x'' + 18 \pi^2 x = F(t) ,
\end{equation}
where
\begin{equation*}
F(t) =
\begin{cases}
-1 & \text{if } \; {-1} < t < 0 , \\
1 & \text{if } \; \phantom{-}0 < t < 1 ,
\end{cases}
\end{equation*}
extended periodically.  We note that
\begin{equation*}
F(t) =
\sum_{\substack{n=1 \\ n \text{ odd}}}^\infty
\frac{4}{\pi n}
\sin (n \pi t) . 
\end{equation*}

\begin{exercise}
Compute the Fourier series of $F$ to verify the equation above.
\end{exercise}

As $\sqrt{\frac{k}{m}} = \sqrt{\frac{18\pi^2}{2}} = 3\pi$,
the solution to \eqref{afs:eq-resonance} is
\begin{equation*}
x(t) = c_1 \cos  (3\pi t) + c_2 \sin (3\pi t) + x_p (t)
\end{equation*}
for some particular solution $x_p$.

If we just try an $x_p$ given as a Fourier series with $\sin (n\pi t)$ as usual,
the complementary equation, $2x''+18\pi^2x=0$, eats our $3^\text{rd}$ harmonic.  That is, the term
with $\sin(3 \pi t)$
is already in
in our complementary solution.
Therefore, we pull that term out and
multiply it by $t$.  We also add a cosine term to get everything right.
That is, we try
\begin{equation*}
x_p(t) =
a_3
t \cos (3 \pi t )
+
b_3
t \sin (3 \pi t)
+
\sum_{\substack{n=1 \\ n~\text{odd} \\ n\not= 3}}^\infty
b_n
\sin (n \pi t) . 
\end{equation*}
Let us compute the second derivative.
\begin{multline*}
x_p''(t) =
- 6 a_3
\pi \, \sin (3 \pi t) - 9\pi^2 a_3 \, t \, \cos (3 \pi t)
+
6 b_3
\pi \, \cos (3 \pi t) - 9\pi^2 b_3 \, t \, \sin (3 \pi t)
\\
{} +
\sum_{\substack{n=1 \\ n~\text{odd} \\ n\not= 3}}^\infty
(-n^2 \pi^2 b_n ) \,
\sin (n \pi t) . 
\end{multline*}
We now plug into the left-hand side of the differential equation.
\begin{align*}
2x_p'' + 18\pi^2 x_p 
= & 
- 12 a_3 \pi \sin (3 \pi t)
- 18\pi^2 a_3 t \cos (3 \pi t)
+ 12 b_3 \pi \cos (3 \pi t)
- 18\pi^2 b_3 t \sin (3 \pi t)
\\
& \phantom{\, - 12 a_3 \pi \sin (3 \pi t)} ~
{} + 18 \pi^2 a_3 t \cos (3 \pi t)
\phantom{\, + 12 b_3 \pi \cos (3 \pi t)} ~
{} + 18 \pi^2 b_3 t \sin (3 \pi t)
\\
& {} + \sum_{\substack{n=1 \\ n~\text{odd} \\ n\not= 3}}^\infty
(-2n^2 \pi^2 b_n + 18\pi^2 b_n) \,
\sin (n \pi t) . 
\end{align*}
We simplify,
\begin{equation*}
2x_p'' + 18\pi^2 x_p =
- 12 a_3
\pi \sin (3 \pi t)
+
12 b_3
\pi \cos (3 \pi t)
+
\sum_{\substack{n=1 \\ n~\text{odd} \\ n\not= 3}}^\infty
(-2n^2 \pi^2 b_n + 18\pi^2 b_n)
\sin (n \pi t) . 
\end{equation*}
This series has to equal to the series for $F(t)$.
We equate the coefficients and solve for $a_3$ and $b_n$.
\begin{align*}
& a_3 = \frac{4/(3\pi)}{-12\pi} = \frac{-1}{9\pi^2} , \\
& b_3 = 0 , \\
& b_n = \frac{4}{n\pi(18\pi^2 - 2n^2 \pi^2)} 
= \frac{2}{\pi^3 n(9 - n^2)} \qquad \text{for } n \text{ odd and } n\not=3 .
\end{align*}

That is,
\begin{equation*}
x_p(t) =
\frac{-1}{9\pi^2}
\,
t \, \cos (3 \pi t)
+
\sum_{\substack{n=1 \\ n~\text{odd} \\ n\not= 3}}^\infty
\frac{2}{\pi^3 n(9 - n^2)}
\sin (n \pi t) . 
\end{equation*}
\end{example}

When $c > 0$, you do not have to worry about pure resonance.  That is,
there are never any conflicts and you do not need to multiply any
terms by $t$.  There is a corresponding concept of
\myindex{practical resonance}
and it is very similar to the ideas we already explored in
\chapterref{ho:chapter}.
Basically what happens in practical resonance is that one of the
coefficients in the series for $x_{sp}$ can get very big.  Let us not go
into details here.

\subsection{Exercises}

\begin{exercise}
Let $F(t) = \frac{1}{2} + \sum_{n=1}^\infty \frac{1}{n^2} \cos (n \pi t)$.
Find
the steady periodic solution to
$x'' + 2 x = F(t)$.  Express your solution as a Fourier series.
\end{exercise}

\begin{exercise}
Let $F(t) = \sum_{n=1}^\infty \frac{1}{n^3} \sin (n \pi t)$.  Find
the steady periodic solution to
$x'' + x' + x = F(t)$.  Express your solution as a Fourier series.
\end{exercise}

\begin{exercise}
Let $F(t) = \sum_{n=1}^\infty \frac{1}{n^2} \cos (n \pi t)$.  Find
the steady periodic solution to
$x'' + 4 x = F(t)$.  Express your solution as a Fourier series.
\end{exercise}

\begin{exercise}
Let $F(t) = t$ for $-1 < t < 1$ and extended periodically.
Find the steady periodic solution to
$x'' + x = F(t)$.  Express your solution as a series.
\end{exercise}

\begin{exercise}
Let $F(t) = t$ for $-1 < t < 1$ and extended periodically.
Find the steady periodic solution to
$x'' + \pi^2 x = F(t)$.  Express your solution as a series.
\end{exercise}

\setcounter{exercise}{100}

\begin{exercise}
Let $F(t) = \sin(2\pi t) + 0.1 \cos(10 \pi t)$.
Find the steady periodic solution to $x'' + \sqrt{2}\, x = F(t)$.
Express your solution as a Fourier series.
\end{exercise}
\exsol{%
$x = \frac{1}{\sqrt{2}-4 \pi^2} \sin(2\pi t) + \frac{0.1}{\sqrt{2}-100 \pi^2} \cos(10 \pi t)$
}

\begin{exercise}
Let $F(t) = \sum_{n=1}^\infty e^{-n} \cos(2 n t)$.
Find the steady periodic solution to $x'' + 3 x = F(t)$.
Express your solution as a Fourier series.
\end{exercise}
\exsol{%
$x =
\sum\limits_{n=1}^\infty
\frac{e^{-n}}{3-{(2n)}^2} \cos(2n t)$
}

\begin{exercise}
Let $F(t) = \lvert t \rvert$ for $-1 \leq t \leq 1$ extended periodically.
Find the steady periodic solution to $x'' + \sqrt{3}\, x = F(t)$.
Express your solution as a series.
\end{exercise}
\exsol{%
$x =
\frac{1}{2\sqrt{3}} + 
\sum\limits_{\substack{n=1 \\ n \text{ odd}}}^\infty \frac{-4}{n^2 \pi^2
(\sqrt{3}-n^2 \pi^2)} \cos (n \pi t)$
}

\begin{exercise}
Let $F(t) = \lvert t \rvert$ for $-1 \leq t \leq 1$ extended periodically.
Find the steady periodic solution to $x'' + \pi^2 x = F(t)$.
Express your solution as a series.
\end{exercise}
\exsol{%
$x =
\frac{1}{2\pi^2} -
\frac{2}{\pi^3} t \sin(\pi t) + 
\sum\limits_{\substack{n=3 \\ n \text{ odd}}}^\infty \frac{-4}{n^2 \pi^4
(1-n^2)} \cos (n \pi t)$
}


%%%%%%%%%%%%%%%%%%%%%%%%%%%%%%%%%%%%%%%%%%%%%%%%%%%%%%%%%%%%%%%%%%%%%%%%%%%%%%

\sectionnewpage
\section{PDEs, separation of variables, and the heat equation}
\label{heateq:section}

%mbxINTROSUBSECTION

\sectionnotes{2 lectures\EPref{, \S9.5 in \cite{EP}}\BDref{,
\S10.5 in \cite{BD}}}

Let us recall that a \emph{\myindex{partial differential equation}} or
\emph{\myindex{PDE}} is an equation containing the partial derivatives
with respect to \emph{several} independent variables.  Solving PDEs
will be our main application of Fourier series.

A PDE is said to be \emph{linear\index{linear PDE}} if the dependent
variable and its derivatives appear at most to the first power and in no
functions.  We will only talk about linear PDEs.  Together with a PDE\@,
we usually specify some
\emph{boundary conditions\index{boundary conditions for a PDE}},
where the value of the solution or its derivatives is given along
the boundary of a region, and/or
some 
\emph{initial conditions\index{initial conditions for a PDE}} where the value
of the solution or its derivatives is given for some initial time.
Sometimes such conditions are mixed together and we will refer to them
simply as 
\emph{side conditions\index{side conditions for a PDE}}.

We will study three specific
partial differential equations, each one representing a
general class of equations.  First, we will study the
\emph{\myindex{heat equation}}, which is an example of
a \emph{\myindex{parabolic PDE}}.  Next, we will study the
\emph{\myindex{wave equation}}, which is an example of
a \emph{\myindex{hyperbolic PDE}}.  Finally, we will study the
\emph{\myindex{Laplace equation}}, which is an example of
an \emph{\myindex{elliptic PDE}}.  Each of our examples will illustrate
behavior that is typical for the whole class.

\subsection{Heat on an insulated wire}

Let us start with the heat equation.
Consider a wire (or a thin metal rod) of length $L$
that is insulated
except at the
endpoints.  Let $x$ denote the position along the wire and let $t$ denote
time.  See \figurevref{heat:wirefig}.

\begin{myfig}
\capstart
\inputpdft{heat-wire}
\caption{Insulated wire.\label{heat:wirefig}}
\end{myfig}

Let $u(x,t)$ denote the temperature at point $x$ at time $t$.
The equation governing this setup is the
so-called \emph{\myindex{one-dimensional heat equation}}\index{heat equation}:
\begin{equation*}
\mybxbg{~~
\frac{\partial u}{\partial t} =
k \frac{\partial^2 u}{\partial x^2} ,
~~}
\end{equation*}
where $k > 0$ is a constant (the \emph{\myindex{thermal conductivity}} of the material).
That is, the change in heat at a specific point is proportional to the second
derivative of the heat along the wire.  This makes sense;
if at a fixed $t$
the graph of the heat distribution has a maximum (the graph is concave down),
then heat flows away from the maximum.  And vice versa.

We generally use a more convenient notation for partial derivatives.
We write $u_t$ instead of $\frac{\partial u}{\partial t}$,
and we write $u_{xx}$ instead of $\frac{\partial^2 u}{\partial x^2}$.
With this notation the heat equation becomes
\begin{equation*}
u_t = k u_{xx} .
\end{equation*}

For the heat equation, we must also have some
boundary conditions.
We assume that the ends of the wire are either exposed 
and touching some body of constant heat, or the ends are insulated.
If the ends of the wire are kept at temperature 0, then
the conditions are
\begin{equation*}
u(0,t) = 0 \qquad \text{and} \qquad u(L,t) = 0.
\end{equation*}
If, on the other hand, the ends are also insulated, the conditions are
\begin{equation*}
u_x(0,t) = 0 \qquad \text{and} \qquad
u_x(L,t) = 0 .
\end{equation*}
Let us see why that is so.
If $u_x$ is positive at some point $x_0$, then at a particular time,
$u$ is smaller to the left of $x_0$, and higher to the right of $x_0$.
Heat is flowing from high heat to low heat, that is to the left.
On the other hand if $u_x$ is negative then heat is again flowing
from high heat to low
heat, that is to the right.  So when $u_x$ is zero, that is a point through
which heat is not flowing.  In other words, $u_x(0,t) = 0$ means
no heat is flowing in or out of the wire at the point $x=0$.

We have two conditions along the $x$-axis as there are
two derivatives in the $x$ direction.
These side conditions are said to be
\emph{homogeneous\index{homogeneous side conditions}}
(i.e., $u$ or a derivative of $u$ is set to zero).

We also need an initial condition---the temperature distribution
at time $t=0$.  That is,
\begin{equation*}
u(x,0) = f(x) ,
\end{equation*}
for some known function $f(x)$.
This initial condition is not a homogeneous side condition.

\subsection{Separation of variables}

The heat equation is linear as $u$ and its derivatives do not
appear to any powers or in any functions.
Thus the principle of \myindex{superposition} still applies for
the heat equation
(without side conditions):
If $u_1$ and $u_2$ are
solutions and $c_1$, $c_2$ are constants, then
$u = c_1 u_1 + c_2 u_2$ is also a solution.

\begin{exercise}
Verify the principle of superposition for the heat equation.
\end{exercise}

Superposition preserves some of the side conditions.  In particular,
if $u_1$ and $u_2$ are
solutions that satisfy $u(0,t) = 0$ and $u(L,t) = 0$,
and $c_1$, $c_2$ are constants, then
$u = c_1 u_1 + c_2 u_2$ is still a solution
that satisfies $u(0,t) = 0$ and $u(L,t) = 0$.  Similarly
for the side conditions $u_x(0,t) = 0$ and $u_x(L,t) = 0$.  In general,
superposition preserves all homogeneous side conditions.

The method of
\emph{separation of variables\index{separation of variables}} is to
try to find solutions that are products of functions of one variable.
For the heat equation, we try to find solutions of the form
\begin{equation*}
u(x,t) = X(x)T(t) .
\end{equation*}
That the desired solution we are looking for is of this form is too much to
hope for.  What is perfectly reasonable to ask, however, is to find
enough \myquote{building-block} solutions of the form
$u(x,t) = X(x)T(t)$ using this procedure
so that the desired solution to the PDE is somehow constructed from these
building blocks by the use of superposition.

Let us try to solve the heat equation
\begin{equation*}
u_t = k u_{xx}
\qquad \text{with} \quad
u(0,t) = 0 ,\quad \quad u(L,t) = 0,
\quad \text{and} \quad u(x,0) = f(x) .
\end{equation*}
We guess $u(x,t) = X(x)T(t)$.  We will try to make this guess satisfy the
differential equation, $u_t = k u_{xx}$, and the homogeneous side conditions,
$u(0,t) = 0$ and $u(L,t) = 0$.  Then, as superposition preserves the
differential equation and the homogeneous side conditions, we will try to
build up a solution from these building blocks to solve the
nonhomogeneous initial condition $u(x,0) = f(x)$.

First we plug $u(x,t) = X(x)T(t)$ into the heat equation to
obtain
\begin{equation*}
X(x)T'(t) = k X''(x)T(t) .
\end{equation*}
We rewrite as
\begin{equation*}
\frac{T'(t)}{k T(t)} =
\frac{X''(x)}{X(x)} .
\end{equation*}
This equation must hold for all $x$ and all $t$.  But the
left-hand side does not depend on $x$ and the right-hand side does not
depend on $t$.  Hence, each side must be a constant.  Let us call this
constant $-\lambda$ (the minus sign is for convenience later).
We obtain the two equations
\begin{equation*}
\frac{T'(t)}{k T(t)} = -\lambda =
\frac{X''(x)}{X(x)} .
\end{equation*}
In other words,
\begin{align*}
X''(x) + \lambda X(x) &= 0 , \\
T'(t) + \lambda k T(t) &= 0 .
\end{align*}
The boundary condition $u(0,t) = 0$ implies $X(0)T(t) = 0$.  We are looking
for a nontrivial solution and so we can assume that $T(t)$ is not identically
zero.  Hence $X(0) = 0$.  Similarly, $u(L,t) = 0$ implies $X(L) = 0$.  We
are looking for nontrivial solutions $X$ of the eigenvalue problem
$X'' + \lambda X = 0$, $X(0) = 0$, $X(L) = 0$.  We have previously found that
the only eigenvalues are $\lambda_n = \frac{n^2 \pi^2}{L^2}$, for integers
$n \geq 1$,
where eigenfunctions are $\sin \left(\frac{n \pi}{L} x\right)$.  Hence, let us pick
the solutions
\begin{equation*}
X_n (x) = \sin \left(\frac{n \pi}{L} x \right) .
\end{equation*}
The corresponding $T_n$ must satisfy the equation
\begin{equation*}
T_n'(t) + \frac{n^2 \pi^2}{L^2} k T_n(t) = 0 .
\end{equation*}
This is one of our
\hyperref[subsection:fourfundamental]{fundamental equations},
and the solution is just
an exponential:
\begin{equation*}
T_n(t) = e^{\frac{-n^2 \pi^2}{L^2} k t} .
\end{equation*}
It will be useful to note that $T_n(0) = 1$.
Our building-block solutions are
\begin{equation*}
u_n(x,t) = X_n(x)T_n(t) =
\sin \left( \frac{n \pi}{L} x \right)
e^{\frac{-n^2 \pi^2}{L^2} k t} .
\end{equation*}

We note that $u_n(x,0) = \sin \left( \frac{n \pi}{L} x \right)$.  Let us
write $f(x)$ as the sine series
\begin{equation*}
f(x) = \sum_{n=1}^\infty b_n \sin \left(\frac{n \pi}{L}  x \right) .
\end{equation*}
That is, we find the Fourier series of the odd periodic extension of $f(x)$.
We used the sine series as it corresponds to the eigenvalue problem for
$X(x)$ above.
Finally, we use superposition to write the solution as
\begin{equation*}
\mybxbg{~~
u(x,t) = 
\sum_{n=1}^\infty
b_n
u_n(x,t)
=
\sum_{n=1}^\infty
b_n
\sin \left( \frac{n \pi}{L}  x \right)
e^{\frac{-n^2 \pi^2}{L^2} k t} .
~~}
\end{equation*}

Why does this solution work?  First note that it is a solution to
the heat equation by superposition.  It satisfies $u(0,t) = 0$
and $u(L,t) = 0$, because $x=0$ or $x=L$ makes all the sines vanish.
Finally, plugging in $t=0$, we notice that $T_n(0) = 1$ and so
\begin{equation*}
u(x,0) = 
\sum_{n=1}^\infty
b_n
u_n(x,0)
=
\sum_{n=1}^\infty
b_n
\sin \left( \frac{n \pi}{L} x \right)
=
f(x) .
\end{equation*}

\begin{example}
Consider
an insulated wire of length 1 whose
ends are embedded in ice (temperature 0).
Let $k=0.003$ and let the initial heat distribution
be $u(x,0) = 50\,x\,(1-x)$.
See \figurevref{heat:wireexinitfig}.
Suppose we want to find the temperature function $u(x,t)$.  Let us also suppose we
want to find when (at what $t$) does the maximum temperature in the wire 
drop to one half of the initial maximum of 12.5.

\begin{myfig}
\capstart
\diffyincludegraphics{width=3in}{width=4.5in}{heat-wireex-init}
\caption{Initial distribution of temperature in the
wire.\label{heat:wireexinitfig}}
\end{myfig}

We are solving the following PDE problem:
\begin{align*}
& u_t = 0.003 \, u_{xx} , \\
& u(0,t) = u(1,t) = 0 , \\
& u(x,0) = 50\,x\,(1-x) \qquad \text{for } \; 0 < x < 1 .
\end{align*}
We write $f(x) = 50\,x\,(1-x)$ for $0 < x < 1$ as a sine series.  That is,
$
f(x) = \sum_{n=1}^\infty b_n \sin (n \pi x) ,
$
where
\begin{equation*}
b_n = 2 \int_0^1 50\,x\,(1-x) \sin (n \pi x) \,dx
= 
\frac{200}{{\pi }^{3}{n}^{3}}-\frac{200\,{\left( -1\right) }^{n}}{{\pi }^{3}{n}^{3}}
=
\begin{cases}
0 & \text{if } n \text{ even} , \\
\frac{400}{\pi^3 n^3} & \text{if } n \text{ odd} .
\end{cases}
\end{equation*}

The solution $u(x,t)$, plotted in
\figurevref{heat:wireexfig} for $0 \leq t \leq 100$,
is given by the series:
\begin{equation*}
u(x,t) = 
\sum_{\substack{n=1 \\ n \text{ odd}}}^\infty
\frac{400}{\pi^3 n^3}
\sin (n \pi x )
\, e^{-n^2 \pi^2 \, 0.003 \, t} .
\end{equation*}

\begin{myfig}
\capstart
\diffyincludegraphics{width=5in}{width=7.5in}{heat-wireex}
\caption{Plot of the temperature of the wire at position $x$
at time $t$.\label{heat:wireexfig}}
\end{myfig}

Finally, let us answer the question about the maximum temperature.  It is
relatively easy to see
that the maximum temperature at any fixed time is always at $x=0.5$, in
the middle of the wire.  The plot of $u(x,t)$ confirms this intuition.
If we plug in $x=0.5$, we get
\begin{equation*}
u(0.5,t) = 
\sum_{\substack{n=1 \\ n \text{ odd}}}^\infty
\frac{400}{\pi^3 n^3}
\sin (n \pi\, 0.5 )
\, e^{-n^2 \pi^2 \, 0.003 \, t} .
\end{equation*}
For $n=3$ and higher (remember $n$ is only odd), the terms
of the series
are insignificant compared to the first term.
The first term in the series is already a very good approximation
of the function.  
Hence 
\begin{equation*}
u(0.5,t) \approx
\frac{400}{\pi^3}
\, e^{-\pi^2 \, 0.003 \, t} .
\end{equation*}
The approximation gets better and better as $t$ gets larger as the other
terms decay much faster.
Let us plot the function $u(0.5,t)$, the temperature at the midpoint of the wire
at time $t$, in \figurevref{heat:wireexmaxfig}.  The figure also
plots the approximation by the first term.

\begin{myfig}
\capstart
\diffyincludegraphics{width=3in}{width=4.5in}{heat-wireex-max}
\caption{Temperature at the midpoint of the wire (the bottom curve),
and the approximation of this temperature by using only the first term in
the series (top curve).\label{heat:wireexmaxfig}}
\end{myfig}

After $t=5$ or so
it would be hard to tell the difference
between the first term of the series for $u(x,t)$ and 
the real solution $u(x,t)$.  This behavior
is a general feature of solving the heat equation.
If you are interested in behavior for large enough $t$, only the
first one or two terms may be necessary.

Let us
get back to the question of when is the maximum temperature one half of the
initial maximum temperature.  That is, when is the temperature
at the midpoint $\nicefrac{12.5}{2} = 6.25$.  We notice on the graph that if we use
the approximation by the first term we will be close enough.  We
solve
\begin{equation*}
6.25 =
\frac{400}{\pi^3}
\, e^{-\pi^2 \, 0.003 \, t} .
\end{equation*}
That is,
\begin{equation*}
t =
\frac{\ln \frac{6.25\,\pi^3}{400}}{-\pi^2 0.003}
\approx 24.5 .
\end{equation*}
So the maximum temperature drops to half at about $t=24.5$.
\end{example}

We mention an interesting behavior of the solution to the heat equation.
The heat equation
\myquote{smoothes} out the function $f(x)$ as $t$ grows.  For a fixed $t$,
the solution is a Fourier series with coefficients
$b_n e^{\frac{-n^2 \pi^2}{L^2} k t}$.  If $t > 0$, then these coefficients
go to zero faster than any $\frac{1}{n^p}$ for any power $p$.  In other
words, the Fourier series has infinitely many derivatives everywhere.
Thus even if the function $f(x)$ has jumps and corners, then for
a fixed $t > 0$, the solution
$u(x,t)$ as a function of $x$ is as smooth as we want it
to be.

\begin{example}
When the initial condition is already a sine series, then there is no need
to compute anything, you just need to plug in.  Consider
\begin{equation*}
u_t = 0.3 \, u_{xx}, \qquad u(0,t)=u(1,t)=0, \qquad u(x,0) = 0.1 \sin(\pi t) +
\sin(2\pi t) .
\end{equation*}
The solution is then
\begin{equation*}
u(x,t) =
0.1 \sin(\pi t) e^{- 0.3 \pi^2 t}
+ 
\sin(2 \pi t) e^{- 1.2 \pi^2 t} .
\end{equation*}
\end{example}

\subsection{Insulated ends}

Now suppose the ends of the wire are insulated.  In this case, we are solving
the equation
\begin{equation*}
u_t = k u_{xx}
\qquad \text{with} \quad
u_x(0,t) = 0, \quad u_x(L,t) = 0,
\quad \text{and} \quad u(x,0) = f(x) .
\end{equation*}
Yet again we try a solution of the form $u(x,t) = X(x)T(t)$.  By the same
procedure as before we plug into the heat equation and arrive at the
following
two equations
\begin{align*}
X''(x) + \lambda X(x) &= 0 , \\
T'(t) + \lambda k T(t) &= 0 .
\end{align*}
At this point the story changes slightly.
The boundary condition $u_x(0,t) = 0$ implies $X'(0)T(t) = 0$.
Hence $X'(0) = 0$.  Similarly, $u_x(L,t) = 0$ implies $X'(L) = 0$.  We
are looking for nontrivial solutions $X$ of the eigenvalue problem
$X'' + \lambda X = 0$, $X'(0) = 0$, $X'(L) = 0$.  We have previously found that
the only eigenvalues are $\lambda_n = \frac{n^2 \pi^2}{L^2}$, for integers
$n \geq 0$,
where eigenfunctions are $\cos \left( \frac{n \pi}{L} x\right)$
(we include the constant
eigenfunction).  Hence, let us pick
solutions
\begin{equation*}
X_n (x) = \cos \left( \frac{n \pi}{L} x \right)
\qquad \text{and} \qquad
X_0 (x) = 1.
\end{equation*}
The corresponding $T_n$ must satisfy the equation
\begin{equation*}
T_n'(t) + \frac{n^2 \pi^2}{L^2} k T_n(t) = 0 .
\end{equation*}
For $n \geq 1$, as before,
\begin{equation*}
T_n(t) = e^{\frac{-n^2 \pi^2}{L^2} k t} .
\end{equation*}
For $n = 0$, we have $T_0'(t) = 0$ and hence $T_0(t) = 1$.
Our building-block solutions are
\begin{equation*}
u_n(x,t) = X_n(x)T_n(t) =
\cos \left( \frac{n \pi}{L} x \right)
e^{\frac{-n^2 \pi^2}{L^2} k t} ,
\end{equation*}
and
\begin{equation*}
u_0(x,t) = 1 .
\end{equation*}

We note that $u_n(x,0) = \cos \left( \frac{n \pi}{L} x \right)$.  Let us
write $f$ using the cosine series
\begin{equation*}
f(x) = \frac{a_0}{2} + \sum_{n=1}^\infty a_n \cos \left( \frac{n \pi}{L} x
\right) .
\end{equation*}
That is, we find the Fourier series of the even periodic extension of $f(x)$.

We use superposition to write the solution as
\begin{equation*}
\mybxbg{~~
u(x,t) = 
\frac{a_0}{2} + 
\sum_{n=1}^\infty
a_n
u_n(x,t)
=
\frac{a_0}{2} + 
\sum_{n=1}^\infty
a_n
\cos \left( \frac{n \pi}{L} x \right)
e^{\frac{-n^2 \pi^2}{L^2} k t} .
~~}
\end{equation*}

\begin{example}
Let us try the same equation as before, but for insulated ends.
We are solving the following PDE problem
\begin{align*}
& u_t = 0.003 \, u_{xx} , \\
& u_x(0,t) = u_x(1,t) = 0 , \\
& u(x,0) = 50\,x\,(1-x) \qquad \text{for } \; 0 < x < 1 .
\end{align*}

For this problem, we must find the cosine series of $u(x,0)$.
For $0 < x < 1$ we have
\begin{equation*}
50\, x\,(1-x)
=
\frac{25}{3} +
\sum_{\substack{n=2 \\ n \text{ even}}}^\infty
\left( \frac{-200}{\pi^2 n^2} \right)
\cos (n \pi x) .
\end{equation*}
The calculation is left to the reader.
Hence, the solution to the PDE problem, plotted in
\figurevref{heat:wireisolexfig}, is given by the series
\begin{equation*}
u(x,t)
=
\frac{25}{3} +
\sum_{\substack{n=2 \\ n \text{ even}}}^\infty
\left( \frac{-200}{\pi^2 n^2} \right)
\cos ( n \pi x)
\, e^{-n^2 \pi^2 \, 0.003 \, t} .
\end{equation*}

\begin{myfig}
\capstart
\diffyincludegraphics{width=5in}{width=7.5in}{heat-wireisolex}
\caption{Plot of the temperature of the insulated wire at position $x$
at time $t$.\label{heat:wireisolexfig}}
\end{myfig}

Note in the graph
that as time goes on, the temperature evens out across the wire.  Eventually, all the
terms except the constant
die out, and you will be left with a uniform temperature
of $\frac{25}{3} \approx 8.33$ along the entire length of the wire.
\end{example}

Let us expand on the last point.  The constant term in the series is
\begin{equation*}
\frac{a_0}{2} = \frac{1}{L} \int_0^L f(x) \, dx .
\end{equation*}
In other words, $\frac{a_0}{2}$ is the average value of $f(x)$, that is,
the average of the initial temperature.  As the wire is insulated
everywhere, no heat can get out, no heat can get in.  So the temperature
tries to distribute evenly over time, and the average temperature must always be the
same, in particular it is always $\frac{a_0}{2}$.  As time goes to
infinity, the temperature goes to the constant $\frac{a_0}{2}$ everywhere.

\subsection{Exercises}

\begin{exercise}
Consider a wire of length 2, with $k=0.001$ and an initial
temperature distribution $u(x,0) = 50 x$.  Both ends
are embedded in ice (temperature 0).  Find the solution as a series.
\end{exercise}

\begin{exercise}
Find a series solution of
\begin{align*}
& u_t =  u_{xx} , \\
& u(0,t) = u(1,t) = 0 , \\
& u(x,0) = 100 \qquad \text{for } \; 0 < x < 1 .
\end{align*}
\end{exercise}

\begin{exercise}
Find a series solution of
\begin{align*}
& u_t =  u_{xx} , \\
& u_x(0,t) = u_x(\pi,t) = 0 , \\
& u(x,0) = 3\cos (x) + \cos (3x) \qquad \text{for } \; 0 < x < \pi .
\end{align*}
\end{exercise}

\begin{exercise} \label{heat:cosexr}
Find a series solution of
\begin{align*}
& u_t = \frac{1}{3} u_{xx} , \\
& u_x(0,t) = u_x(\pi,t) = 0 , \\
& u(x,0) = \frac{10x}{\pi} \qquad \text{for } \; 0 < x < \pi .
\end{align*}
\end{exercise}

\begin{exercise} \label{heat:oneto100exr}
Find a series solution of
\begin{align*}
& u_t =  u_{xx} , \\
& u(0,t) = 0 , \quad u(1,t) = 100 , \\
& u(x,0) = \sin (\pi x) \qquad \text{for } \; 0 < x < 1 .
\end{align*}
Hint: Use the fact that $u(x,t) = 100 x$ is a solution satisfying
$u_t = u_{xx}$, $u(0,t) = 0$, $u(1,t) = 100$.  Then use superposition.
\end{exercise}

\begin{exercise}
Find the \emph{\myindex{steady state temperature}} solution as a function
of $x$ alone,
by letting $t \to
\infty$ in the solution from
exercises \ref{heat:cosexr} and \ref{heat:oneto100exr}.
Verify that it satisfies the equation $u_{xx} = 0$.
\end{exercise}

\begin{exercise}
Use separation variables to find a nontrivial
solution to $u_{xx} + u_{yy} = 0$, where $u(x,0) = 0$ and $u(0,y) = 0$.
Hint: Try $u(x,y) = X(x)Y(y)$.
\end{exercise}

\begin{exercise}[challenging]
Suppose that one end of the wire is insulated (say at $x=0$) and
the other end is kept at zero temperature.  That is,
find a series solution of
\begin{align*}
& u_t = k u_{xx} , \\
& u_x(0,t) = u(L,t) = 0 , \\
& u(x,0) = f(x) \qquad \text{for } \; 0 < x < L .
\end{align*}
Express any coefficients in the series by integrals of $f(x)$.
\end{exercise}

\begin{exercise}[challenging]
Suppose that the wire is circular and insulated, so there are no ends. 
You can think of this as simply connecting the two ends and 
making sure the solution matches up at the ends.
That is, find a series solution of
\begin{align*}
& u_t = k u_{xx} , \\
& u(0,t) = u(L,t) , \qquad
u_x(0,t) = u_x(L,t) , \\
& u(x,0) = f(x) \qquad \text{for } \; 0 < x < L .
\end{align*}
Express any coefficients in the series by integrals of $f(x)$.
\end{exercise}

\begin{exercise}
Consider a wire insulated on both ends, $L=1$, $k=1$,
and $u(x,0) = \cos^2(\pi x)$.
\begin{tasks}
\task
Find the solution $u(x,t)$.  Hint: a trig identity.
\task
Find the average temperature.
\task
Initially the temperature variation is 1 (maximum minus the minimum).
Find the time when the variation is $\nicefrac{1}{2}$.
\end{tasks}
\end{exercise}

\setcounter{exercise}{100}

%u(x,t) = 
%\sum_{n=1}^\infty
%b_n
%u_n(x,t)
%=
%\sum_{n=1}^\infty
%b_n
%\sin \left( \frac{n \pi}{L} x \right)
%e^{\frac{-n^2 \pi^2}{L^2} k t} .

\begin{exercise}
Find a series solution of
\begin{align*}
& u_t =  3 u_{xx} , \\
& u(0,t) = u(\pi,t) = 0 , \\
& u(x,0) = 5\sin (x) + 2\sin (5x) \qquad \text{for } \; 0 < x < \pi .
\end{align*}
\end{exercise}
\exsol{%
$u(x,t) = 
5
\sin (x)
\, e^{- 3 t}
+
2
\sin (5x)
\, e^{-75 t}$
}

%u(x,t) = 
%\frac{a_0}{2} + 
%\sum_{n=1}^\infty
%a_n
%u_n(x,t)
%=
%\frac{a_0}{2} + 
%\sum_{n=1}^\infty
%a_n
%\cos \left( \frac{n \pi}{L} x \right)
%\, e^{\frac{-n^2 \pi^2}{L^2} k t} .

\begin{exercise}
Find a series solution of
\begin{align*}
& u_t =  0.1 u_{xx} , \\
& u_x(0,t) = u_x(\pi,t) = 0 , \\
& u(x,0) = 1 + 2\cos (x) \qquad \text{for } \; 0 < x < \pi .
\end{align*}
\end{exercise}
\exsol{%
$u(x,t) = 
1 + 
2
\cos (x)
\, e^{-0.1 t}$
}

\begin{exercise}
Use separation of variables to find a nontrivial solution to
$u_{xt} = u_{xx}$.
\end{exercise}
\exsol{%
$u(x,t) = e^{\lambda t} e^{\lambda x}$ for some $\lambda$
}

\begin{exercise}
Use separation of variables 
to find a nontrivial solution to
$u_{x} + u_{t} = u$.  Hint: Try $u(x,t) = X(x)+T(t)$.
\end{exercise}
\exsol{%
$u(x,t) = Ae^x + Be^t$
}

\begin{exercise}
Suppose that the temperature on the wire is fixed at $0$
at the ends, $L=1$, $k=1$, and $u(x,0) = 100\sin(2 \pi x)$.
\begin{tasks}
\task
What is the temperature at $x = \nicefrac{1}{2}$ at any time.
\task
What is the maximum and the minimum temperature on the wire
at $t=0$.
\task
At what time is the maximum temperature on the wire exactly
one half of the initial maximum at $t=0$.
\end{tasks}
\end{exercise}
\exsol{%
a) $0$,
\quad b) minimum $-100$, maximum $100$,
\quad c) $t = \frac{\ln 2}{4 \pi^2}$.
}

%%%%%%%%%%%%%%%%%%%%%%%%%%%%%%%%%%%%%%%%%%%%%%%%%%%%%%%%%%%%%%%%%%%%%%%%%%%%%%

\sectionnewpage
\section{One-dimensional wave equation} \label{we:section}

\sectionnotes{1 lecture\EPref{, \S9.6 in \cite{EP}}\BDref{,
\S10.7 in \cite{BD}}}

Imagine we have a tensioned guitar string of length $L$.  Let us 
only consider vibrations in one direction.  Let
$x$ denote the position along the string, let $t$ denote time, and let $y$
denote the displacement of the string from the rest position.
See
\figurevref{we:vibstrfig}.

\begin{myfig}
\capstart
\inputpdft{sps-vibstr}
\caption{Vibrating string of length $L$, $x$ is position, $y$ is displacement.\label{we:vibstrfig}}
\end{myfig}

The equation that governs this setup is the so-called
\emph{\myindex{one-dimensional wave equation}}\index{wave equation}:
\begin{equation*}
\mybxbg{~~
y_{tt} =
a^2 y_{xx} ,
~~}
\end{equation*}
for some constant $a > 0$.
The intuition is similar to the heat equation, replacing velocity with
acceleration: the acceleration at a specific point is proportional to the second
derivative of the shape of the string.  In other words
when the string is
concave down then $u_{xx}$ is negative and the string wants to accelerate
downwards, so $u_{tt}$ should be negative.  And vice versa.
The wave equation is an example of a hyperbolic PDE.

Assume that the ends of the string are fixed in place as on the guitar:
\begin{equation*}
y(0,t) = 0 \qquad \text{and} \qquad y(L,t) = 0.
\end{equation*}
Note that we have two conditions along the $x$-axis as there are
two derivatives in the $x$ direction.

There are also two derivatives along the $t$ direction and hence we need
two further conditions here.  We need to know the initial position
and the initial velocity of the string.  That is,
for some known functions $f(x)$ and $g(x)$, we impose
\begin{equation*}
y(x,0) = f(x)  \qquad \text{and} \qquad y_t (x,0) =
g(x) .
\end{equation*}

The equation is linear, so superposition works just as it did for the
heat equation.  And again we will use separation of variables to find
enough building-block solutions to get the overall solution.  There is
one change however.  It will be easier to solve two separate problems
and add their solutions.

The two problems we will solve are
\begin{equation} \label{wave:weq}
\begin{array}{ll}
w_{tt} = a^2 w_{xx} , &  \\
w(0,t) = w(L,t) = 0 , &  \\
w(x,0) = 0 & \qquad \text{for } \; 0 < x < L , \\
w_t(x,0) = g(x) & \qquad \text{for } \; 0 < x < L ,
\end{array}
\end{equation}
and
\begin{equation} \label{wave:zeq}
\begin{array}{ll}
z_{tt} = a^2 z_{xx} , &  \\
z(0,t) = z(L,t) = 0 , &  \\
z(x,0) = f(x) & \qquad \text{for } \; 0 < x < L , \\
z_t(x,0) = 0 & \qquad \text{for } \; 0 < x < L .
\end{array}
\end{equation}

The principle of superposition implies that
$y = w + z$ solves the wave equation and furthermore
$y(x,0) = w(x,0) + z(x,0) = f(x)$ and
$y_t(x,0) = w_t(x,0) + z_t(x,0) = g(x)$.  Hence, $y$ is
a solution to
\begin{equation} \label{wave:yeq}
\begin{array}{ll}
y_{tt} = a^2 y_{xx} , &  \\
y(0,t) = y(L,t) = 0 , &  \\
y(x,0) = f(x) & \qquad \text{for } \; 0 < x < L , \\
y_t(x,0) = g(x) & \qquad \text{for } \; 0 < x < L .
\end{array}
\end{equation}

The reason for all this complexity is that superposition only works for
homogeneous conditions such as
$y(0,t) = y(L,t) = 0$, $y(x,0) = 0$, or $y_t(x,0) = 0$.  Therefore,
we can
use separation of variables to find many building-block
solutions solving all the homogeneous conditions.  We can then use them to
construct a solution satisfying the remaining nonhomogeneous condition.

Let us start with \eqref{wave:weq}.
We try a solution of the form $w(x,t) = X(x) T(t)$ again.  We plug into
the wave equation to obtain
\begin{equation*}
X(x)T''(t) = a^2 X''(x) T(t) .
\end{equation*}
Rewriting we get
\begin{equation*}
\frac{T''(t)}{a^2 T(t)} = \frac{X''(x)}{X(x)} .
\end{equation*}
Again, left-hand side depends only on $t$ and the right-hand side depends
only on $x$.  So both sides equal a constant, which we denote by
$-\lambda$:
\begin{equation*}
\frac{T''(t)}{a^2 T(t)} = -\lambda = \frac{X''(x)}{X(x)} .
\end{equation*}
We solve to get two ordinary differential equations
\begin{align*}
X''(x) + \lambda X(x) &= 0 , \\
T''(t) + \lambda a^2 T(t) &= 0 .
\end{align*}
The condition $0 = w(0,t) = X(0) T(t)$ implies $X(0) = 0$ and
$w(L,t) = 0$ implies that $X(L) = 0$.  Therefore, the only nontrivial
solutions for the first equation are when
$\lambda = \lambda_n = \frac{n^2 \pi^2}{L^2}$ and they are
\begin{equation*}
X_n(x) = \sin \left( \frac{n \pi}{L} x \right) .
\end{equation*}
The general solution for $T$ for this particular $\lambda_n$ is
\begin{equation*}
T_n(t) = A \cos \left( \frac{n \pi a}{L} t \right)
+ B \sin \left( \frac{n \pi a}{L} t \right).
\end{equation*}
We also have the condition that $w(x,0) = 0$ or $X(x)T(0) = 0$.  This
implies that $T(0) = 0$, which in turn forces $A = 0$.  It is
convenient to pick $B=\frac{L}{n \pi a}$ (you will see why in a moment)
and hence
\begin{equation*}
T_n(t) = \frac{L}{n \pi a} \sin \left( \frac{n \pi a}{L} t \right).
\end{equation*}
Our building-block solutions are
\begin{equation*}
w_n(x,t) = 
\frac{L}{n \pi a} 
\sin \left( \frac{n \pi}{L} x \right)
\sin \left( \frac{n \pi a}{L} t \right) .
\end{equation*}
We differentiate in $t$:
\begin{equation*}
\frac{\partial w_n}{\partial t}(x,t) = 
\sin \left( \frac{n \pi}{L} x \right)
\cos \left( \frac{n \pi a}{L} t \right) .
\end{equation*}
Hence,
\begin{equation*}
\frac{\partial w_n}{\partial t}(x,0) =
\sin \left( \frac{n \pi}{L} x \right) .
\end{equation*}
We expand $g(x)$ in terms of these sines as
\begin{equation*}
g(x) =
\sum_{n=1}^\infty b_n \sin \left( \frac{n \pi}{L} x \right) .
\end{equation*}
Using superposition
we write the solution to \eqref{wave:weq} as a series
\begin{equation*}
w(x,t) =
\sum_{n=1}^\infty
b_n
w_n(x,t)
=
\sum_{n=1}^\infty
b_n
\frac{L}{n \pi a}
\sin \left( \frac{n \pi}{L} x \right)
\sin \left( \frac{n \pi a}{L} t \right) .
\end{equation*}

\begin{exercise}
Check that $w(x,0) = 0$ and
$w_t(x,0) = g(x)$.
\end{exercise}

We solve \eqref{wave:zeq} similarly.  We again try
$z(x,y) = X(x)T(t)$.  The procedure works exactly the same at first.
We obtain
\begin{align*}
X''(x) + \lambda X(x) &= 0 , \\
T''(t) + \lambda a^2 T(t) &= 0 ,
\end{align*}
and the conditions $X(0) = 0$, $X(L) = 0$.  So again
$\lambda = \lambda_n = \frac{n^2 \pi^2}{L^2}$ and
\begin{equation*}
X_n(x) = \sin \left( \frac{n \pi}{L} x \right) .
\end{equation*}
This time
the condition on $T$ is $T'(0) = 0$.  Thus 
we get that $B = 0$ and we take
\begin{equation*}
T_n(t) = \cos \left( \frac{n \pi a}{L} t \right).
\end{equation*}
Our building-block solution is
\begin{equation*}
z_n(x,t) = 
\sin \left( \frac{n \pi}{L} x \right)
\cos \left( \frac{n \pi a}{L} t \right) .
\end{equation*}
As $z_n(x,0) = \sin \left( \frac{n \pi}{L} x \right)$,
we expand $f(x)$ in terms of these sines as
\begin{equation*}
f(x) =
\sum_{n=1}^\infty c_n \sin \left( \frac{n \pi}{L} x \right) .
\end{equation*}
And we
write down the solution to \eqref{wave:zeq} as a series
\begin{equation*}
z(x,t) =
\sum_{n=1}^\infty
c_n
z_n(x,t)
=
\sum_{n=1}^\infty
c_n
\sin \left( \frac{n \pi}{L} x \right)
\cos \left( \frac{n \pi a}{L} t \right) .
\end{equation*}

\begin{exercise}
Fill in the details in the derivation of the solution of \eqref{wave:zeq}.
Check that the solution satisfies all the side conditions.
\end{exercise}

Putting these two solutions together, let us state the result as a theorem.
\begin{theorem}
Take the equation
\begin{equation} \label{wave:tyeq}
\begin{array}{ll}
y_{tt} = a^2 y_{xx} , &  \\
y(0,t) = y(L,t) = 0 , &  \\
y(x,0) = f(x) & \qquad \text{for } \; 0 < x < L , \\
y_t(x,0) = g(x) & \qquad \text{for } \; 0 < x < L ,
\end{array}
\end{equation}
where
\begin{equation*}
f(x) =
\sum_{n=1}^\infty c_n \sin \left( \frac{n \pi}{L} x \right)
\qquad \text{and} \qquad
g(x) =
\sum_{n=1}^\infty b_n \sin \left( \frac{n \pi}{L} x \right) .
\end{equation*}
Then the solution $y(x,t)$ can be written as a sum of the solutions
of \eqref{wave:weq} and \eqref{wave:zeq}:
\begin{equation*}
\mybxbg{~~
\begin{aligned}
y(x,t)
& =
\sum_{n=1}^\infty
b_n
\frac{L}{n \pi a}
\sin \left( \frac{n \pi}{L} x \right)
\sin \left( \frac{n \pi a}{L} t \right) 
+
c_n
\sin \left( \frac{n \pi}{L} x \right)
\cos \left( \frac{n \pi a}{L} t \right) 
\\
& =
\sum_{n=1}^\infty
\sin \left( \frac{n \pi}{L} x \right)
\left[
b_n
\frac{L}{n \pi a}
\sin \left( \frac{n \pi a}{L} t \right) 
+
c_n
\cos \left( \frac{n \pi a}{L} t \right) 
\right] .
\end{aligned}
~~}
\end{equation*}
\end{theorem}

\begin{example} \label{example:pluckedstring}
Consider a string of length 2 plucked in the middle,
it has an initial shape given in \figurevref{wave:pluckedstrfig}.
That is,
\begin{equation*}
f(x) = \begin{cases}
0.1\, x & \text{if } \; 0 \leq x \leq 1 , \\
0.1\, (2-x) & \text{if } \; 1 < x \leq 2 .
\end{cases}
\end{equation*}

\begin{myfig}
\capstart
\inputpdft{wave-pluckedstr}
\caption{Initial shape of a plucked string from
\exampleref{example:pluckedstring}.\label{wave:pluckedstrfig}}
\end{myfig}

Let the string start at rest ($g(x) =
0$), and let $a=1$ for simplicity.  In other words, we wish to
solve the problem:
\begin{align*}
& y_{tt} = y_{xx}, \\
& y(0,t) = y(2,t)= 0 , \\
& y(x,0) = f(x) \quad \text{and} \quad y_t(x,0)= 0 .
\end{align*}

We leave it to the reader to compute the sine series of $f(x)$.  The series
will be
\begin{equation*}
f(x) = \sum_{n=1}^\infty
\frac{0.8}{n^2 \pi^2}
\sin \left( \frac{n \pi}{2} \right)
\sin \left( \frac{n \pi}{2} x \right) .
\end{equation*}
Note that 
$\sin \left( \frac{n \pi}{2} \right)$
is the sequence $1, 0, -1, 0, 1, 0, -1, \ldots$
for $n = 1,2,3,4,\ldots$.  Therefore,
\begin{equation*}
f(x) = 
\frac{0.8}{\pi^2}
\sin \left( \frac{\pi}{2} x \right)
-
\frac{0.8}{9 \pi^2}
\sin \left( \frac{3 \pi}{2} x \right)
+
\frac{0.8}{25 \pi^2}
\sin \left( \frac{5 \pi}{2} x \right)
- \cdots
\end{equation*}
The solution $y(x,t)$ is given by
\begin{equation*}
\begin{split}
y(x,t) & = 
\sum_{n=1}^\infty
\frac{0.8}{n^2 \pi^2}
\sin \left( \frac{n \pi}{2} \right)
\sin \left( \frac{n \pi}{2} x \right)
\cos \left( \frac{n \pi}{2} t \right)
\\
& = 
\sum_{m=1}^\infty
\frac{0.8 {(-1)}^{m+1}}{{(2m-1)}^2 \pi^2}
\sin \left( \frac{(2m-1) \pi}{2} x \right)
\cos \left( \frac{(2m-1) \pi}{2} t \right)
\\
& =
\frac{0.8}{\pi^2} 
\sin \left( \frac{\pi}{2}  x \right)
\cos \left( \frac{\pi}{2}  t \right)
-
\frac{0.8}{9 \pi^2} 
\sin \left( \frac{3 \pi}{2}  x \right)
\cos \left( \frac{3 \pi}{2}  t \right)
\\
& \hspace{20em}
+
\frac{0.8}{25 \pi^2}
\sin \left( \frac{5 \pi}{2}  x \right)
\cos \left( \frac{5 \pi}{2}  t \right) 
- \cdots
\end{split}
\end{equation*}

See 
\figurevref{wave:pluckedexfig} for a plot
for $0 < t < 3$.  Notice
that unlike the heat equation, the solution does not become
\myquote{smoother,}
the \myquote{sharp edges} remain.  We will see the reason for this behavior in the
next section where we derive the solution to the wave equation in a different
way.

\begin{myfig}
\capstart
\diffyincludegraphics{width=5in}{width=7.5in}{wave-pluckedex}
\caption{Shape of the plucked string for $0 < t < 3$.\label{wave:pluckedexfig}}
\end{myfig}

Make sure you understand what the plot, such as the one in the figure, is
telling you.  For each fixed $t$, you can think of the function 
$y(x,t)$ as just a function of $x$.  This function gives you the shape of the
string at time $t$.  See \figurevref{wave:pluckedtsfig} for plots of
at $y$ as a function of $x$ at several different values of $t$.
On this plot you can see the sharp edges remaining much better.

\begin{myfig}
\capstart
%original files wave-plucked-slicet0 wave-plucked-slicet0p4 wave-plucked-slicet0p8 wave-plucked-slicet1p2
\diffyincludegraphics{width=6.24in}{width=9in}{wave-plucked-t0-t0p4}
\\[5pt]
\diffyincludegraphics{width=6.24in}{width=9in}{wave-plucked-t0p8-t1p2}
\caption{Plucked string for $t=0$, $t=0.4$, $t=0.8$, and
$t=1.2$.%
\label{wave:pluckedtsfig}}
\end{myfig}
\end{example}

One thing to take away from all this is how a guitar sounds.  Notice that
the (angular) frequencies that come up in the solution are
$n \frac{\pi a}{L}$.  That is, there is a certain base
\emph{\myindex{fundamental frequency}} $\frac{\pi a}{L}$, and then we also
get all the multiples of this frequency, which in music are called
the \emph{\myindex{overtones}}.  Which overtones appear and with what amplitude
is what musicians call the \emph{\myindex{timbre}} of the note.
Mathematicians usually call this the \emph{\myindex{spectrum}}.
Because all the frequencies are multiples of one frequency (the fundamental)
we get a nice pleasing sound.

The fundamental frequency $\frac{\pi a}{L}$ increases as we decrease length $L$.  That is, if
we place a finger on the fingerboard and then pluck a string we get a higher
note.  The constant $a$ is given by
\begin{equation*}
a = \sqrt{\frac{T}{\rho}} ,
\end{equation*}
where $T$ is tension and $\rho$ is the linear density of the string.
Tightening the string (turning the tuning peg on a guitar) increases $a$ and
hence produces a higher fundamental frequency (a higher note).
On the other hand using a heavier string 
reduces $a$ and produces a lower fundamental frequency (a lower note).
A bass guitar has longer thicker strings, while a ukulele has short strings
made of lighter material.

Something rather interesting is the almost symmetry between space and time.
In its simplest form we see this symmetry in the solutions
\begin{equation*}
\sin \left( \frac{n \pi}{L} x \right)
\sin \left( \frac{n \pi a}{L} t \right)  .
\end{equation*}
Except for the $a$, time and space are just the same.

In general, the solution for a fixed $x$ is a Fourier series in $t$, for
a fixed $t$ it is a Fourier series in $x$, and the coefficients are related.
If the shape $f(x)$ or the initial velocity have lots of corners, then
the sound wave will have lots of corners.  That is because the Fourier coefficients
of the initial shape decay to zero (as $n \to \infty$) at the same rate as the Fourier coefficients
of the wave in time (for some fixed $x$).  So if you use a sharp object to
pick the string, you get a sharper sound with lots of high frequency
components, while if you use your thumb, you get a softer sound without
so many high overtones.  Similarly if you pluck close to the bridge, you are
getting a pluck that looks more like the sawtooth, and you get an even
sharper sound.

In fact, if you look at the formula for the solution, you see that for any
fixed $x$ we get an almost arbitrary Fourier series in $t$, everything
except the constant term.  You can essentially obtain any sound you want
by plucking the string in just the right way.
Of course we are considering an ideal string of no stiffness and no air
resistance.  Those variables clearly impact the sound as well.

\subsection{Exercises}

\begin{exercise}
Solve
\begin{equation*}
\begin{array}{ll}
y_{tt} = 9 y_{xx} , &  \\
y(0,t) = y(1,t) = 0 , &  \\
y(x,0) = \sin (3\pi x) + \frac{1}{4} \sin (6 \pi x) & \qquad \text{for } \; 0 < x < 1 , \\
y_t(x,0) = 0 & \qquad \text{for } \; 0 < x < 1 .
\end{array}
\end{equation*}
\end{exercise}

\begin{exercise}
Solve
\begin{equation*}
\begin{array}{ll}
y_{tt} = 4 y_{xx} , &  \\
y(0,t) = y(1,t) = 0 , &  \\
y(x,0) = \sin (3\pi x) + \frac{1}{4} \sin (6 \pi x) & \qquad \text{for } \; 0 < x < 1 , \\
y_t(x,0) = \sin (9 \pi x) & \qquad \text{for } \; 0 < x < 1 .
\end{array}
\end{equation*}
\end{exercise}

\begin{exercise}
Derive the solution for a general plucked string of length $L$ and
any constant $a$ (in the equation $y_{tt} = a^2 y_{xx}$), where we
raise the string some distance $b$ at the midpoint and let go.
\end{exercise}

\begin{samepage}
\begin{exercise}
Imagine that a stringed musical instrument falls on the floor.  Suppose that
the length of the string is 1 and $a=1$.  When the musical instrument hits
the ground the string was in rest position and hence $y(x,0) = 0$.  However,
the string was moving at some velocity at impact ($t=0$),
say $y_t(x,0) = -1$.  Find the
solution
$y(x,t)$ for the shape of the string at time $t$.
\end{exercise}
\end{samepage}

\begin{exercise}[challenging]
Suppose that you have a vibrating string and that
there is air resistance proportional to the velocity.  That is, you have
\begin{equation*}
\begin{array}{ll}
y_{tt} = a^2 y_{xx} - k y_t , &  \\
y(0,t) = y(1,t) = 0 , &  \\
y(x,0) = f(x) & \qquad \text{for } \; 0 < x < 1 , \\
y_t(x,0) = 0 & \qquad \text{for } \; 0 < x < 1 .
\end{array}
\end{equation*}
Suppose that $0 < k < 2 \pi a$.
Derive a series solution to the problem.  Any coefficients in the series
should be expressed as integrals of $f(x)$.
\end{exercise}

\begin{exercise}
Suppose you touch the guitar string exactly in the middle to
ensure another condition $u(\nicefrac{L}{2},t) = 0$ for all time.
Which multiples of the fundamental frequency $\frac{\pi a}{L}$
show up in the solution?
\end{exercise}

\setcounter{exercise}{100}

\begin{exercise}
Solve
\begin{equation*}
\begin{array}{ll}
y_{tt} = y_{xx} , &  \\
y(0,t) = y(\pi,t) = 0 , &  \\
y(x,0) = \sin(x) & \qquad \text{for } \; 0 < x < \pi , \\
y_t(x,0) = \sin(x) & \qquad \text{for } \; 0 < x < \pi .
\end{array}
\end{equation*}
\end{exercise}
\exsol{%
$
y(x,t)
=
\sin(x)
\bigl(\sin(t) + \cos(t)\bigr)
$
}

\begin{exercise}
Solve
\begin{equation*}
\begin{array}{ll}
y_{tt} = 25 y_{xx} , &  \\
y(0,t) = y(2,t) = 0 , &  \\
y(x,0) = 0 & \qquad \text{for } \; 0 < x < 2 , \\
y_t(x,0) = \sin(\pi t) + 0.1 \sin(2\pi t) & \qquad \text{for } \; 0 < x < 2 .
\end{array}
\end{equation*}
\end{exercise}
\exsol{%
$y(x,t)
=
\frac{1}{5 \pi}
\sin (\pi x)
\sin (5 \pi t)
+
\frac{1}{100 \pi}
\sin(2\pi x)
\sin(10\pi t)$
}
%$
%y(x,t)
%=
%\sum\limits_{n=1}^\infty
%\sin \left( \frac{n \pi}{L} x \right)
%\left[
%b_n
%\frac{L}{n \pi a}
%\sin \left( \frac{n \pi a}{L} t \right) 
%+
%c_n
%\cos \left( \frac{n \pi a}{L} t \right) 
%\right] .

\begin{exercise}
Solve
\begin{equation*}
\begin{array}{ll}
y_{tt} = 2 y_{xx} , &  \\
y(0,t) = y(\pi,t) = 0 , &  \\
y(x,0) = x & \qquad \text{for } \; 0 < x < \pi , \\
y_t(x,0) = 0 & \qquad \text{for } \; 0 < x < \pi .
\end{array}
\end{equation*}
\end{exercise}
\exsol{%
$
y(x,t)
=
\sum\limits_{n=1}^\infty
\frac{2{(-1)}^{n+1}}{n}
\sin(nx)
\cos( n \sqrt{2}\,t ) 
$
%$
%y(x,t)
%=
%\sum\limits_{n=1}^\infty
%\sin \left( \frac{n \pi}{L} x \right)
%\left[
%b_n
%\frac{L}{n \pi a}
%\sin \left( \frac{n \pi a}{L} t \right) 
%+
%c_n
%\cos \left( \frac{n \pi a}{L} t \right) 
%$
}

\begin{exercise}
Let's see what happens when $a=0$.  Find a solution to
$y_{tt} = 0$, $y(0,t) = y(\pi,t) = 0$,
$y(x,0) = \sin(2x)$,
$y_t(x,0) = \sin(x)$.
\end{exercise}
\exsol{%
$y(x,t) = \sin(2x)+t\sin(x)$
}

%%%%%%%%%%%%%%%%%%%%%%%%%%%%%%%%%%%%%%%%%%%%%%%%%%%%%%%%%%%%%%%%%%%%%%%%%%%%%%

\sectionnewpage
\section{D'Alembert solution of the wave equation}
\label{dalemb:section}

%mbxINTROSUBSECTION

\sectionnotes{1 lecture\EPref{, different from \S9.6 in \cite{EP}}\BDref{,
part of \S10.7 in \cite{BD}}}

We have solved the wave equation by using Fourier series.  But it is often
more convenient to use the so-called
\emph{\myindex{d'Alembert solution to the wave equation}}%
\footnote{Named after the French mathematician
\href{https://en.wikipedia.org/wiki/D\%27Alembert}{Jean le Rond d'Alembert}
(1717--1783).}.
While this solution can be derived using Fourier series as well, it
is really an awkward use of those concepts.  It is easier and more
instructive to derive this
solution by making a correct change of variables to get an equation that
can be solved by simple integration.

Suppose we wish to solve the \myindex{wave equation}
\begin{equation} \label{dalemb:weq}
y_{tt} = a^2 y_{xx}
\end{equation}
subject to the side conditions
\begin{equation} \label{dalemb:weqside}
\begin{aligned}
y(0,t) &=  y(L,t) = 0 &
& \text{for all } t , \\
y(x,0) &= f(x) &  & 0 < x < L , \\
y_t(x,0) &= g(x) &  & 0 < x < L .
\end{aligned}
\end{equation}

\subsection{Change of variables}

We will transform the equation into a simpler form where it can be solved by
simple integration.
We change variables to $\xi = x - at$, $\eta = x + at$.
The chain rule says:
\begin{align*}
& \frac{\partial}{\partial x}
=
\frac{\partial \xi}{\partial x}
\frac{\partial}{\partial \xi}
+
\frac{\partial \eta}{\partial x}
\frac{\partial}{\partial \eta}
=
\frac{\partial}{\partial \xi}
+
\frac{\partial}{\partial \eta} , \\
& \frac{\partial}{\partial t}
=
\frac{\partial \xi}{\partial t}
\frac{\partial}{\partial \xi}
+
\frac{\partial \eta}{\partial t}
\frac{\partial}{\partial \eta}
=
-a
\frac{\partial}{\partial \xi}
+
a
\frac{\partial}{\partial \eta} .
\end{align*}
We compute
\begin{align*}
& y_{xx} = \frac{\partial^2 y}{\partial x^2}
=
\left(
\frac{\partial}{\partial \xi}
+
\frac{\partial}{\partial \eta}
\right)
\left(
\frac{\partial y}{\partial \xi}
+
\frac{\partial y}{\partial \eta}
\right)
=
\frac{\partial^2 y}{\partial \xi^2}
+
2 \frac{\partial^2 y}{\partial \xi \partial \eta}
+
\frac{\partial^2 y}{\partial \eta^2} ,
\\
& y_{tt} = \frac{\partial^2 y}{\partial t^2}
=
\left(
-a
\frac{\partial}{\partial \xi}
+ a
\frac{\partial}{\partial \eta}
\right)
\left(
-a
\frac{\partial y}{\partial \xi}
+
a
\frac{\partial y}{\partial \eta}
\right)
=
a^2
\frac{\partial^2 y}{\partial \xi^2}
-
2 a^2 \frac{\partial^2 y}{\partial \xi \partial \eta}
+
a^2
\frac{\partial^2 y}{\partial \eta^2} .
\end{align*}
In the computations above, we used the fact from calculus that
$\frac{\partial^2 y}{\partial \xi \partial \eta} = 
\frac{\partial^2 y}{\partial \eta \partial \xi}$.
We plug what we got into the wave equation,
\begin{equation*}
0 = a^2 y_{xx} - y_{tt} =
4 a^2 \frac{\partial^2 y}{\partial \xi \partial \eta} = 4 a^2 y_{\xi\eta} .
\end{equation*}
Therefore, the wave equation \eqref{dalemb:weq} transforms into
$y_{\xi\eta} = 0$.
It is easy to find the general solution to this equation by integrating
twice.  Keeping $\xi$ constant, we integrate with respect to $\eta$
first\footnote{There is nothing special about $\eta$, you can integrate with
$\xi$ first, if you wish.}
and notice that
the constant of integration depends on $\xi$; for each $\xi$ we might get a
different constant of integration.  We get
$y_{\xi} = C(\xi)$.
Next, we integrate with respect to $\xi$ and notice that the constant of
integration depends on $\eta$.
Thus,
$y = \int C(\xi) \, d\xi + B(\eta)$.
The solution must, therefore, be of the following form for some functions
$A(\xi)$ and $B(\eta)$:
\begin{equation*}
y = A(\xi) + B(\eta) = A(x-at) + B(x+at) .
\end{equation*}
The solution is a superposition of two functions (waves) traveling at speed
$a$
in opposite directions.  The coordinates $\xi$ and $\eta$ are called the
\emph{\myindex{characteristic coordinates}}, and a similar technique can
be applied to more complicated hyperbolic PDE\@.
In \sectionref{fopde:section} it is used to solve first
order linear PDE\@.  Basically, to solve the wave equation (or more general
hyperbolic equations) we find certain characteristic curves along which
the equation is really just an ODE\@, or a pair of ODEs.  In this case
these are the curves where $\xi$ and $\eta$ are constant.

\subsection{D'Alembert's formula}

We know what any solution must look like, but we need to solve for the
given side conditions.  We will just give the formula and see that it works.
First let $F(x)$
denote the odd periodic extension of $f(x)$, and let $G(x)$ denote the
odd periodic extension of $g(x)$.  Define
\begin{equation*}
A(x) = \frac{1}{2} F(x) - \frac{1}{2a} \int_0^x G(s) \,ds ,
\qquad
B(x) = \frac{1}{2} F(x) + \frac{1}{2a} \int_0^x G(s) \,ds .
\end{equation*}
We claim this $A(x)$ and $B(x)$ give the solution.  Explicitly, the
solution is $y(x,t) = A(x-at) + B(x+at)$ or in other words:
\begin{equation} \label{dalemb:form}
\mybxbg{~~
\begin{aligned}
y(x,t) & =
\frac{1}{2} F(x-at) - \frac{1}{2a} \int_0^{x-at} G(s) \,ds 
+
\frac{1}{2} F(x+at) + \frac{1}{2a} \int_0^{x+at} G(s) \,ds \\
& =
\frac{F(x-at) + F(x+at)}{2} + \frac{1}{2a} \int_{x-at}^{x+at} G(s) \,ds .
\end{aligned}
~~}
\end{equation}

Let us check that the d'Alembert formula really works.
\begin{equation*}
y(x,0) =
\frac{F(x) + F(x)}{2} + \frac{1}{2a} \int_{x}^{x} G(s) \,ds 
=
F(x) .
\end{equation*}
So far so good.  Assume for simplicity $F$ is differentiable.
And we use the first form of \eqref{dalemb:form} as it is
easier to differentiate.
By the fundamental theorem of calculus we have
\begin{equation*}
y_t(x,t) =
\frac{-a}{2} F'(x-at) + \frac{1}{2} G(x-at)
+
\frac{a}{2} F'(x+at) + \frac{1}{2} G(x+at) .
\end{equation*}
So
\begin{equation*}
y_t(x,0) =
\frac{-a}{2} F'(x) + \frac{1}{2} G(x)
+
\frac{a}{2} F'(x) + \frac{1}{2} G(x) = G(x) .
\end{equation*}
Yay!  We're smoking now.  OK\@, now the boundary conditions.  Note
that $F(x)$ and $G(x)$ are odd.  So
\begin{equation*}
y(0,t) =
\frac{F(-at) + F(at)}{2} + \frac{1}{2a} \int_{-at}^{at} G(s) \,ds 
=
\frac{-F(at) + F(at)}{2} + \frac{1}{2a} \int_{-at}^{at} G(s) \,ds 
= 0 + 0 = 0.
\end{equation*}
Now $F(x)$ is odd and $2L$-periodic, so
\begin{equation*}
F(L-at)+F(L+at)
= 
F(-L-at)+F(L+at)
=
-F(L+at)+F(L+at)
= 0 .
\end{equation*}
Next, $G(s)$ is odd and $2L$-periodic, so we change
variables $v = s-L$.  We then notice that $G(v+L)=G(v-L)=-G(-v+L)$,
so $G(v+L)$ is odd as a function of $v$:
\begin{equation*}
\int_{L-at}^{L+at}
G(s)\,ds
=
\int_{-at}^{at}
G(v+L)\,dv
=
0 .
\end{equation*}
Hence
\begin{equation*}
y(L,t) =
\frac{F(L-at) + F(L+at)}{2} + \frac{1}{2a} \int_{L-at}^{L+at} G(s) \,ds 
=
0 + 0  = 0.
\end{equation*}
And voil\`a, it works.

\begin{example} \label{dalemb:implusexample}
D'Alembert says that the solution is a superposition of
two functions (waves) moving in the opposite direction at \myquote{speed} $a$.
To get an idea of
how it works, let us work out an example.  Consider the simpler setup
\begin{align*}
& y_{tt} = y_{xx} , \\
& y(0,t) = y(1,t) = 0 , \\
& y(x,0) = f(x) , \\
& y_t(x,0) = 0 .
\end{align*}
Here $f(x)$ is an impulse of height 1 centered at $x=0.5$:
\begin{equation*}
f(x) =
\begin{cases}
0 & \text{if } \; \phantom{0.5}0 \leq x < 0.45, \\
20\,(x-0.45) & \text{if } \; 0.45 \leq x < 0.5, \\
20\,(0.55-x) & \text{if } \; \phantom{5}0.5 \leq x < 0.55, \\
0 & \text{if } \; 0.55 \leq x \leq 1 .
\end{cases}
\end{equation*}
The graph of this impulse is the top left plot in
\figurevref{dalemb:impulsfig}.

Let $F(x)$ be the odd periodic extension of $f(x)$.  Then
\eqref{dalemb:form} says that
the solution is
\begin{equation*}
y(x,t) = \frac{F(x-t) + F(x+t)}{2} .
\end{equation*}
It is not hard to compute specific values of $y(x,t)$.  For example, to
compute $y(0.1,0.6)$ we notice $x-t = -0.5$ and $x+t = 0.7$.  Now $F(-0.5) =
-f(0.5) = - 20\,(0.55 - 0.5) = -1$
and $F(0.7) = f(0.7) = 0$.  Hence
$y(0.1,0.6) = \frac{-1 + 0}{2} = -0.5$.  As you can see the d'Alembert
solution is much easier to actually compute and to plot than the Fourier series
solution.  See \figurevref{dalemb:impulsfig} for plots of the solution $y$
for several different $t$.
\begin{myfig}
\capstart
%original files dalemb-impuls1 dalemb-impuls2 dalemb-impuls3 dalemb-impuls4
\diffyincludegraphics{width=6.24in}{width=9in}{dalemb-impuls-1-2}
\\[5pt]
\diffyincludegraphics{width=6.24in}{width=9in}{dalemb-impuls-3-4}
\caption{Plot of the d'Alembert solution for $t=0$, $t=0.2$, $t=0.4$, and
$t=0.6$.%
\label{dalemb:impulsfig}}
\end{myfig}
\end{example}

\subsection{Another way to solve for the side conditions}

It is perhaps easier and more useful
to memorize the procedure rather than the formula
itself.  The important thing to remember is that a solution to the wave
equation is a superposition of two waves traveling in opposite directions.
That is,
\begin{equation*}
y(x,t) = A(x-at) + B(x+at) .
\end{equation*}
If you think about it, the exact formulas for $A$ and $B$ are not hard
to guess once you realize what kind of side conditions $y(x,t)$ is supposed to
satisfy.  Let us find the formula again, but slightly differently.
Best approach is to do it in stages.  When $g(x) = 0$ (and hence
$G(x) = 0$) the solution is
\begin{equation*}
\frac{ F(x-at) + F(x+at) }{2} .
\end{equation*}
On the other hand,
when $f(x) = 0$ (and hence $F(x) = 0$), we let
\begin{equation*}
H(x) = \int_0^x G(s) \,ds .
\end{equation*}
The solution in this case is
\begin{equation*}
\frac{1}{2a} \int_{x-at}^{x+at} G(s) \,ds
=
\frac{ -H(x-at) + H(x+at) }{2a} .
\end{equation*}
By superposition we get a solution for the general side conditions
\eqref{dalemb:weqside} (when neither $f(x)$ nor $g(x)$ are identically zero).
\begin{equation} \label{dalemb:altform}
y(x,t) = \frac{ F(x-at) + F(x+at) }{2} +
\frac{ -H(x-at) + H(x+at) }{2a} .
\end{equation}
Do note the minus sign before the $H$, and the $a$ in the second denominator.

\begin{exercise}
Check that the new formula \eqref{dalemb:altform} satisfies
the side conditions
\eqref{dalemb:weqside}.
\end{exercise}

\textbf{Warning:}
Make sure you use the odd periodic extensions $F(x)$ and $G(x)$,
when you have formulas for $f(x)$ and $g(x)$.
The thing is, those formulas in general hold
only for $0 < x < L$, and are not usually equal to $F(x)$ and $G(x)$
for other $x$.

\subsection{Some remarks}

Let us remark that the formula $y(x,t) = A(x-at) + B(x+at)$ is the reason
why the solution of the wave equation doesn't get \myquote{nicer} as time
goes on, that is, why in the examples where the initial conditions
had corners, the solution also has corners at every time $t$.

\medskip

The corners bring us to another interesting remark.  Nobody ever notices at first
that our example solutions are not even differentiable (they have corners):
In \exampleref{dalemb:implusexample} above, the solution is not
differentiable whenever $x=t+0.5$ or $x=-t+0.5$ for example.
Really to be able to compute $u_{xx}$ or $u_{tt}$, you need not one, but two
derivatives.  Fear not, we could think of a shape that is very nearly
$F(x)$ but does have two derivatives by rounding the corners a little bit,
and then the solution would be very nearly
$\frac{F(x-t)+F(x+t)}{2}$ and nobody would notice the switch.

\medskip

One final remark is what the d'Alembert solution tells us about what
part of the initial conditions influence the solution at a certain point.
We can figure this out by \myquote{traveling backwards along the
characteristics.}  Let us suppose that the string is very long (perhaps
infinite) for simplicity.  Since the solution at time $t$ is
\begin{equation*}
y(x,t) =
\frac{F(x-at) + F(x+at)}{2} + \frac{1}{2a} \int_{x-at}^{x+at} G(s) \,ds ,
\end{equation*}
we notice that we have only used the initial conditions in the interval
$[x-at,x+at]$.  These two endpoints are called the
\emph{\myindex{wavefronts}}, as that is where the wave front is given an
initial ($t=0$) disturbance at $x$.
So if $a=1$, an observer sitting at $x=0$ at time $t=1$ has only seen the
initial conditions for $x$ in the range $[-1,1]$
and is blissfully unaware of anything else.
This is why for example we do not know that a supernova has occurred in the
universe until we see its light, millions of years from the time
when it did in fact happen.

\subsection{Exercises}

\begin{exercise}
Using the d'Alembert solution solve $y_{tt} = 4y_{xx}$, $0 < x < \pi$, $t >
0$,
$y(0,t) = y(\pi, t) = 0$, $y(x,0) = \sin x$, and
$y_t(x,0) = \sin x$.  Hint: Note that $\sin x$ is the odd periodic extension of
$y(x,0)$ and $y_t(x,0)$.
\end{exercise}

\begin{exercise}
Using the d'Alembert solution solve $y_{tt} = 2y_{xx}$, $0 < x < 1$, $t > 0$,
$y(0,t) = y(1, t) = 0$, $y(x,0) = \sin^5 (\pi x)$, and
$y_t(x,0) = \sin^3 (\pi x)$.
\end{exercise}

\begin{exercise}
Take
$y_{tt} = 4y_{xx}$, $0 < x < \pi$, $t > 0$,
$y(0,t) = y(\pi, t) = 0$, $y(x,0) = x(\pi-x)$, and
$y_t(x,0) = 0$.
\begin{tasks}
\task Solve using the d'Alembert formula.  Hint: You can use the sine series
for $y(x,0)$.
\task Find the solution as a function of $x$ for a fixed $t=0.5$, $t=1$, and
$t=2$.  Do not use the sine series here.
\end{tasks}
\end{exercise}

\begin{exercise}
Derive the d'Alembert solution for $y_{tt} = a^2 y_{xx}$, $0 < x < \pi$, $t >
0$,
$y(0,t) = y(\pi, t) = 0$, $y(x,0) = f(x)$, and
$y_t(x,0) = 0$, using the Fourier series solution of the wave equation,
by applying an appropriate trigonometric identity.
Hint: Do it first for a single term of the Fourier series solution,
in particular do it when $y$ is
$\sin\left(\frac{n \pi}{L} x \right)\sin\left(\frac{n \pi a}{L} t \right)$.
\end{exercise}

\begin{exercise}
The d'Alembert solution still works if there are no boundary conditions and
the initial condition is defined on the whole real line.  Suppose that
$y_{tt} = y_{xx}$ (for all $x$ on the real line and $t \geq 0$),
$y(x,0) = f(x)$, and
$y_t(x,0) = 0$, where
\begin{equation*}
f(x) =
\begin{cases}
0 & \text{if } \; \phantom{{-1} \leq {} }x < -1, \\
x+1 & \text{if } \; {-1} \leq x < 0, \\
-x+1 & \text{if } \; \phantom{-}0 \leq x < 1, \\
0 & \text{if } \; \phantom{-}1 < x .
\end{cases}
\end{equation*}
Solve using the d'Alembert solution. That is, write down a piecewise
definition for the solution.  Then sketch the solution for $t=0$,
$t=\nicefrac{1}{2}$, $t=1$, and $t=2$.
\end{exercise}

\setcounter{exercise}{100}

\begin{exercise}
Using the d'Alembert solution solve $y_{tt} = 9y_{xx}$, $0 < x < 1$, $t >
0$,
$y(0,t) = y(1, t) = 0$, $y(x,0) = \sin (2 \pi x)$, and
$y_t(x,0) = \sin (3 \pi x)$.
\end{exercise}
\exsol{%
$y(x,t)=
\frac{\sin(2 \pi (x-3 t))+\sin(2 \pi (3 t+x))}{2}
+
\frac{\cos(3 \pi (x-3 t))-\cos(3 \pi (3 t+x))}{18\pi}$
}

\begin{exercise}
Take $y_{tt} = 4y_{xx}$, $0 < x < 1$, $t > 0$,
$y(0,t) = y(1, t) = 0$, $y(x,0) = x-x^2$, and
$y_t(x,0) = 0$.  Using the d'Alembert solution find
the solution at
\begin{tasks}(3)
\task $t=0.1$,
\task $t=\nicefrac{1}{2}$,
\task $t=1$.
\end{tasks}
You may have to split your answer up by cases.
\end{exercise}
\exsol{%
a)
$y(x,0.1) =
\begin{cases}
%\frac{(x-0.2)-{(x-0.2)}^2+(x+0.2)-{(x+0.2)}^2}{2}
x-x^2-0.04
& \text{if } \; 0.2 \leq x \leq 0.8 \\
%\frac{-(x+0.8)+{(x+0.8)}^2+(x+0.2)-{(x+0.2)}^2}{2}
0.6x
& \text{if } \; x \leq 0.2 \\
%\frac{(x-0.2)-{(x-0.2)}^2-(x-0.8)+{(x-0.8)}^2}{2}
0.6-0.6x
& \text{if } \; x \geq 0.8 \\
\end{cases}$
\\
b)
$y(x,\nicefrac{1}{2}) = -x+x^2$
\quad
c)
$y(x,1) = x-x^2$
}

\begin{exercise}
Take $y_{tt} = 100y_{xx}$, $0 < x < 4$, $t > 0$,
$y(0,t) = y(4, t) = 0$, $y(x,0) = F(x)$, and
$y_t(x,0) = 0$.  Suppose that
$F(0)=0$,
$F(1)=2$,
$F(2)=3$,
$F(3)=1$.
Using the d'Alembert solution find
\begin{tasks}(3)
\task $y(1,1)$,
\task $y(4,3)$,
\task $y(3,9)$.
\end{tasks}
\end{exercise}
\exsol{%
a) $y(1,1) = -\nicefrac{1}{2}$
\quad
b) $y(4,3) = 0$
\quad
c) $y(3,9) = \nicefrac{1}{2}$
}

%%%%%%%%%%%%%%%%%%%%%%%%%%%%%%%%%%%%%%%%%%%%%%%%%%%%%%%%%%%%%%%%%%%%%%%%%%%%%%

\sectionnewpage
\section{Steady state temperature and the Laplacian}
\label{dirich:section}

\sectionnotes{1 lecture\EPref{, \S9.7 in \cite{EP}}\BDref{,
\S10.8 in \cite{BD}}}

Consider an insulated wire, a plate, or a 3-dimensional object.
We apply
certain fixed temperatures on the ends of the wire, the edges of the plate,
or on all sides of the 3-dimensional object.  We wish to find out what is the
\emph{\myindex{steady state temperature}} distribution.  That is, we wish to know what will
be the temperature after long enough period of time.

We are really looking for a solution to the heat equation that is not
dependent on time.  Let us first solve the problem in one space variable.  We are
looking for a function $u$ that satisfies
\begin{equation*}
u_t = k u_{xx} ,
\end{equation*}
but such that $u_t = 0$ for all $x$ and $t$.  Hence, we are looking for a
function of $x$ alone that satisfies $u_{xx} = 0$.  It is easy to solve this
equation by integration and we see that $u = Ax+B$ for some constants $A$ and
$B$.

Consider an insulated wire where we apply constant temperature $T_1$
at one end (say where $x=0$) and $T_2$ on the other end (at $x=L$ where $L$
is the length of the wire).  Our steady state solution is
\begin{equation*}
u(x) = \frac{T_2-T_1}{L} x + T_1 .
\end{equation*}
This solution agrees with our common sense intuition with how the heat should be
distributed in the wire.  So in one dimension, the steady state solutions
are basically just straight lines.

Things are more complicated in two or more space dimensions.  Let us
restrict to two space dimensions for simplicity.  The heat equation in two
space variables is
\begin{equation} \label{dirich:heateq}
u_t = k(u_{xx} + u_{yy}) ,
\end{equation}
or more commonly written as
$u_t = k \Delta u$ or
$u_t = k \nabla^2 u$.  Here the $\Delta$ and $\nabla^2$ symbols
mean $\frac{\partial^2}{\partial x^2} +
\frac{\partial^2}{\partial y^2}$.  We will use $\Delta$
from now on.  The reason for using such a notation is that you
can define $\Delta$ to be the right thing for any number of space
dimensions and then the heat equation is always
$u_t = k \Delta u$.  The operator $\Delta$ is called the \emph{\myindex{Laplacian}}.

OK\@, now that we have notation out of the way, let us see what does an equation
for the steady state solution look like.  We are looking for a solution to
\eqref{dirich:heateq} that does not depend on $t$, or in other words $u_t =
0$.  Hence we are looking for a
function $u(x,y)$ such that
\begin{equation*}
\mybxbg{~~
\Delta u = 
u_{xx} + u_{yy} = 0 .
~~}
\end{equation*}
This equation is called the \emph{\myindex{Laplace equation}}%
\footnote{Named after the French mathematician
\href{https://en.wikipedia.org/wiki/Laplace}{Pierre-Simon, marquis de Laplace}
(1749--1827).}, and is an example of an elliptic equation.
Solutions to the Laplace equation
are called \emph{harmonic functions\index{harmonic function}}
and have many nice properties and
applications far beyond the steady state heat problem.

Harmonic functions in two variables are no longer just linear (plane
graphs).  For example, you can check that the functions
$x^2-y^2$ and $xy$ are harmonic.  However, note that if $u_{xx}$ is positive, $u$ is concave
up in the $x$ direction, then $u_{yy}$ must be negative and $u$ must be
concave down in the $y$ direction.  A harmonic function can never
have any \myquote{hilltop} or \myquote{valley} on the graph.  This observation is
consistent with our intuitive idea of steady state heat distribution;
the hottest or coldest spot will not be inside.

Commonly the Laplace equation is part of a so-called
\emph{\myindex{Dirichlet problem}}%
\footnote{Named after the German mathematician
\href{https://en.wikipedia.org/wiki/Dirichlet}{Johann Peter Gustav Lejeune Dirichlet}
(1805--1859).}.
That
is, we have a region in the $xy$-plane and we specify certain values along
the boundaries of the region.  We then try to find a solution $u$ to the
Laplace equation defined on
this region such that $u$ agrees with the values we specified on the
boundary.

In this section we consider a rectangular region.  For simplicity
we specify boundary values to be zero at 3 of the four edges and only
specify an arbitrary function at one edge.  As we still have the
principle of superposition, we can use this simpler
solution to derive the general
solution for arbitrary boundary values by solving 4 different problems,
one for each edge, and adding those solutions together.
This setup is left as an exercise.

We wish to solve the following problem.  Let $h$ and $w$
be the height and width of our rectangle, with one corner at the origin and
lying in the first quadrant.

% FIXME: numbering does not work now since there are no hard coded numbers
% here!
%mbx <mdn>
%mbx   <mrow xml:id="dirich_eq1" number="%MBXEQNNUMBER%">
%mbx     &amp; \Delta u = 0 ,
%mbx   </mrow>
%mbx   <mrow xml:id="dirich_eq2" number="%MBXEQNNUMBER%">
%mbx     &amp; u(0,y) = 0 \quad \text{for } 0 &lt; y &lt; h,
%mbx   </mrow>
%mbx   <mrow xml:id="dirich_eq3" number="%MBXEQNNUMBER%">
%mbx     &amp; u(x,h) = 0 \quad \text{for } 0 &lt; x &lt; w,
%mbx   </mrow>
%mbx   <mrow xml:id="dirich_eq4" number="%MBXEQNNUMBER%">
%mbx     &amp; u(w,y) = 0 \quad \text{for } 0 &lt; y &lt; h,
%mbx   </mrow>
%mbx   <mrow xml:id="dirich_eq5" number="%MBXEQNNUMBER%">
%mbx     &amp; u(x,0) = f(x) \quad \text{for } 0 &lt; x &lt; w.
%mbx   </mrow>
%mbx </mdn>

\begin{center}
%mbxSTARTIGNORE
\begin{minipage}[b]{2.8in}
\vspace{\fill}
\begin{align}
& \Delta u = 0 , & &  \label{dirich:eq1} \\
& u(0,y) = 0 & & \text{for }  0 < y < h,\label{dirich:eq2} \\
& u(x,h) = 0 & & \text{for }  0 < x < w,\label{dirich:eq3} \\
& u(w,y) = 0 & & \text{for }  0 < y < h,\label{dirich:eq4} \\
& u(x,0) = f(x) & & \text{for }  0 < x < w.\label{dirich:eq5}
\end{align}
\vspace{\fill}
\end{minipage}
\qquad
%mbxENDIGNORE
\inputpdft{dirichsetup}
%mbxSTARTIGNORE
\qquad
%mbxENDIGNORE
\end{center}

The method we apply is separation of variables.  Again, we will
come up with enough building-block
solutions satisfying all the homogeneous boundary conditions
(all conditions except \eqref{dirich:eq5}).  We notice that superposition
still works for the equation and all the homogeneous conditions.
Therefore,
we can use the Fourier series for $f(x)$ to solve the
problem as before.

We try $u(x,y) = X(x)Y(y)$.  We plug $u$ into the equation to get
\begin{equation*}
X''Y + XY'' = 0 .
\end{equation*}
We put the $X$s on one side and the $Y$s on the other to get
\begin{equation*}
- \frac{X''}{X} = \frac{Y''}{Y} .
\end{equation*}
The left-hand side only depends on $x$ and the right-hand side only depends
on $y$.  Therefore, there is some constant $\lambda$ such that
$\lambda = \frac{-X''}{X} = \frac{Y''}{Y}$.
And we get two equations
\begin{align*}
& X'' + \lambda X = 0 , \\
& Y'' - \lambda Y = 0 .
\end{align*}
Furthermore, the homogeneous boundary conditions imply that
$X(0) = X(w) = 0$ and $Y(h) = 0$.  Taking the equation for $X$
we have already seen that we have a nontrivial solution if and only if
$\lambda = \lambda_n = \frac{n^2 \pi^2}{w^2}$ and the solution is
a multiple of
\begin{equation*}
X_n(x) = \sin \left( \frac{n \pi}{w} x \right) .
\end{equation*}
For these given $\lambda_n$,
the general solution for $Y$ (one for each $n$) is
\begin{equation} \label{dirich:Yngensol}
Y_n(y) = A_n \cosh \left( \frac{n \pi}{w} y \right)
+ B_n \sinh \left( \frac{n \pi}{w} y \right) .
\end{equation}
There is only one condition on $Y_n$ and hence we can pick one of $A_n$
or $B_n$
to be something convenient.
It will be useful to have $Y_n(0) = 1$, so let $A_n=1$.
Setting $Y_n(h) = 0$ and solving for $B_n$, we get
\begin{equation*}
B_n = \frac{- \cosh \left( \frac{n \pi h }{w} \right)}%
{\sinh \left( \frac{n \pi h }{w} \right)} .
\end{equation*}
After we plug the $A_n$ and $B_n$ we
into \eqref{dirich:Yngensol} and simplify by using
the identity $\sinh(\alpha-\beta) =
\sinh(\alpha) \cosh(\beta) -
\cosh(\alpha) \sinh(\beta)$, we find
\begin{equation*}
Y_n(y) =
\frac{\sinh \left( \frac{n \pi (h-y) }{w} \right)}%
{\sinh \left( \frac{n \pi h }{w} \right)} .
\end{equation*}
We define $u_n(x,y) = X_n(x)Y_n(y)$.
And note that $u_n$
satisfies \eqref{dirich:eq1}--\eqref{dirich:eq4}.
Observe
\begin{equation*}
u_n(x,0) = X_n(x)Y_n(0) = \sin \left( \frac{n \pi}{w} x \right) .
\end{equation*}
Suppose
\begin{equation*}
f(x) =
%\frac{a_0}{2} +
\sum_{n=1}^\infty
%a_n \cos \left( \frac{n \pi x }{w} \right)
%+
b_n \sin \left( \frac{n \pi x }{w} \right) .
\end{equation*}
Then we get a solution of \eqref{dirich:eq1}--\eqref{dirich:eq5} of the
following form.
\begin{equation*}
\mybxbg{
~~
u(x,y) =
\sum_{n=1}^\infty
b_n u_n(x,y)
=
\sum_{n=1}^\infty
b_n 
\sin \left( \frac{n \pi}{w} x \right)
\left( \frac{\sinh \left( \frac{n \pi (h-y) }{w} \right)}%
{\sinh \left( \frac{n \pi h }{w} \right)} \right)
.
~~
}
\avoidbreak
\end{equation*}
As $u_n$ satisfies \eqref{dirich:eq1}--\eqref{dirich:eq4} and any linear
combination (finite or infinite) of $u_n$ also satisfies 
\eqref{dirich:eq1}--\eqref{dirich:eq4}, then $u$ satisfies
\eqref{dirich:eq1}--\eqref{dirich:eq4}.
We plug in $y=0$ to see $u$
satisfies 
\eqref{dirich:eq5} as well.

\begin{example}
Take $w=h=\pi$ and let $f(x) = \pi$.  Let us compute the sine
series for the function $\pi$ (same as the series for the square wave).
For $0 < x < \pi$, we have
\begin{equation*}
f(x) =
\sum_{\substack{n=1 \\ n \text{ odd}}}^\infty
\frac{4}{n}
\sin (n x) .
\end{equation*}
The solution $u(x,y)$, see \figurevref{dirichsquareplot:fig},
to the corresponding Dirichlet problem is
given as
\begin{equation*}
u(x,y) =
\sum_{\substack{n=1 \\ n \text{ odd}}}^\infty
\frac{4}{n}
\sin (n x)
\left( \frac{\sinh \bigl( n (\pi-y) \bigr) }{\sinh (n \pi)} \right)
.
\end{equation*}

\begin{myfig}
\capstart
\diffyincludegraphics{width=5in}{width=7.5in}{dirichsquareplot}
\caption{Steady state temperature of a square plate, three sides
held at zero and one side held at $\pi$.\label{dirichsquareplot:fig}}
\end{myfig}
\end{example}

This scenario
corresponds to the steady state temperature on a square plate of width $\pi$
with 3 sides held at 0 degrees and one side held at $\pi$ degrees.
If we have arbitrary initial data on all sides, then we solve four problems,
each using one piece of nonhomogeneous data.  Then we use the principle of
superposition to add up all four solutions to have a solution to the
original problem.

A different
way to visualize solutions of the Laplace equation is to
take a wire and bend
it so that it corresponds to the graph of the
temperature above the boundary of your region.  Cut a rubber sheet in
the shape of your region---a square in our case---and stretch it
fixing the edges of the sheet to the wire.
The rubber sheet is a good approximation of the graph of the solution to
the Laplace equation with the given boundary data.

\subsection{Exercises}

\begin{exercise}
Let $R$ be the region described by $0 < x < \pi$ and $0 < y < \pi$.
Solve the problem
\begin{equation*}
\Delta u = 0, \quad u(x,0) = \sin x, \quad u(x,\pi) = 0,
\quad u(0,y) = 0, 
\quad u(\pi,y) = 0 .
\end{equation*}
\end{exercise}

\begin{exercise}
Let $R$ be the region described by $0 < x < 1$ and $0 < y < 1$.
Solve the problem
\begin{align*}
& u_{xx} + u_{yy} = 0, \\
& u(x,0) = \sin (\pi x) - \sin (2\pi x), \quad u(x,1) = 0, \\
& u(0,y) = 0, \quad u(1,y) = 0 .
\end{align*}
\end{exercise}

\begin{exercise}
Let $R$ be the region described by $0 < x < 1$ and $0 < y < 1$.
Solve the problem
\begin{align*}
& u_{xx} + u_{yy} = 0, \\
& u(x,0) = u(x,1) = u(0,y) = u(1,y) = C .
\end{align*}
for some constant $C$.  Hint: Guess, then check your intuition.
\end{exercise}

\begin{exercise} \label{dirich:diffsepexr}
Let $R$ be the region described by $0 < x < \pi$ and $0 < y < \pi$.
Solve
\begin{equation*}
\Delta u = 0,
\quad u(x,0) = 0,
\quad u(x,\pi) = \pi,
\quad u(0,y) = y,
\quad u(\pi,y) = y .
\end{equation*}
Hint: Try a solution of the form $u(x,y) = X(x) + Y(y)$ (different separation
of variables).
\end{exercise}

\begin{exercise}
Use the solution of \exerciseref{dirich:diffsepexr} to solve
\begin{equation*}
\Delta u = 0,
\quad u(x,0) = \sin x,
\quad u(x,\pi) = \pi,
\quad u(0,y) = y,
\quad u(\pi,y) = y .
\end{equation*}
Hint: Use superposition.
\end{exercise}

\begin{exercise}
Let $R$ be the region described by $0 < x < w$ and $0 < y < h$.
Solve the problem
\begin{align*}
& u_{xx} + u_{yy} = 0, \\
& u(x,0) = 0, \quad u(x,h) = f(x), \\
& u(0,y) = 0, \quad u(w,y) = 0.
\end{align*}
The solution should be in series form using the Fourier series coefficients
of $f(x)$.
\end{exercise}

\begin{exercise}
Let $R$ be the region described by $0 < x < w$ and $0 < y < h$.
Solve the problem
\begin{align*}
& u_{xx} + u_{yy} = 0, \\
& u(x,0) = 0, \quad u(x,h) = 0, \\
& u(0,y) = f(y), \quad u(w,y) = 0.
\end{align*}
The solution should be in series form using the Fourier series coefficients
of $f(y)$.
\end{exercise}

\begin{exercise}
Let $R$ be the region described by $0 < x < w$ and $0 < y < h$.
Solve the problem
\begin{align*}
& u_{xx} + u_{yy} = 0, \\
& u(x,0) = 0, \quad u(x,h) = 0, \\
& u(0,y) = 0, \quad u(w,y) = f(y).
\end{align*}
The solution should be in series form using the Fourier series coefficients
of $f(y)$.
\end{exercise}

\begin{exercise}
Let $R$ be the region described by $0 < x < 1$ and $0 < y < 1$.
Solve the problem
\begin{align*}
& u_{xx} + u_{yy} = 0, \\
& u(x,0) = \sin (9 \pi x), \quad u(x,1) = \sin (2 \pi x), \\
& u(0,y) = 0, \quad u(1,y) = 0 .
\end{align*}
Hint: Use superposition.
\end{exercise}

\begin{exercise}
Let $R$ be the region described by $0 < x < 1$ and $0 < y < 1$.
Solve the problem
\begin{align*}
& u_{xx} + u_{yy} = 0, \\
& u(x,0) = \sin (\pi x), \quad u(x,1) = \sin (\pi x), \\
& u(0,y) = \sin (\pi y), \quad u(1,y) = \sin (\pi y) .
\end{align*}
Hint: Use superposition.
\end{exercise}

\begin{exercise}[challenging]
Using only your intuition find $u(\nicefrac{1}{2},\nicefrac{1}{2})$,
for the problem
$\Delta u = 0$, where $u(0,y) = u(1,y) = 100$ for $0 < y < 1$, and
$u(x,0) = u(x,1) = 0$ for $0 < x < 1$.  Explain.
\end{exercise}

\setcounter{exercise}{100}

\begin{exercise}
Let $R$ be the region described by $0 < x < 1$ and $0 < y < 1$.
Solve the problem
\begin{equation*}
\Delta u = 0, \quad u(x,0) = \sum_{n=1}^\infty \frac{1}{n^2} \sin (n \pi x),
\quad u(x,1) = 0,
\quad u(0,y) = 0, 
\quad u(1,y) = 0 .
\end{equation*}
\end{exercise}
\exsol{%
$u(x,y) =
\sum\limits_{n=1}^\infty
\frac{1}{n^2}
\sin ( n \pi x )
\left( \frac{\sinh ( n \pi (1-y) )}{\sinh ( n \pi )} \right)
$
}

\begin{exercise}
Let $R$ be the region described by $0 < x < 1$ and $0 < y < 2$.
Solve the problem
\begin{equation*}
\Delta u = 0, \quad u(x,0) = 0.1 \sin (\pi x),
\quad u(x,2) = 0,
\quad u(0,y) = 0, 
\quad u(1,y) = 0 .
\end{equation*}
\end{exercise}
\exsol{%
$u(x,y) =
0.1
\sin ( \pi x )
\left( \frac{\sinh ( \pi (2-y)  )}%
{\sinh ( 2 \pi )} \right)$
}

%\begin{equation*}
%f(x) =
%%\frac{a_0}{2} +
%\sum_{n=1}^\infty
%%a_n \cos \left( \frac{n \pi x }{w} \right)
%%+
%b_n \sin \left( \frac{n \pi x }{w} \right) .
%\end{equation*}
%
%u(x,y) =
%\sum_{n=1}^\infty
%b_n u_n(x,y)
%=
%\sum_{n=1}^\infty
%b_n 
%\sin \left( \frac{n \pi}{w} x \right)
%\left( \frac{\sinh \left( \frac{n \pi (h-y) }{w} \right)}%
%{\sinh \left( \frac{n \pi h }{w} \right)} \right)

%%%%%%%%%%%%%%%%%%%%%%%%%%%%%%%%%%%%%%%%%%%%%%%%%%%%%%%%%%%%%%%%%%%%%%%%%%%%%%

\sectionnewpage
\section{Dirichlet problem in the circle and the Poisson kernel}
\label{dirichdisc:section}

%mbxINTROSUBSECTION

\sectionnotes{2 lectures\EPref{, \S9.7 in \cite{EP}}\BDref{,
\S10.8 in \cite{BD}}}

\subsection{Laplace in polar coordinates}

A more natural setting for the Laplace equation $\Delta u = 0$
is a circle rather than a rectangle.  On the other hand, what makes the
problem somewhat more difficult is that we need polar coordinates.

\begin{mywrapfigsimp}[5]{1.3in}{1.5in}
\diffypdfversion{\vspace*{5pt}}
\noindent
\inputpdft{polarcoords}
\diffypdfversion{\vspace*{5pt}}
\end{mywrapfigsimp}
Recall that the polar coordinates for the $(x,y)$-plane are $(r,\theta)$: 
\begin{equation*}
x = r \cos \theta , \quad y = r \sin \theta ,
\end{equation*}
where $r \geq 0$ and $-\pi < \theta \leq \pi$.  So the point $(x,y)$ is
distance $r$ from the origin at an angle $\theta$ from the positive
$x$-axis.

Now that we know our coordinates, let us give the problem we wish
to solve.  We have a circular region of radius 1, and we are interested
in the Dirichlet problem for the Laplace equation for this region.  Let
$u(r,\theta)$ denote the temperature at the point $(r,\theta)$ in polar
coordinates.

\begin{mywrapfigsimp}{2.4in}{2.7in}
\noindent
\inputpdft{dirichdiscsetup}
\end{mywrapfigsimp}
We have the problem:
\begin{equation} \label{dirichdisc:theprobeq}
\begin{aligned}
& \Delta u = 0 , & & \text{for } \; r < 1, \\
& u(1,\theta) = g(\theta), & & \text{for } \; {-\pi} < \theta \leq \pi.
\end{aligned}
\end{equation}

The first issue we face is that we do not know the Laplacian
in polar coordinates.
Normally we would find $u_{xx}$ and $u_{yy}$ in terms of
the derivatives in $r$ and $\theta$.  We would need to solve
for $r$ and $\theta$ in terms of $x$ and $y$.  In this case
it is more convenient to work in
reverse.  We compute derivatives in $r$ and $\theta$ in terms
of derivatives in $x$ and $y$ and then we solve.  The
computations are easier this way.  First
\begin{equation*}
\begin{aligned}
& x_r = \cos \theta, & &
x_\theta = - r \sin \theta, \\
& y_r = \sin \theta, & &
y_\theta = r \cos \theta.
\end{aligned}
\end{equation*}
Next by chain rule we obtain
\begin{align*}
u_r & = u_x x_r + u_y y_r = \cos(\theta) u_x + \sin(\theta) u_y ,
\\
u_{rr} & =
\cos(\theta) ( u_{xx} x_r +u_{xy} y_r )
+ \sin(\theta) ( u_{yx} x_r +u_{yy} y_r )
\\
&
=
\cos^2(\theta) u_{xx} +
2 \cos(\theta)\sin(\theta) u_{xy} +
\sin^2(\theta) u_{yy} .
\end{align*}
Similarly for the $\theta$ derivative.  Note that we have to use
the product rule for the second derivative.
\begin{align*}
u_\theta & = u_x x_\theta + u_y y_\theta =
-r\sin(\theta) u_x + r\cos(\theta) u_y ,
\\
u_{\theta\theta} & =
-r\cos(\theta) u_x
-r\sin(\theta) (u_{xx} x_\theta + u_{xy} y_\theta)
-r\sin(\theta) u_y
+
r\cos(\theta) (u_{yx} x_\theta + u_{yy} y_\theta)
%\\
%& = 
%-r\cos(\theta) u_x
%-r\sin(\theta) (u_{xx} (-r\sin(\theta)) + u_{xy} (r \cos(\theta)))
%-r\sin(\theta) u_y
%+
%r\cos(\theta) (u_{yx} (-r\sin(\theta)) + u_{yy} (r \cos (\theta)))
\\
& = 
-r\cos(\theta) u_x
-r\sin(\theta) u_y
+r^2 \sin^2(\theta) u_{xx}
-r^2 2\sin(\theta)\cos(\theta) u_{xy}
+r^2 \cos^2(\theta) u_{yy} .
\end{align*}
Let us now try to solve for $u_{xx} + u_{yy}$.  We start with
$\frac{1}{r^2} u_{\theta\theta}$ to get rid of those pesky $r^2$.
If we add $u_{rr}$
and use the fact that $\cos^2(\theta) +\sin^2(\theta) = 1$, we get
\begin{equation*}
\frac{1}{r^2} u_{\theta\theta}
+
u_{rr}
=
u_{xx} + u_{yy} - \frac{1}{r} \cos(\theta) u_x - \frac{1}{r} \sin(\theta)
u_y .
\end{equation*}
We're not quite there yet, but all we are lacking is 
$\frac{1}{r} u_r$.  Adding it we obtain the
\emph{\myindex{Laplacian in polar coordinates}}:
\begin{equation*}
\mybxbg{~~
\Delta u 
=
u_{xx} + u_{yy} =
\frac{1}{r^2} u_{\theta\theta}
+
\frac{1}{r} u_{r}
+
u_{rr} .
~~}
\end{equation*}

Notice that the Laplacian in polar coordinates no longer has constant
coefficients.
%It looks more complicated than the original, but in fact
%the computations get easier from now on.

\subsection{Series solution}

Let us separate variables as usual.  That is let us try
$u(r,\theta) = R(r)\Theta(\theta)$.  Then
\begin{equation*}
0 = \Delta u = 
\frac{1}{r^2} R \Theta''
+
\frac{1}{r} R' \Theta
+
R'' \Theta .
\end{equation*}
Let us put $R$ on one side and $\Theta$ on the other and conclude
that both sides must be constant.
\begin{align*}
\frac{1}{r^2} R \Theta''
& =
-
\left(\frac{1}{r} R' + R''\right) \Theta 
\\
\frac{\Theta''}{\Theta}
& =
-
\frac{r R' + r^2 R''}{R} = -\lambda
\end{align*}
We get two equations:
\begin{align*}
& \Theta'' + \lambda \Theta = 0 ,
\\
& r^2 R'' + r R' -\lambda R = 0.
\end{align*}
Let us first focus on $\Theta$.  We know that $u(r,\theta)$ ought to be
$2\pi$-periodic in $\theta$, that is,
$u(r,\theta) = u(r,\theta+2\pi)$.  Therefore, the solution to
$\Theta'' + \lambda \Theta = 0$ must be $2\pi$-periodic.
We have seen such a problem in \exampleref{bvp-periodic:example}.
We conclude
that
%$\lambda = 0,1,4,9,\ldots$.  That is,
$\lambda = n^2$ for a
nonnegative integer $n=0,1,2,3,\ldots$.  The equation becomes
$\Theta'' + n^2 \Theta = 0$.  When $n=0$ the equation is just
$\Theta'' = 0$, so we have the general solution $A \theta + B$.  As
$\Theta$ is periodic,
$A=0$.
For convenience we write this solution as
\begin{equation*}
\Theta_0 = \frac{a_0}{2}
\end{equation*}
for some constant $a_0$.  For positive $n$,
the 
solution to
$\Theta'' + n^2 \Theta = 0$ is
\begin{equation*}
\Theta_n = a_n \cos(n\theta) + b_n \sin(n\theta) ,
\end{equation*}
for some constants $a_n$ and $b_n$.

Next, we consider the equation for $R$,
\begin{equation*}
r^2 R'' + r R' - n^2 R = 0.
\end{equation*}
This equation appeared in exercises before---we
solved it in \exerciseref{sol:eulerex}
and \exercisevref{sol:eulerexln}.  The idea is to try a solution
$r^s$ and if that does not give us two solutions, also try a solution of the form
$r^s \ln r$.  Let us name the solution for $R_n$.  When $n=0$ we obtain
\begin{equation*}
R_0 = A r^0 + B r^0 \ln r = A + B \ln r ,
\end{equation*}
and if $n > 0$, we get
\begin{equation*}
R_n = A r^n + B r^{-n} .
\end{equation*}
The function $u(r,\theta)$ must be finite at the origin, that is, when $r=0$.
So $B=0$ in both
cases.  Set $A=1$ in both cases as well; the constants in $\Theta_n$
will pick up the slack so nothing is lost.  Let
\begin{equation*}
R_0 = 1 , \qquad \text{and} \qquad
R_n = r^n .
\end{equation*}
Hence our building block solutions are
\begin{align*}
& u_0(r,\theta) = \frac{a_0}{2} ,
& u_n(r,\theta) = a_n r^n \cos(n \theta) + b_n r^n \sin(n \theta) .
\end{align*}
Putting everything together our solution is:
\begin{equation*}
\mybxbg{~~
u(r,\theta)
=
\frac{a_0}{2} +
\sum_{n=1}^\infty
a_n r^n \cos(n \theta) + b_n r^n \sin(n \theta) .
~~}
\end{equation*}

We look at the boundary condition in \eqref{dirichdisc:theprobeq},
\begin{equation*}
g(\theta) = u(1,\theta)
=
\frac{a_0}{2} +
\sum_{n=1}^\infty
a_n \cos(n \theta) + b_n \sin(n \theta) .
\end{equation*}
Therefore, to solve \eqref{dirichdisc:theprobeq}
we expand $g(\theta)$, which is 
a $2\pi$-periodic function, as a Fourier series, and then 
multiply the $n^{\text{th}}$ term by $r^n$.  To
find the $a_n$ and the $b_n$ we compute
\begin{equation*}
a_n =
\frac{1}{\pi} \int_{-\pi}^\pi g(\theta) \cos (n\theta) \, d\theta , \qquad
\text{and} \qquad
b_n =
\frac{1}{\pi} \int_{-\pi}^\pi g(\theta) \sin (n\theta) \, d\theta.
\end{equation*}

\begin{example}
Suppose we wish to solve
\begin{align*}
& \Delta u = 0 , \qquad 0 \leq r < 1, \quad -\pi < \theta \leq \pi,\\
& u(1,\theta) = \cos(10\,\theta), \qquad -\pi < \theta \leq \pi.
\end{align*}

The solution is
\begin{equation*}
u(r,\theta) = r^{10} \cos(10\,\theta) .
\end{equation*}

See the plot in \figurevref{dirichdisc:tenspeedfig}.
The thing to notice in this example is that the effect of a high frequency
is mostly felt at the boundary.  In the middle of the disc, the solution
is very close to zero.  That is because $r^{10}$ is rather small when $r$
is close to 0.
\begin{myfig}
\capstart
\diffyincludegraphics{width=5in}{width=7.5in}{dirichdisc-tenspeed}
\caption{The solution of the Dirichlet problem in the disc with
$\cos(10\,\theta)$ as boundary data.\label{dirichdisc:tenspeedfig}}
\end{myfig}
\end{example}

\begin{example}
Let us solve a more difficult problem.  Consider a long
rod with circular cross section of radius 1.  Suppose we wish to solve the
steady state heat problem in the rod.
If the rod is long enough, we simply need to solve
the Laplace equation in two dimensions.  Let us put the center of the rod at
the origin and we have exactly the region we are currently
studying---a circle of radius 1.  For the boundary conditions, suppose in
Cartesian coordinates $x$ and
$y$, the temperature on the boundary is 0 when $y < 0$, and it is $2y$ when $y > 0$.

Let us set the problem up.
As $y = r\sin(\theta)$, then on the
circle of radius 1, that is, where $r=1$, we have $2y = 2\sin(\theta)$.  So
%The problem becomes
\begin{align*}
& \Delta u = 0 , \qquad 0 \leq r < 1, \quad -\pi < \theta \leq \pi,\\
& u(1,\theta) = 
\begin{cases}
2\sin(\theta) & \text{if } \; \phantom{-}0 \leq \theta \leq \pi, \\
0 & \text{if } \; {-\pi} < \theta < 0.
\end{cases}
\end{align*}

We must now compute the Fourier series for the boundary
condition.  By now the reader has plentiful experience in computing
Fourier series and so we simply state that 
\begin{equation*}
u(1,\theta) = 
\frac{2}{\pi}
+
\sin(\theta)
+
\sum_{n=1}^\infty \frac{-4}{\pi(4n^2-1)} \cos(2n\theta) .
\end{equation*}

\begin{exercise}
Compute the series for $u(1,\theta)$ and verify that it really is what
we have just claimed.  Hint: Be careful, make sure not to divide by zero.
\end{exercise}

We now simply write the solution (see \figurevref{dirichdisc:zero2yfig}) by multiplying by $r^n$ in the right places.
\begin{equation*}
u(r,\theta) = 
\frac{2}{\pi}
+
r\sin(\theta)
+
\sum_{n=1}^\infty \frac{-4r^{2n}}{\pi(4n^2-1)} \cos(2n\theta) .
\end{equation*}
\begin{myfig}
\capstart
\diffyincludegraphics{width=5in}{width=7.5in}{dirichdisc-zero2y}
\caption{The solution of the Dirichlet problem with
boundary data 0 for $y < 0$ and $2y$ for $y > 0$.\label{dirichdisc:zero2yfig}}
\end{myfig}
%The plot of the solution is given in \figurevref{dirichdisc:zero2yfig}.
\end{example}

%Note that the formulas are not difficult to generalize
%for a circle of
%any radius and this generalization is left to the reader in the exercises.

\subsection{Poisson kernel}

There is another way to solve the Dirichlet problem with the help of an
integral kernel.  That is, we will find a function $P(r,\theta,\alpha)$
called the \emph{\myindex{Poisson kernel}}\footnote{%
Named for the French mathematician
\href{https://en.wikipedia.org/wiki/Sim\%C3\%A9on_Denis_Poisson}{Sim\'eon
Denis Poisson}
(1781--1840).} such that
\begin{equation*}
u(r,\theta) = 
\frac{1}{2\pi}
\int_{-\pi}^{\pi}
P(r,\theta,\alpha) \, g(\alpha) \,d\alpha .
\end{equation*}
While the integral will generally not be solvable analytically, it can
be evaluated numerically.   In fact, unless the boundary data is given
as a Fourier series already, it may be much easier to numerically
evaluate this formula as there is only one integral to evaluate.

The formula also has theoretical applications.
For instance, as $P(r,\theta,\alpha)$ 
will have infinitely many derivatives, then
via differentiating under the integral we find
that the solution $u(r,\theta)$ has infinitely many derivatives, at least
when inside the circle, $r < 1$.  By \myquote{having infinitely many
derivatives,} what you
should think of is that $u(r,\theta)$ has \myquote{no corners} and all of its
partial derivatives of all orders exist and also have \myquote{no corners.}

%A similar integral formula and an integral kernel
%always exists no matter what the shape of the region is,
%however it is much more difficult to compute when the region is
%very irregular.  For a circle, the formula we will obtain in the
%end is quite simple and has a nice geometric interpretation.

We will compute
the formula for $P(r,\theta,\alpha)$ from the series
solution, and this idea can be applied anytime you have a convenient
series solution where the coefficients are obtained via integration.
Hence you can apply this reasoning to obtain such integral kernels
for other equations, such as the heat equation.
The computation is long and \myindex{tedious}, but not overly difficult.
Since the ideas are often applied in similar contexts, it is good to
understand how this computation works.

What we do is start with the series solution and replace the coefficients
with the integrals that compute them.  Then we try to write everything as
a single integral.  We must use a different dummy variable for the
integration and hence we use $\alpha$ instead of $\theta$.
\begin{equation*}
\begin{split}
u(r,\theta)
& =
\frac{a_0}{2} +
\sum_{n=1}^\infty
a_n r^n \cos(n \theta) + b_n r^n \sin(n \theta)
\\
& =
\underbrace{
 \left(
  \frac{1}{2\pi} \int_{-\pi}^\pi g(\alpha) \, d\alpha
 \right)
}_{\frac{a_0}{2}}
+
%\\
%& ~~~~~~~~ +
\sum_{n=1}^\infty
\underbrace{
 \left(
  \frac{1}{\pi} \int_{-\pi}^\pi g(\alpha) \cos (n\alpha) \, d\alpha
 \right)
}_{a_n}
r^n \cos(n \theta) +
\\
& ~~~~~~~~ +
\underbrace{
 \left(
  \frac{1}{\pi} \int_{-\pi}^\pi g(\alpha) \sin (n\alpha) \, d\alpha
 \right)
}_{b_n}
r^n \sin(n \theta)
\\
& =
\frac{1}{2\pi}
\int_{-\pi}^\pi
\left(  g(\alpha)
+
2
\sum_{n=1}^\infty
g(\alpha) \cos (n\alpha) 
\, r^n \cos(n \theta) +
g(\alpha) \sin (n\alpha)
\, r^n \sin(n \theta)
\right) \,d\alpha
\\
& =
\frac{1}{2\pi}
\int_{-\pi}^\pi
\underbrace{
 \left( 1
 +
 2
 \sum_{n=1}^\infty
 r^n 
 \bigl(
 \cos (n\alpha) 
 \cos(n \theta) +
 \sin (n\alpha)
 \sin(n \theta) \bigr)
 \right)
}_{P(r,\theta,\alpha)}
g(\alpha) \,d\alpha
%\\
%& =
%\frac{1}{2\pi}
%\int_{-\pi}^\pi
%\left( 1
%+
%2
%\sum_{n=1}^\infty
%r^n 
%\cos \bigl(n(\theta-\alpha)\bigr)
%\right) g(\alpha) \,d\alpha .
\end{split}
\end{equation*}
OK\@, so we have what we wanted, the expression in the parentheses is the
Poisson kernel, $P(r,\theta,\alpha)$.  However, we can do a lot better.  It is still given as a
series, and we would really like to have a nice simple expression for it. 
We must work a little harder.  The trick is to rewrite everything in terms of
complex exponentials.  Let us work
just on the kernel.
\begin{equation*}
\begin{split}
P(r,\theta,\alpha)
& =
1
+
2
\sum_{n=1}^\infty
r^n 
\bigl(
\cos (n\alpha) 
\cos(n \theta) +
\sin (n\alpha)
\sin(n \theta) \bigr)
\\
& =
1
+
2
\sum_{n=1}^\infty
r^n 
\cos \bigl(n(\theta-\alpha)\bigr)
\\
& =
1
+
\sum_{n=1}^\infty
r^n 
\bigl(
e^{in(\theta-\alpha)} +
e^{-in(\theta-\alpha)} \bigr)
\\
& =
1
+
\sum_{n=1}^\infty
{\bigl(
re^{i(\theta-\alpha)}\bigr)}^{n}
+
\sum_{n=1}^\infty
{\bigl(
re^{-i(\theta-\alpha)}\bigr)}^{n} .
\end{split}
\end{equation*}
In the expression above, we recognize the
\emph{\myindex{geometric series}}.
Recall from calculus that if $z$ is a complex number where $\lvert z \rvert < 1$, then
\begin{equation*}
\sum_{n=1}^\infty z^n = \frac{z}{1-z} .
\end{equation*}
Note that $n$ starts at $1$ and that is why we have the $z$ in the numerator.
It is the standard geometric series multiplied by $z$.
We can use $z = re^{i(\theta-\alpha)}$, as
lo and behold $\lvert re^{i(\theta-\alpha)} \rvert = r < 1$.
Let us
continue with the computation.
\begin{equation*}
\begin{split}
P(r,\theta,\alpha)
& =
1
+
\sum_{n=1}^\infty
{\bigl(
re^{i(\theta-\alpha)}\bigr)}^{n}
+
\sum_{n=1}^\infty
{\bigl(
re^{-i(\theta-\alpha)}\bigr)}^{n}
\\
& =
1
+
\frac{re^{i(\theta-\alpha)}}{1-re^{i(\theta-\alpha)}}
+
\frac{re^{-i(\theta-\alpha)}}{1-re^{-i(\theta-\alpha)}}
\\
& = 
\frac{
\bigl(1-re^{i(\theta-\alpha)}\bigr)\bigl(1-re^{-i(\theta-\alpha)}\bigr)
+
\bigl(1-re^{-i(\theta-\alpha)}\bigr)re^{i(\theta-\alpha)} +
\bigl(1-re^{i(\theta-\alpha)}\bigr)re^{-i(\theta-\alpha)}}
{\bigl(1-re^{i(\theta-\alpha)}\bigr)\bigl(1-re^{-i(\theta-\alpha)}\bigr)}
\\
& = 
\frac{1 -r^2}{1 - re^{i(\theta-\alpha)} - re^{-i(\theta-\alpha)} +r^2}
\\
& = 
\frac{1 -r^2}{1 - 2r\cos(\theta-\alpha) +r^2} .
\end{split}
\end{equation*}
That's a formula we can live with.  The
solution to the Dirichlet problem using the Poisson kernel is
\begin{equation*}
\mybxbg{~~
u(r,\theta) = 
\frac{1}{2\pi} \int_{-\pi}^{\pi}
\frac{1 -r^2}{1 - 2r\cos(\theta-\alpha) +r^2} g(\alpha) \, d\alpha .
~~}
\end{equation*}
Sometimes the formula for the Poisson kernel is
given together with the constant $\frac{1}{2\pi}$, in which case we should
of course not leave it in front of the integral.
Also, often the limits
of the integral are given as 0 to $2\pi$; everything inside is
$2\pi$-periodic in $\alpha$, so this does not change the integral.

%12 is the number of lines, must be adjusted
\begin{mywrapfigsimp}[12]{2.1in}{2.4in}
\diffypdfversion{\vspace*{5pt}}
\noindent
\inputpdft{poisson}
\end{mywrapfigsimp}
Let us not leave the Poisson kernel without explaining its geometric
meaning.  Let $s$ be the distance from $(r,\theta)$ to
$(1,\alpha)$.
You may recall from calculus that
this distance $s$ in polar coordinates is given precisely by the square root
of $1 - 2r\cos(\theta-\alpha) +r^2$.  That is, the Poisson kernel is really
the formula
\begin{equation*}
\frac{1-r^2}{s^2} .
\end{equation*}
%See the figure on the right.

One final note we make about the formula is that it is really
a weighted average of the boundary values.
First let us look
at what happens at the origin,
that is when $r=0$. %(and then $\theta$ can be anything at all)
\begin{equation*}
\begin{split}
u(0,0) &= 
\frac{1}{2\pi} \int_{-\pi}^{\pi}
\frac{1 -0^2}{1 - 2(0)\cos(0-\alpha) +0^2} g(\alpha) \, d\alpha
\\
& =
\frac{1}{2\pi} \int_{-\pi}^{\pi}
g(\alpha) \, d\alpha .
\end{split}
\end{equation*}
So $u(0,0)$ is precisely the average value of $g(\theta)$ and
therefore the average value of $u$ on the boundary.  This is
a general feature of harmonic functions, the value at some point $p$
is equal to the average of the values on a circle centered at $p$.

What the formula says is that the value of the solution at any point in the
circle is a weighted average of the boundary data $g(\theta)$.  The kernel
is bigger when $(1,\alpha)$ is closer to $(r,\theta)$.  Therefore when
computing $u(r,\theta)$, we
give more weight to the values $g(\alpha)$ when $(1,\alpha)$ is closer to $(r,\theta)$ and less
weight to the values $g(\alpha)$ when $(1,\alpha)$ far from $(r,\theta)$.

\subsection{Exercises}

\begin{exercise}
Using series solve
$\Delta u = 0$, $u(1,\theta) = \lvert \theta \rvert$, for $-\pi < \theta
\leq \pi$.
\end{exercise}

\begin{exercise}
Using series solve $\Delta u = 0$, $u(1,\theta) = g(\theta)$ for the
following data.  Hint: trig identities.
\begin{tasks}(2)
\task
$g(\theta) = 
\nicefrac{1}{2} + 3\sin(\theta) + \cos(3\theta)$
\task
$g(\theta) = 
3\cos(3\theta) + 3\sin(3\theta) + \sin(9\theta)$
\task
$g(\theta) = 2 \cos(\theta+1)$
\task
$g(\theta) = \sin^2(\theta)$
\end{tasks}
\end{exercise}

\begin{exercise}
Using the Poisson kernel, give the solution to
$\Delta u = 0$, where $u(1,\theta)$ is zero for $\theta$ outside
the interval $[-\nicefrac{\pi}{4},\nicefrac{\pi}{4}]$ and 
$u(1,\theta)$ is 1 for $\theta$ on the interval
$[-\nicefrac{\pi}{4},\nicefrac{\pi}{4}]$.
\end{exercise}

\begin{exercise}
\pagebreak[2]
\leavevmode
\begin{tasks}
\task Draw a graph for the Poisson kernel as a function of $\alpha$
when $r=\nicefrac{1}{2}$ and $\theta = 0$.
\task Describe what happens to the graph when you make $r$ bigger (as it
approaches 1).
\task Knowing that the solution $u(r,\theta)$ is the weighted average
of $g(\theta)$ with Poisson kernel as the weight, explain what your answer
to part b) means.
\end{tasks}
\end{exercise}

\begin{exercise} \label{exercise:dirichproblemxy}
Let $g(\theta)$ be the function $xy = \cos \theta \sin
\theta$ on the boundary.  Use the series solution to find a solution
to the Dirichlet problem $\Delta u = 0$, $u(1,\theta) = g(\theta)$.  Now
convert the solution to Cartesian coordinates $x$ and $y$.  Is this
solution surprising?  Hint: use your trig identities.
\end{exercise}

\begin{exercise}
Carry out the computation we needed in the separation of variables and solve
$r^2 R'' + r R' - n^2 R = 0$, for $n=0,1,2,3,\ldots$.
\end{exercise}

\begin{exercise}[challenging]
Derive the series solution to the Dirichlet problem if the region is a
circle of radius $\rho$ rather
than 1.
That is, solve $\Delta u = 0$, $u(\rho,\theta) = g(\theta)$.
\end{exercise}

\begin{exercise}[challenging]
\leavevmode
\begin{tasks}
\task
Find the solution for
$\Delta u = 0$, $u(1,\theta) = x^2y^3 + 5 x^2$.  Write the answer in Cartesian coordinates.
\task
Now solve
$\Delta u = 0$, $u(1,\theta) = x^k y^\ell$.
Write the solution in Cartesian coordinates.
\task
Suppose you have a polynomial $P(x,y) = \sum_{j=0}^m \sum_{k=0}^n c_{j,k}
x^j y^k$, solve $\Delta u = 0$, $u(1,\theta) = P(x,y)$ (that is, write down
the formula for the answer).  Write the answer
in Cartesian coordinates.
\end{tasks}
Notice the answer is again a polynomial in $x$ and $y$.
See also \exerciseref{exercise:dirichproblemxy}.
\end{exercise}

\setcounter{exercise}{100}

\begin{exercise}
Using series solve
$\Delta u = 0$, $u(1,\theta) = 1+ \sum\limits_{n=1}^\infty \frac{1}{n^2}\sin(n\theta)$.
\end{exercise}
\exsol{%
$u = 1+ \sum\limits_{n=1}^\infty \frac{1}{n^2}r^n\sin(n\theta)$
}

\begin{exercise}
Using the series solution find the solution to
$\Delta u = 0$, $u(1,\theta) = 1- \cos(\theta)$.  Express the solution
in Cartesian coordinates (that is, using $x$ and $y$).
\end{exercise}
\exsol{%
$u = 1-x$
}

\begin{exercise}
\leavevmode
\begin{tasks}
\task
Try and guess a solution to $\Delta u = -1$, $u(1,\theta) = 0$.
Hint: try a solution that only depends on $r$.  Also first, don't worry
about the boundary condition.
\task
Now solve $\Delta u = -1$, $u(1,\theta) = \sin(2\theta)$ using
superposition.
\end{tasks}
\end{exercise}
\exsol{%
a) $u = \frac{-1}{4} r^2 + \frac{1}{4}$
b) $u = \frac{-1}{4} r^2 + \frac{1}{4} + r^2 \sin(2\theta)$
}

\begin{exercise}[challenging]
Derive the Poisson kernel solution
if the region is a circle of radius $\rho$ rather
than 1.  That is, solve $\Delta u = 0$, $u(\rho,\theta) = g(\theta)$.
\end{exercise}
\exsol{%
$\displaystyle
u(r,\theta) = 
\frac{1}{2\pi} \int_{-\pi}^{\pi}
\frac{\rho^2 -r^2}{\rho^2 - 2r\rho\cos(\theta-\alpha) +r^2} g(\alpha) \, d\alpha$
}
