	\begin{objectives}
	In this tutorial you will practice using the determinant to answer geometric and algebraic questions.

	These problems relate to the following course learning objectives:
	\textit{Translate between algebraic and geometric viewpoints to solve problems}, and
		\textit{use determinants to solve problems}.
	\end{objectives}

	\subsection*{Problems}

	\begin{enumerate}
		\item Let $\vec a=\mat{1\\1\\0}$, $\vec b=\mat{2\\1\\1}$, $\vec c=\mat{-1\\3\\1}$,
			$\vec d=\mat{1\\1\\1}$, and $\vec e=\mat{0\\1\\0}$.

			A gem expert has found several crystals in the shape of parallelepipeds (the three-dimensional
			analogue of a parallelogram). Carefully measuring, she discovers:
			crystal $X$ has edges described by $\vec a$, $\vec b$, $\vec c$; crystal $Y$ has edges described by
			$\vec c$, $\vec d$, $\vec e$;
			and crystal $Z$ has edges described by $\vec a$, $\vec d$, $\vec e$.
			\begin{enumerate}
				\item Which crystal has the largest volume?
				\item Which crystal is the ``pointiest''? Justify your conclusion.
			\end{enumerate}
		\item Use row reduction to solve the system
			\[
            \left\{
            \begin{array}{ccccc}
              ax & + & by & = & 1 \\
              cx & + & dy & = & 0
            \end{array}
            \right.				
			\]
			recording all your steps. If $\det\left(\mat{a&b\\c&d}\right)=0$, which row-reduction
			step fails? Why?
		\item Let $\vec a$ and $\vec d$ be as in problem 1, and let $\vec f$ be a unit vector.
			What is the largest possible volume of the parallelepiped with edges $\vec a$, $\vec d$, $\vec f$?
		\item For $\vec u,\vec v\in \R^3$, the \emph{cross product} of $\vec u$ and $\vec v$, written $\vec u\times \vec v$,
			is a vector orthogonal to $\vec u$ and $\vec v$ whose length is the area of the parallelogram
			with sides $\vec u$ and $\vec v$ and such that $\det([\vec u|\vec v|\vec u\times \vec v]) \geq  0$.

			Use the definition\footnote{If you know a formula for the cross product, feel free to
			use it to check your work, but the definition given above is not stated in terms
			of a formula.} of the cross product to
			find the cross product of the vectors $\vec u=\mat{1\\2\\3}$ and $\vec v=\mat{1\\1\\-1}$. \emph{Hint:
			you already know how to use dot products to find the angle between vectors.}
	\end{enumerate}

