\subsection*{Learning Objectives}
	Students need to be able to\ldots
	\begin{itemize}
		\item Identify whether a transformation is linear
		\item Write a coherent argument showing why a function is or isn't linear
		\item Come up with examples of linear and non-linear transformations
	\end{itemize}

\vspace*{-.5cm}
\subsection*{Context}
	Students have seen linear transformations and matrix transformations in class as well
		as range, null space, row space, column space, and rank. The \emph{rank} of a transformation
		is defined as the dimension of the range---it is a theorem, not the definition,
		that that this number is the same as the number of pivots in the reduced row
		echelon form of a matrix for the transformation.
	
	Students should have thought a lot about linear transformations as geometric things.
		In this tutorial we treat them more as formal and algebraic objects. There are
		several valid ways to define a linear transformation and different sections
		may have used different, but equivalent definitions. The most common ones are:

	\begin{itemize}
		\item A function $T:V\to W$ is \emph{linear} if it satisfies (a) $T(\vec u+\vec v)=T\vec u+T\vec v$
			and (b) $T(\alpha\vec u)=\alpha T\vec u$ for all $\vec u,\vec v\in V$ and all scalars $\alpha$.
		\item A function $T:V\to W$ is \emph{linear} if it satisfies $T(\alpha\vec u+\beta\vec v)=\alpha T\vec u+
			\beta T\vec v$ for all $\vec u,\vec v\in V$ and scalars $\alpha,\beta$.
	\end{itemize}

\vspace*{-.5cm}
\subsection*{What to Do}
	Introduce the learning objectives for the day's tutorial. Explain that linear transformations
		are the second pillar of linear algebra (the first being change of basis).
		Explain that today we are focusing on formal arguments to justify claims
		about linearity, and so we will rely heavily on definitions.

	Have students work in small groups. Again, this tutorial starts with a definition. Students from
		different sections may have different, but equivalent, definitions. Make sure everyone
		knows that there are multiple correct ways to define linearity.

	The meat of this tutorial is \#2. The point of \#2 is to have them practice writing formal proofs.
		Many will be able to come up with judgements, or even counterexamples, but not be able to write
		a proof. When the class gets stuck, have a discussion about what needs to be shown
		to satisfy the definition of linearity. You can provide them with a ``proof template'', since
		all linearity proofs follow the form:
		\begin{quote}
			Let $\vec u,\vec v\in$domain and let $\alpha$ be a scalar. Then $T(\vec u+\vec v)=\cdots$application
			of definition of $T$$\cdots=T\vec u+T\vec v$, and so on.
		\end{quote}

	\#2 will likely take the whole tutorial. Don't let groups move on until they \emph{have actually written down}
		proofs for each of the transformations. They think they have it in their head, but they don't!

		As groups come up with proofs, have multiple groups write their proofs on the board (for example,
		two proofs of $\mathcal A$), and have a class discussion about whether the proofs are correct and
		whether they could be improved. Remember: you're the ultimate math authority in the room. Have a
		friendly discussion and allow everyone to share their opinions, but be unambiguous when wrapping up and discussing
		what is ``correct''. In particular, don't say, ``it would be better like this'' if it was wrong initially. Instead,
		say, ``As written, it is not correct, but you could fix it by writing\ldots''.
	
	A good wrap-up for this tutorial is a class discussion/comparison of proofs for \#2. If everyone finishes \#2 (and
	did a good job on it, pick a suitable problem, as usual, and present a solution.

\subsection*{Notes}
	\begin{itemize}
		\item Students from different sections may have different definitions of linearity.
		\item Transformations from $\R^n\to \R$ and piecewise functions confuse students, so
			expect $\mathcal B$ to confuse them.
		\item If they get stuck on \#3, have them write down the definition of rank. The problem
			will be easier after that.
		\item The solutions for \#5 list trivial examples. The students find trivial examples
			harder than non-trivial ones. If you discuss this problem with them, pick non-trivial examples.
	\end{itemize}