		\begin{enumerate}
			\item $f:\R^n\to\R^n$ is invertible if there exists a function $g:\R^n\to\R^n$ so
				that $f\circ g=g\circ f=$id, the identity function.
				The matrix $A$ is invertible if there exists a matrix $B$ so that
				$AB=I$ and $BA=I$.
			\item \begin{enumerate}
					\item Totally incorrect. If $\vec b$ is a column
						vector $\vec b A^{-1}$ won't be defined.
						It should be $\vec x=A^{-1}\vec b$.
					\item Totally correct.
					\item Totally incorrect. There is no division operation
						for matrices. Instead, you must multiply both sides by $A^{-1}$.
					\item Mostly incorrect. The notation $\tfrac{1}{A}$ is not defined. We
						use the notation $A^{-1}$ for the matrix such that $A^{-1}A=AA^{-1}=I$.
					\item Totally incorrect. You cannot divide by a vector.
			\end{enumerate}
			\item \begin{itemize}
					\item[$R$] This transformation is invertible and its inverse is
						clockwise rotation by $30^{\circ}$.
					\item[$D$] This transformation is invertible and its inverse is
						itself.
					\item[$P$] This transformation is not invertible. Since
						$P(\vec 0) = P\left(\mat{-4\\1}\right)=\vec 0$, there
						is no way to ``undo'' $P$. Formally, $P$ is not one-to-one,
						so it cannot be invertible.
					\item[$S$] This transformation is invertible and its inverse is
						halving the length of each vector.
			\end{itemize}
			\item \begin{enumerate}
				\item $R,D,S$ are rank 2 and $P$ is rank 1.
				\item Yes. If a transformation from $\R^n\to\R^n$ is invertible
					its rank must be $n$. We can argue as follows:

					Suppose $T:\R^n\to\R^n$ has rank $< n$. Then the range of $T$ cannot
					be all of $\R^n$. Thus, there cannot exist a transformation $S$ so that
					$T\circ S=$id, where id is the identity function on all of $\R^n$.
				\item
					\[
						[R]_{\mathcal E} = \mat{\sqrt{3}/2&-1/2\\1/2&\sqrt{3}/2}\qquad
						[D]_{\mathcal E} = \tfrac{1}{17}\mat{-15&8\\8&15}
						\]\[
						[P]_{\mathcal E} = \tfrac{1}{17}\mat{1&4\\4&16}\qquad
						[S]_{\mathcal E} = \mat{2&0\\0&2}
					\]
					The matrix for a transformation is invertible if and only if the
					transformation is invertible. This is not affected by the choice of basis.
			\end{enumerate}
		\end{enumerate}