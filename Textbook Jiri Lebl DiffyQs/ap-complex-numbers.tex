\chapter{Complex numbers and Euler's formula}\label{complex:appendix}

A polynomial may have complex roots.  The
equation $r^2 + 1 = 0$ has no real roots, but it does have two complex roots.
Here we review some properties of complex numbers\index{complex number}.

Complex numbers may seem a strange concept, especially because of the
terminology.  There is nothing imaginary or really complicated about complex
numbers.
A complex number is simply a pair of real numbers, $(a,b)$.  
Think of a complex number as a point in the plane.  We add complex numbers
in the straightforward way: $(a,b)+(c,d)=(a+c,b+d)$.  We define
multiplication\index{multiplication of complex numbers} by
\begin{equation*}
(a,b) \times (c,d) \overset{\text{def}}{=} (ac-bd,ad+bc) .
\end{equation*}
It turns out that with this multiplication rule, all the standard properties
of arithmetic hold.  Further, and most importantly $(0,1) \times (0,1) =
(-1,0)$.

Generally we write $(a,b)$ as $a+ib$, and we treat $i$ as if it were an
unknown.  When $b$ is zero, then $(a,0)$ is just the number $a$.
We do arithmetic with complex numbers just as we would
with polynomials.
The property we just mentioned becomes $i^2 = -1$.
So whenever we see $i^2$, we replace it by $-1$.
For example,
\begin{equation*}
(2+3i)(4i) - 5i = 
(2\times 4)i + (3 \times 4) i^2 - 5i
=
8i + 12 (-1) - 5i
=
-12 + 3i .
\end{equation*}

The numbers
$i$ and $-i$ are the two roots of $r^2 + 1 = 0$.
Some engineers use the letter $j$ instead of $i$ for the square
root of $-1$.  We use the mathematicians' convention and use $i$.

\begin{exercise}
Make sure you understand (that you can justify)
the following identities:
\begin{tasks}(2)
\task $i^2 = -1$, $i^3 = -i$, $i^4 = 1$,
\task $\dfrac{1}{i} = -i$,
\task $(3-7i)(-2-9i) = \cdots = -69-13i$,
\task $(3-2i)(3+2i) = 3^2 - {(2i)}^2 = 3^2 + 2^2 = 13$,
\task $\frac{1}{3-2i} = \frac{1}{3-2i} \frac{3+2i}{3+2i} = \frac{3+2i}{13}
= \frac{3}{13}+\frac{2}{13}i$.
\end{tasks}
\end{exercise}

\pagebreak[2]
We also define the exponential $e^{a+ib}$ of a complex number.  We do
this by writing down the Taylor series and plugging in the complex
number.  Because most properties of the exponential can be proved by looking
at the Taylor series, these
properties still hold for the complex
exponential.  For example the very important property: $e^{x+y} = e^x e^y$.  This means that
$e^{a+ib} = e^a e^{ib}$.  Hence if we can compute $e^{ib}$, we can
compute $e^{a+ib}$.  For $e^{ib}$ we use the so-called
\emph{\myindex{Euler's formula}}.

\begin{theorem}[Euler's formula] \label{eulersformula}
\begin{equation*}
\mybxbg{~~
e^{i \theta} = \cos \theta + i \sin \theta
\qquad \text{ and } \qquad
e^{- i \theta} = \cos \theta - i \sin \theta .
~~}
\end{equation*}
\end{theorem}

In other words, $e^{a+ib} = e^a \bigl( \cos(b) + i \sin(b) \bigr) = e^a \cos(b) + i e^a \sin(b)$.

\begin{exercise}
Using Euler's formula, check the identities:
\begin{equation*}
\cos \theta = \frac{e^{i \theta} + e^{-i \theta}}{2}
\qquad \text{and} \qquad
\sin \theta = \frac{e^{i \theta} - e^{-i \theta}}{2i}.
\end{equation*}
\end{exercise}

\begin{exercise}
Double angle identities:
Start with $e^{i(2\theta)} = {\bigl(e^{i \theta} \bigr)}^2$.  Use Euler on
each side and deduce:
\begin{equation*}
\cos (2\theta) = \cos^2 \theta - \sin^2 \theta
\qquad \text{and} \qquad
\sin (2\theta) = 2 \sin \theta \cos \theta .
\end{equation*}
\end{exercise}

For a complex number $a+ib$ we call
$a$ the \emph{\myindex{real part}} and $b$ the \emph{\myindex{imaginary part}} of the number.
Often the following notation is used,
\begin{equation*}
\operatorname{Re}(a+ib) = a
\qquad \text{and} \qquad
\operatorname{Im}(a+ib) = b.
\end{equation*}
