\documentclass[10pt,aspectratio=169]{beamer}

% All the boilerplate is in deslides.sty
\usepackage{deslides}

\author{Ji\v{r}\'i Lebl}

\institute[OSU]{%
Oklahoma State University%
%Departemento pri Matematiko de Oklahoma {\^S}tata Universitato%
}

\title{13. Second order linear ODEs\\(Notes on Diffy Qs, 2.1)}

\date{}

\begin{document}

\begin{frame}
\titlepage

%\bigskip

\begin{center}
The textbook: \url{https://www.jirka.org/diffyqs/}
\end{center}
\end{frame}

\begin{frame}
Second order linear differential equation:
\[
A(x) y'' + B(x)y' + C(x)y = F(x) .
\]
\pause
Or
\[
y'' + p(x)y' + q(x)y = f(x) .
\]
\pause
\emph{Linear} means that the 
equation contains no powers nor
functions of $y$, $y'$, and $y''$.
\medskip
\pause

If $f(x)=0$, then the equation is \emph{homogeneous}:
\[
y'' + p(x)y' + q(x)y = 0 .
\]
\pause
\textbf{Examples:}
\begin{align*}
\qquad y'' + k^2 y & = 0 &
& \text{Two solutions are:} \quad y_1 = \cos (kx), \quad y_2 = \sin(kx) . \qquad \\
\qquad y'' - k^2 y & = 0 &
& \text{Two solutions are:} \quad y_1 = e^{kx}, \quad y_2 = e^{-kx} . \qquad
\end{align*}
\end{frame}

\begin{frame}
\begin{theorem}[Superposition]
Suppose $y_1$ and $y_2$ are two solutions of the
homogeneous equation
$y'' + p(x)y' + q(x)y = 0$.
\pause
Then 
\[
y(x) = C_1 y_1(x) + C_2 y_2(x)
\]
also solves the equation for arbitrary constants $C_1$ and $C_2$.
\end{theorem}
\pause

We call $C_1 y_1 + C_2 y_2$ a \emph{linear combination} of $y_1$ and $y_2$.

\medskip
\pause

\textbf{Proof:}
Let 
$y = C_1 y_1 + C_2 y_2$.  \pause Then

\medskip
$y'' + py' + qy  = (C_1 y_1 + C_2 y_2)'' + p(C_1 y_1 + C_2 y_2)' + q(C_1 y_1
+ C_2 y_2)$

\pause
\phantom{$y'' + py' + qy$}%
${} = C_1 y_1'' + C_2 y_2'' + C_1 p y_1' + C_2 p y_2' + C_1 q y_1 + C_2 q y_2$

\pause
\phantom{$y'' + py' + qy$}%
${} = C_1 ( y_1'' + p y_1' + q y_1 ) + C_2 ( y_2'' + p y_2' + q y_2 )$

\pause
\phantom{$y'' + py' + qy$}%
${} = C_1 \cdot 0 + C_2 \cdot 0 = 0$. \qed

\end{frame}

\begin{frame}

An \emph{operator} is a thing that eats functions and spits out functions.

\medskip
\pause

Define the operator $L$ by
\[
Ly = y'' + py' + qy .
\]

\medskip
\pause

The homogeneous differential equation becomes $Ly=0$.

\medskip
\pause

The operator $L$ is \emph{linear} if 
\[
L(C_1y_1 + C_2y_2) = 
C_1 Ly_1 + C_2 Ly_2
\quad \text{(almost like multiplying by $L$)}
\] 

\medskip
\pause

Nicer proof of the theorem:

\medskip
\pause

Suppose that $Ly_1 =0$ and $Ly_2=0$.
\pause
Then
\[
Ly = L(C_1y_1 + C_2y_2) \pause = 
C_1 Ly_1 + C_2 Ly_2 \pause = C_1 \cdot 0 + C_2 \cdot 0 = 0 .
\]

\end{frame}

\begin{frame}
\textbf{Example:}
$y'' - k^2y = 0$ has solutions
$y_1 = e^{kx}$ and 
$y_2 = e^{-kx}$.

\medskip
\pause

Recall that 
\quad
$\cosh t = \dfrac{e^{t}  + e^{-t}}{2}$,
\quad
$\sinh t = \dfrac{e^{t}  - e^{-t}}{2}$

\medskip
\pause

So
\quad
$\cosh(kx)
\pause = \dfrac{e^{kx}  + e^{-kx}}{2}
\pause = (\nicefrac{1}{2}) e^{kx}  + (\nicefrac{1}{2}) e^{-kx}
\pause = (\nicefrac{1}{2}) y_1  + (\nicefrac{1}{2}) y_2$
\quad
is a solution

\medskip
\pause

And
\quad
$\sinh(kx)
\pause = \dfrac{e^{kx}  - e^{-kx}}{2}
\pause = (\nicefrac{1}{2}) e^{kx}  + (-\nicefrac{1}{2}) e^{-kx}
\pause = (\nicefrac{1}{2}) y_1  + (-\nicefrac{1}{2}) y_2$
\quad
is a solution

\medskip
\pause

$\sinh$ and $\cosh$ are sometimes more convenient than the
exponential.

\medskip
\pause

\textbf{Exercise:} Verify that
\begin{align*}
& \cosh 0  = 1 , &   & \sinh 0 = 0 , \\
& \frac{d}{dt} \Bigl[ \cosh t \Bigr] = \sinh t , &  & \frac{d}{dt} \Bigl[ \sinh t \Bigr] = \cosh t , \\
& \cosh^2 t - \sinh^2 t = 1 .
\end{align*}

\end{frame}

\begin{frame}

\begin{theorem}[Existence and uniqueness]
Suppose $p, q, f$ are continuous functions on some interval
$I$, $a$ is a number in $I$,
and $b_0, b_1$ are constants.
\pause
Then the equation
\begin{equation*}
y'' + p(x) y' + q(x) y = f(x) ,
\end{equation*}
has exactly one solution $y(x)$ defined on the interval $I$ satisfying the initial conditions
\begin{equation*}
y(a) = b_0 , \qquad y'(a) = b_1 .
\end{equation*}
\end{theorem}

\pause

\textbf{Example:}
$y'' + k^2 y = 0$ with $y(0) = b_0$ and $y'(0) = b_1$ has the solution
\[
y(x) = b_0 \cos (kx) + \frac{b_1}{k} \sin (kx) .
\]
\pause
Verify IC: \quad $y(0) = b_0 \cos (0) + \dfrac{b_1}{k} \sin (0) = b_0$.

\medskip
\pause
$y'(x) = - k b_0 \sin (kx) + b_1 \cos (kx)$
\pause
\wthus
$y'(0) = - k b_0 \sin (0) + b_1 \cos (0) = b_1$.

\medskip
\pause

Similarly, $y'' - k^2 y = 0$ with $y(0) = b_0$ and $y'(0) = b_1$
has the solution
\[
y(x) = b_0 \cosh (kx) + \frac{b_1}{k} \sinh (kx) .
\]
\end{frame}

\begin{frame}
Two functions $y_1$ and $y_2$ are
\emph{linearly independent} if one is not a constant multiple of the other.

\pause

\begin{theorem}
Let $p, q$ be continuous functions.
Let $y_1$ and $y_2$ be two linearly independent
solutions to the homogeneous equation
$y'' + p(x) y' + q(x) y = 0$.
\pause
Then every other solution is 
of the form
\begin{equation*}
y = C_1 y_1 + C_2 y_2 .
\qquad \text{(i.e., that's the general solution)}
\end{equation*}
\end{theorem}

\pause

\textbf{Example:}
$y_1 = \sin x$ and $y_2 = \cos x$ solve $y'' + y = 0$.

\medskip
\pause

$\sin$ and $\cos$ are linearly independent:
\pause
If $\sin x = A \cos x$ for some constant $A$,
then let $x=0$ to get $A = 0$.
\pause But then $\sin x = 0$ for all
$x$, that's nonsense.

\medskip
\pause

$y_1$ and $y_2$ are linearly independent and
\[
y = C_1 \cos x + C_2 \sin x 
\]
is the general solution to $y'' + y = 0$.

\end{frame}

\begin{frame}
\textbf{Example:}
$y''-2x^{-2}y = 0$ has solutions $y_1 = x^2$ and $y_2 = \dfrac{1}{x}$.

\medskip
\pause

To see $y_1$ and $y_2$ are linearly independent, suppose
$y_1 = A y_2$.

\medskip
\pause

Solve for $A \pause = \dfrac{y_1}{y_2} = x^3$.
\pause
That's not a constant!
\pause \wthus $y_1$ and $y_2$ are linearly independent.

\medskip
\pause

\thus
\quad
$y = C_1 x^2 + C_2\dfrac{1}{x}$ \quad is the general solution.

\end{frame}

\begin{frame}
If you have one solution we can find the second
via the
\emph{reduction of order method}.

\medskip
\pause

Suppose $y_1$ solves
$y'' + p(x) y' + q(x) y = 0$.

\pause

Try to find a second solution of the form $y_2(x) = y_1(x) v(x)$.
\pause
Plug in:

\medskip

\quad
$
0 = y_2'' + p(x) y_2' + q(x) y_2
\pause
=
\underbrace{y_1'' v + 2 y_1' v' + y_1 v''}_{y_2''}
+ p(x) \underbrace{( y_1' v + y_1 v' )}_{y_2'}
+ q(x) \underbrace{y_1 v}_{y_2}$

\vspace*{-4pt}
\pause
\quad
\phantom{$0 = y_2'' + p(x) y_2' + q(x) y_2$}%
${} =
y_1 v''
+ (2 y_1' + p(x) y_1) v'
+
\cancelto{0}{\bigl( y_1'' + p(x) y_1' + q(x) y_1 \bigr)} v$.


\medskip
\pause

So
$y_1 v'' + (2 y_1' + p(x) y_1) v' = 0$.


\medskip
\pause

Write $w = v'$ and we have a first order ODE
\quad
$y_1 w' + (2 y_1' + p(x) y_1) w = 0$.

\medskip
\pause

Solve for $w$, and find $v$ by antidifferentiating.

\medskip
\pause

\textbf{Example:}
$y_1 = x$ is a solution to $y''+x^{-1}y'-x^{-2} y=0$, let's find $y_2$.

\medskip
\pause

\thus \quad $xw' + 3 w = 0$
\pause
\wthus
$w = Cx^{-3}$
\pause
\wthus
$v = \dfrac{-C}{2x^2}$
\pause
\wthus
$y_2 = y_1 v = \dfrac{-C}{2x}$

\medskip
\pause

Any $C$ works, e.g., $C=-2$ for $y_2 = \dfrac{1}{x}$.
\pause
\wthus
The general solution is \quad $y = C_1 x + C_2\dfrac{1}{x}$.

\end{frame}

\begin{frame}
We can even just write down a formula
\begin{equation*}
y_2(x) = y_1(x) \int \frac{e^{-\int p(x)\,dx}}{{\bigl(y_1(x)\bigr)}^2} \,dx
\end{equation*}
\end{frame}

\begin{frame}
A useful warm-up for next time:

\medskip

\textbf{Exercise:}
For $x^2 y'' - x y' = 0$, find two solutions, show that they
are linearly independent and find the general solution.

\pause
Hint: Try $y = x^r$.

\medskip
\pause

Equations of the form $a x^2 y'' + b x y' + c y = 0$ are called
\emph{Euler's equations} or
\emph{Cauchy--Euler equations}.

\end{frame}

\end{document}
