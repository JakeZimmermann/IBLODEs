\documentclass[10pt,aspectratio=169]{beamer}

% All the boilerplate is in deslides.sty
\usepackage{deslides}

\author{Ji\v{r}\'i Lebl}

\institute[OSU]{%
Oklahoma State University%
%Departemento pri Matematiko de Oklahoma {\^S}tata Universitato%
}

\title{12. Exact equations (part 2)\\(Notes on Diffy Qs, 1.8)}

\date{}

\begin{document}

\begin{frame}
\titlepage

%\bigskip

\begin{center}
The textbook: \url{https://www.jirka.org/diffyqs/}
\end{center}
\end{frame}

\begin{frame}
Brief review:

\medskip
\pause

$M \, dx + N \, dy = 0$ is exact if there is a
function $F$ such that $F_x = M$ and $F_y = N$.

\medskip
\pause

If such an $F$ exists, then $M_y = N_x$.

\pause

Conversely, if $M_y = N_x$, then 
locally such a function $F$ exists.

\medskip
\pause

To solve for $F$, integrate $M$ in $x$, differentiate
in $y$, and set equal to $N$.

\pause
(Or reverse the roles of $x$ and $y$ and $M$ and $N$).

\end{frame}

\begin{frame}

\textbf{Example:}
Solve
\quad
$x^2+y^2 + 2y(x+1) \frac{dy}{dx} = 0$.
\qquad
\pause
$M=x^2+y^2$ \quad and \quad $N=2y(x+1)$.

\medskip
\pause

Exercise:
Check the equation is exact.

\medskip
\pause

Integrate $M$ in $x$, differentiate in $y$ and set to $N$:

\medskip
\pause

$F(x,y) = \frac{1}{3}x^3 + xy^2 + A(y)$
\pause
\wthus
$2y(x+1) = 2xy + A'(y)$
\pause
\wthus
$A'(y)=2y$
\pause
\wthus
$A(y) = y^2$

\medskip
\pause

\thus \quad
$F(x,y) = \frac{1}{3}x^3 + xy^2 + y^2$.
\quad
\pause
Now solve $F(x,y) = C$.
\pause
\begin{equation*}
y^2 = \frac{C-(\nicefrac{1}{3})x^3}{x+1},
\qquad \text{so} \qquad
y = \pm \sqrt{\frac{C-(\nicefrac{1}{3})x^3}{x+1}} .
\end{equation*}
If $x=-1$, the explicit solution invalid.

\medskip
\pause

$x^2+y^2 + 2y(x+1) \frac{dy}{dx} = 0$ has no solution there,
note that term before
$\frac{dy}{dx}$ is zero.

\medskip
\pause

$(x^2+y^2) \, dx + 2y(x+1) \, dy = 0$ does have a solution $x=-1$.

\medskip
\pause

One could solve for $x$ in terms of $y$ for any initial condition (messy).
\end{frame}

\begin{frame}
\textbf{Integrating factors:}

Sometimes \quad $M\, dx + N \, dy = 0$ \quad is not exact, but \quad $u M \, dx + u N \, dy = 0$ \quad is.

\medskip
\pause

The linear
\quad
$\displaystyle
\frac{dy}{dx} + p(x) y = f(x)$, \quad
or \quad $\bigl( p(x) y - f(x) \bigr)\, dx +  dy  = 0$ \quad
is an example:

\medskip
\pause

Let $r(x) = e^{\int p(x)\,dx}$ and multiply
\qquad
$\displaystyle
\bigl(r(x) p(x) y - r(x) f(x) \bigr)\, dx + r(x) \, dy = 0$.

\medskip
\pause

$\dfrac{\partial}{\partial y}
\bigl[ r(x) p(x) y - r(x) f(x) \bigr] = r(x) p(x)$
\quad
and
\quad
$\dfrac{\partial}{\partial x}
\bigl[ r(x) \bigr] = r'(x) = r(x) p(x)$
\pause
\wthus
Exact!

\medskip
\pause

How to find the $u$ in general?

\medskip
\pause

$u$ must satisfy
\quad
$\displaystyle
\frac{\partial}{\partial y} \bigl[ u M \bigr] = 
u_y M + u M_y = 
\frac{\partial}{\partial x} \bigl[ u N \bigr] = 
u_x N + u N_x$.

\medskip
\pause

\thus \quad
$(M_y-N_x)u = u_x N - u_y M$.

\medskip
\pause

OK, that's another differential equation, how does that help?

\medskip
\pause

One strategy: Look for $u$ that is a function of $x$ alone, or $y$ alone.

\end{frame}

\begin{frame}
Let's look for $u(x)$.  Then $u_x = u'$ and $u_y = 0$.

\medskip
\pause

$(M_y-N_x)u = u_x N - u_y M$
\pause
\wthus
$(M_y-N_x)u = u' N$
\pause
\wthus
$\dfrac{M_y-N_x}{N}u = u'$

\medskip
\pause

If $\dfrac{M_y-N_x}{N}$ is a function of $x$ alone, then
\quad
$u' - \dfrac{M_y-N_x}{N} u = 0$ \quad is a linear equation!

\medskip
\pause

Let $P(x) = \frac{M_y-N_x}{N}$
\wthus
$u(x) = e^{\int P(x) \, dx}$

\medskip
\pause

Similarly if $u = u(y)$.
\pause
\wthus
$\dfrac{M_y-N_x}{M} u = - u'$

\medskip
\pause

If $\dfrac{M_y-N_x}{M}$ is a function of $y$ alone, then

\medskip
\pause

Let $Q(y) = \frac{M_y-N_x}{M}$
\wthus
$u(y) = e^{-\int Q(y) \, dy}$.

\end{frame}

\begin{frame}

\textbf{Example:}
Solve \quad
$\displaystyle
\frac{x^2+y^2}{x+1} + 2y \frac{dy}{dx} = 0$.

\medskip
\pause

$M= \frac{x^2+y^2}{x+1}$ and $N=2y$.

\medskip
\pause

$M_y-N_x = \frac{2y}{x+1} - 0 = \frac{2y}{x+1} \pause \not= 0$
\pause
\wthus
Not exact!

\medskip
\pause

Notice
\quad
$\displaystyle
\frac{M_y-N_x}{N}
\pause
= \frac{2y}{x+1} \, \frac{1}{2y}
\pause
= \frac{1}{x+1}
= P(x)
$
\quad
is a function of $x$ alone.

\medskip
\pause

Compute \quad
$u(x) = 
e^{\int  P(x) \, dx}
\pause
=
e^{\ln (x+1)} = x+1$.

\medskip
\pause

The equation becomes: \quad
$\displaystyle
x^2+y^2 + 2y(x+1) \frac{dy}{dx} = 0$.

\medskip
\pause

This is exact and we solved it a moment ago:

\medskip

\quad $\displaystyle y = \pm \sqrt{\frac{C-(\nicefrac{1}{3})x^3}{x+1}}$
\end{frame}

\begin{frame}
\textbf{Example:}
Solve
$\displaystyle
y^2 + (xy+1) \frac{dy}{dx} = 0$.

\medskip
\pause

$M_y-N_x \pause = 2y-y = y \pause \not= 0$
\wthus
Not exact.

\medskip
\pause

$\displaystyle
\frac{M_y-N_x}{M} \pause = \frac{y}{y^2} = \frac{1}{y} =Q(y)$ 
\quad
is a function of $y$ alone.

\medskip
\pause

$\displaystyle
u(y)
= e^{-\int  Q(y) \, dy}
\pause
=
e^{-\ln y} = \frac{1}{y}$.

\medskip
\pause

Equation becomes
\quad
$\displaystyle y + \frac{xy+1}{y} \frac{dy}{dx} = 0$. \quad (Exercise: exact)

\medskip
\pause

Solve: \quad
$F(x,y) = xy + A(y)$,
\pause
\quad
$\displaystyle
\frac{xy+1}{y} = x+\frac{1}{y} \pause = x+ A'(y)$.

\medskip
\pause

$A'(y) = \nicefrac{1}{y}$
\pause
\wthus
$A(y) = \ln \, \lvert y \rvert$
\pause
\wthus
 $F(x,y) = xy + \ln \, \lvert y \rvert$.

\medskip
\pause

Implicit solution:
$xy + \ln \, \lvert y \rvert = C$.

\medskip
\pause

We divided by $y$, so we should check $y=0$:
\pause
\quad
$y=0$ is a solution too.
\end{frame}

\end{document}
