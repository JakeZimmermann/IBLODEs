\documentclass[10pt,aspectratio=169]{beamer}

% All the boilerplate is in deslides.sty
\usepackage{deslides}

\author{Ji\v{r}\'i Lebl}

\institute[OSU]{%
Oklahoma State University%
%Departemento pri Matematiko de Oklahoma {\^S}tata Universitato%
}

\title{19. Nonhomogeneous equations,\\part 1: undetermined coefficients\\(Notes on Diffy Qs, 2.5)}

\date{}

\begin{document}

\begin{frame}
\titlepage

%\bigskip

\begin{center}
The textbook: \url{https://www.jirka.org/diffyqs/}
\end{center}
\end{frame}

\begin{frame}
How do we solve nonhomogeneous equations?  E.g.,
\[
y'' + 5y'+ 6y = 2x+1
\pause
\qquad
\text{(Write $Ly = 2x+1$ for short).}
\]
\pause
Consider the \emph{associated homogeneous equation}:
\[
y'' + 5y'+ 6y = 0
\pause
\qquad
\text{($Ly = 0$).}
\]
\pause
Its general solution $y_c$ (i.e. $Ly_c = 0$) is called the
\emph{complementary solution}.

\medskip
\pause

Find any one \emph{particular solution} $y_p$ of $Ly_p=2x+1$.

\medskip
\pause

$y = y_c + y_p$ \quad is the general solution to $Ly=2x+1$
as $L$ is a linear operator:
\pause
\[
L y = L ( y_c + y_p) \pause = L y_c + L y_p \pause = 0 + (2x+1).
\]
\pause
Why is it the \emph{general} solution?
Suppose
$y_p$ and $\tilde{y}_p$ be two particular solutions:
Then
\[
L(y_p-\tilde{y}_p) \pause =
Ly_p-L\tilde{y}_p \pause =
(2x+1)-(2x+1) = 0 .
\]
\pause
The difference is a solution to the
associated homogeneous equation.

\medskip
\pause

\textbf{Theorem:} If $y_c$ is the general solution to $Ly_c = 0$
and $y_p$ is any particular solution to $Ly_p = f(x)$, then
\[
y=y_c+y_p
\qquad
\text{is the general solution to $Ly = f(x)$.}
\]
\end{frame}

\begin{frame}
OK, so how do we find some particular $y_p$ solving
$y'' + 5y'+ 6y = 2x+1$.

\medskip
\pause

\textbf{Undetermined coefficients}

\medskip
\pause

We guess in a smart way.
We want $y_p$ such that $Ly_p = 2x+1$.
\pause
Derivatives of polynomials are polynomials.
\pause
So are their linear combinations.

\medskip
\pause

Try: \quad $y_p = Ax + B$
\pause
\[
y_p'' + 5y_p'+ 6y_p  \pause =
(Ax+B)'' + 5(Ax+B)' + 6(Ax+B)
\pause  = 
0 + 5A + 6Ax + 6B
\pause
= 6Ax+ (5A+6B) .
\]
\pause
We want \quad $6Ax+(5A+6B) = 2x+1$ \quad or \quad $6A=2$ and $5A+6B=1$.

\medskip
\pause

\thus \quad  $A = \nicefrac{1}{3}$ and $B = \nicefrac{-1}{9}$.

\medskip
\pause

Let \quad $\displaystyle y_p = \frac{1}{3}\, x - \frac{1}{9} = \frac{3x-1}{9}$.

\medskip
\pause

Solving the complementary problem:
\quad
$y_c = C_1 e^{-2x} + C_2 e^{-3x}$.

\medskip
\pause

The general solution is
\[
y = C_1 e^{-2x} + C_2 e^{-3x} + \frac{3x-1}{9} .
\]
\end{frame}

\begin{frame}
Suppose we had an initial value problem:
\quad $y'' + 5y'+ 6y = 2x+1$, \quad
$y(0) = 0$ and $y'(0) = \nicefrac{1}{3}$.

\medskip
\pause

\quad $y = C_1 e^{-2x} + C_2 e^{-3x} + \frac{3x-1}{9}$.
\qquad
\pause
$y' = - 2C_1 e^{-2x} - 3C_2 e^{-3x} + \nicefrac{1}{3}$.

\medskip
\pause
Then
$0 = y(0) \pause = C_1 + C_2 -\nicefrac{1}{9}$, \pause \qquad
$\nicefrac{1}{3} = y'(0) \pause = - 2C_1 - 3C_2 + \nicefrac{1}{3}$.
\medskip
\pause

Solve to get $C_1 = \nicefrac{1}{3}$ and $C_2 = \nicefrac{-2}{9}$.

\medskip
\pause

The solution to the initial value problem is
\[
y = \frac{1}{3} e^{-2x} - \frac{2}{9} e^{-3x} + \frac{3x-1}{9} =
\frac{3 e^{-2x} - 2 e^{-3x} + 3x-1}{9} .
\]
\pause

\textbf{Warning:}
Do not solve for constants in $y_c$ before adding $y_p$!
Write $y=y_c+y_p$ and \textit{then} solve for constants.

\medskip
\pause

\textbf{Note:}
Do not forget lower degree terms even if they don't appear on the right
hand side:

\medskip

E.g., for \quad $Ly = x^3+1$, \quad try $y_p = Ax^3+Bx^2+Cx+D$.

\end{frame}

\begin{frame}
Sines, cosines and exponentials are similar.

\pause
\medskip

\textbf{Example:}
Find a particular solution to
\quad
$y''+2y'+2y = \cos (2x)$.

\medskip
\pause

Linear combinations of derivatives of $\cos(2x)$ will be combinations
of $\cos (2x)$ and $\sin (2x)$.

\medskip
\pause

Try:
\quad
$y_p = A \cos (2x) + B \sin (2x)$.
\pause
\begin{multline*}
\underbrace{-4 A \cos (2x) - 4 B \sin (2x)}_{y_p''}
+2 \underbrace{\bigl(-2A \sin (2x) + 2B \cos (2x)\bigr)}_{y_p'}
\\
+
2 \underbrace{\bigl(A \cos (2x) + 2B \sin (2x)\bigr)}_{y_p}
= \cos (2x) ,
\end{multline*}
\pause
\thus\quad
$(-4A+4B+2A) \cos(2x) +
(-4B-4A+2B) \sin(2x)
= \cos(2x)$.

\medskip
\pause
\thus \quad $-4A + 4B + 2A = 1$ ~and~
$-4B - 4A + 2B = 0$
\pause
\wthus
$-2A+4B =1$ ~and~ $2A+B=0$

\pause
\thus
\quad
$A=\nicefrac{-1}{10}$ ~and~ $B=\nicefrac{1}{5}$.

\medskip
\pause

So \quad
$\displaystyle
y_p = A \cos (2x) + B \sin (2x) = \frac{-\cos (2x) + 2 \sin (2x)}{10} .
$
\end{frame}

\begin{frame}
Similarly for exponentials:

\medskip

For
\quad $Ly = e^{3x}$, \quad try
$y_p = A e^{3x}$.

\medskip
\pause

And we can combine exponentials, sines, cosines, and polynomials:

\medskip
\pause

For
\quad
$Ly = (1+3x^2)\,e^{-x}\cos (\pi x)$,

\medskip
try
$y_p = (A + Bx + Cx^2)\,e^{-x} \cos (\pi x) + 
(D + Ex + Fx^2)\,e^{-x} \sin (\pi x)$.

\medskip
\pause

Plug in, then solve for $A$, $B$, $C$, $D$, $E$, and $F$.
\end{frame}

\begin{frame}
Those guesses may not \emph{always} work. \pause
The guess could get eaten by the left hand side.

\medskip
\pause

\textbf{Example:}
Consider
\quad $y'' - 9y = e^{3x}$.

\medskip
\pause

Try guessing \quad $y_p = Ae^{3x}$:
\pause
\[
y_p''-9y_p = (9Ae^{3x}) - 9(Ae^{3x}) \pause = 0 \not= e^{3x} .
\]
\pause
No way to solve for $A$.

\medskip
\pause

\textbf{Hint:} To save work, compute the complimentary solution first.
Here, $y_c = C_1 e^{-3x} + C_2 e^{3x}$.

\medskip
\pause

\textbf{Trick:} Multiply guess by $x$: \quad $y_p = Axe^{3x}$.

\medskip
\pause

\quad $y_p' = Ae^{3x} + 3Axe^{3x}$ \quad and \quad $y_p'' = 6Ae^{3x} + 9Axe^{3x}$

\medskip
\pause

\quad $y_p'' -9y_p \pause = 6Ae^{3x} + 9Axe^{3x} - 9Axe^{3x} \pause = 6Ae^{3x}$.

\medskip
\pause

\thus \quad $6Ae^{3x} = e^{3x}$ \wthus $6A = 1$ \wthus $A=\nicefrac{1}{6}$.

\medskip
\pause

The general solution is
\[
y = y_c + y_p = 
C_1 e^{-3x} + C_2 e^{3x} + \frac{1}{6}\,xe^{3x} .
\]
\end{frame}

\begin{frame}
Multiplying by $x$ once may not be enough:

\medskip

\textbf{Example:}
Consider
\quad
$y''-6y'+9y = e^{3x}$.

\medskip
\pause

\quad $y_c = C_1 e^{3x} + C_2 x e^{3x}$.

\medskip
\pause

So $y_p = A xe^{3x}$ does not get rid of the duplication with $y_c$.

\medskip
\pause

Solution? \pause Guess \quad $y_p = Ax^2e^{3x}$.

\pause
\medskip
So keep multiplying by $x$ until duplication is gone but no more.
\end{frame}

\begin{frame}
What if the right-hand side has several terms, e.g.,
\[
Ly = e^{2x} + \cos x .
\]
\pause
Find a particular $u$ that solves
\[
Lu = e^{2x} .
\]
\pause
Find particular $v$ that solves
\[
Lv = \cos x .
\]
\pause
$y=u+v$ \quad solves \quad
$Ly = e^{2x} + \cos x$:
\[
Ly=L(u+v) = Lu+Lv = e^{2x} + \cos x .
\]
\end{frame}

\end{document}
