\documentclass[10pt,aspectratio=169]{beamer}

% All the boilerplate is in deslides.sty
\usepackage{deslides}

\author{Ji\v{r}\'i Lebl}

\institute[OSU]{%
Oklahoma State University%
%Departemento pri Matematiko de Oklahoma {\^S}tata Universitato%
}

\title{14. Constant coefficient second order linear ODEs (part 1)\\(Notes on Diffy Qs, 2.2)}

\date{}

\begin{document}

\begin{frame}
\titlepage

%\bigskip

\begin{center}
The textbook: \url{https://www.jirka.org/diffyqs/}
\end{center}
\end{frame}

\begin{frame}
Consider the problem
\qquad $y''-5y'+6y = 0$, \qquad $y(0) = 1$, \qquad $y'(0) = 7$.

\medskip
\pause

It's a second order \pause linear \pause homogeneous equation \pause  with
constant coefficients.

\pause
(\emph{Constant coefficients} means that in $y'' + p(x)y' + q(x)y = f(x)$,
the $p$ and $q$ are constants.)

\medskip
\pause

Let's guess a solution.
\pause
What function stays more or less the same when differentiated?

\pause
The exponential!

\medskip
\pause

We try ~$y=e^{rx}$ \quad (Germans call this ``ansatz'')
\pause
\qquad
$y' = r e^{rx}$ \qquad $y'' = r^2 e^{rx}$

\pause
\vspace*{-12pt}
\begin{align*}
\uncover<11->{y''-5y'+6y & = 0 , \\}
\uncover<12->{\underbrace{r^2 e^{rx}}_{y''} -5 \underbrace{r e^{rx}}_{y'}+6 \underbrace{e^{rx}}_{y} & = 0 , \\}
\uncover<13->{r^2 -5 r +6 & = 0 \qquad \text{(divide through by $e^{rx}$)},\\}
\uncover<14->{(r-2)(r-3) & = 0 .}
\end{align*}
\uncover<15->{So $r=2$ or $r=3$.}
\quad
\uncover<16->{Let $y_1 = e^{2x}$ and $y_2 = e^{3x}$.}

\medskip

\uncover<17->{
\textbf{Exercise:}
Check that $y_1$ and $y_2$ are solutions.
}
\end{frame}

\begin{frame}
Still considering
\qquad $y''-5y'+6y = 0$, \qquad $y(0) = 1$, \qquad $y'(0) = 7$.

\medskip

$e^{2x}$ and $e^{3x}$ are linearly independent:
\pause
If not, $e^{3x} = C e^{2x}$ for a constant $C$.
\pause
So $e^x = C$.
\pause
Nonsense!

\medskip
\pause

The general solution is \quad $y = C_1 e^{2x} + C_2 e^{3x}$
\pause
\qquad
$y' = 2 C_2 e^{2x} + 3 C_2 e^{3x}$

\medskip
\pause

$1 = y(0) \pause = C_1 + C_2$,

\medskip
\pause

$7 = y'(0) \pause = 2 C_1 + 3 C_2$.

\medskip
\pause

Solve!

\pause
e.g., $2 = 2C_1 + 2C_2$
\pause\wthus
$(7-2) = (2-2)C_1 + (3-2)C_2$
\pause
\wthus
$C_2 = 5$
\pause
\wthus
$C_1 = -4$.

\medskip
\pause

So the solution is: \quad
$y = -4 e^{2x} + 5 e^{3x}$

\end{frame}

\begin{frame}
Given
\quad
$a y'' + b y' + c y = 0$
\qquad
\pause
try $y = e^{rx}$:
\quad
$a r^2 e^{rx} + 
b r e^{rx} + 
c e^{rx} = 0$

\medskip
\pause
Divide by $e^{rx}$ to get the
\emph{characteristic equation} of the ODE:
\[
a r^2 + 
b r + 
c = 0 .
\]
\pause
Solve for the $r$:
\qquad 
$r_1, r_2 = \dfrac{-b \pm \sqrt{b^2 - 4ac}}{2a}$.

\medskip
\pause

\wthus $e^{r_1 x}$ and $e^{r_2 x}$ are solutions (a bit more complicated if
$r_1=r_2$)

\pause
\begin{theorem}
Suppose that $r_1$ and $r_2$ are the roots of the characteristic equation.
\begin{enumerate}[(i)]
\item\pause
If $r_1$ and $r_2$ are distinct and real (when $b^2 - 4ac > 0$),
the general solution is
\[
y = C_1 e^{r_1 x} + C_2 e^{r_2 x} .
\]
\item
\pause
If $r_1 = r_2$ (when $b^2 - 4ac = 0$), 
the general solution is
\[
y = (C_1 + C_2 x)\, e^{r_1 x} .
\]
\end{enumerate}
\end{theorem}
\end{frame}

\begin{frame}

\textbf{Example:}
Solve
\quad $y'' - k^2 y = 0$.

\medskip
\pause

Characteristic equation is $r^2 - k^2 = 0$ or $(r-k)(r+k) = 0$.

\medskip
\pause

\wthus $e^{-k x}$ and $e^{kx}$ are the two
linearly independent solutions,
\pause
and the general solution is
\[
y = C_1 e^{kx} + C_2e^{-kx} .
\]

\medskip
\pause

\textbf{Example:}
Solve \quad
$y'' -8 y' + 16 y = 0$.

\medskip
\pause

Characteristic equation is $r^2 - 8 r + 16 = {(r-4)}^2 = 0$.

\medskip
\pause

We have
a double root $r_1 = r_2 = 4$.
\medskip
The general solution is
\[
y = (C_1 + C_2 x)\, e^{4 x} = C_1 e^{4x} + C_2 x e^{4x} .
\]
\end{frame}

\begin{frame}
In a sense, a doubled root rarely happens for a randomly chosen equation,

but it does occur in real-world problems
(e.g., critically damped mass-spring system).

\medskip
\pause

Why does $xe^{rx}$ work if we the root is doubled?

\medskip
\pause

Think of two distinct roots $r_1$ and $r_2$ very close.
\pause
\[
\frac{e^{r_2 x} - e^{r_1 x}}{r_2 - r_1}
\qquad \text{is a solution.}
\]
\pause
Doubled root is like taking the limit
$r_1 \to r_2$.

\medskip
\pause

So we are taking the derivative of $e^{rx}$ as if $r$ is the variable:
\[
\frac{d}{dr} \bigl[ e^{rx} \bigr] = x e^{rx}
\]
\pause
Voila!

\end{frame}

\end{document}
