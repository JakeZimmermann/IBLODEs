\documentclass[red]{tutorial}
\usepackage[no-math]{fontspec}
\usepackage{xpatch}
	\renewcommand{\ttdefault}{ul9}
	\xpatchcmd{\ttfamily}{\selectfont}{\fontencoding{T1}\selectfont}{}{}
	\DeclareTextCommand{\nobreakspace}{T1}{\leavevmode\nobreak\ }
\usepackage{polyglossia} % English please
	\setdefaultlanguage[variant=us]{english}
%\usepackage[charter,cal=cmcal]{mathdesign} %different font
%\usepackage{avant}
\usepackage{microtype} % Less badboxes


\usepackage[charter,cal=cmcal]{mathdesign} %different font
%\usepackage{euler}
 
\usepackage{tikz}
\usepackage{pgfplots}
\usetikzlibrary{arrows.meta}

\usepackage{blindtext}
\usepackage{calc, ifthen, xparse, xspace}
\usepackage{makeidx}
\usepackage[hidelinks, urlcolor=blue]{hyperref}   % Internal hyperlinks
\usepackage{mathtools} % replaces amsmath
\usepackage{bbm} %lower case blackboard font
\usepackage{amsthm, bm}
\usepackage{thmtools} % be able to repeat a theorem
\usepackage{thm-restate}
\usepackage{graphicx}
\usepackage{xcolor}
\usepackage{multicol}
\usepackage{fnpct} % fancy footnote spacing

 
\newcommand{\xh}{{{\mathbf e}_1}}
\newcommand{\yh}{{{\mathbf e}_2}}
\newcommand{\zh}{{{\mathbf e}_3}}
\newcommand{\R}{\mathbb{R}}
\newcommand{\Z}{\mathbb{Z}}
\newcommand{\N}{\mathbb{N}}
\newcommand{\proj}{\mathrm{proj}}
\newcommand{\Proj}{\mathrm{proj}}
\newcommand{\Perp}{\mathrm{perp}}
\newcommand{\Span}{\mathrm{span}\,}
\newcommand{\Img}{\mathrm{img}\,}
\newcommand{\Null}{\mathrm{null}\,}
\newcommand{\Range}{\mathrm{range}\,}
\newcommand{\rref}{\mathrm{rref}}
\newcommand{\Rank}{\mathrm{rank}}
\newcommand{\nnul}{\mathrm{nullity}}
\newcommand{\mat}[1]{\begin{bmatrix}#1\end{bmatrix}}
\renewcommand{\d}{\mathrm{d}}
\newcommand{\Id}{\operatorname{id}}


\theoremstyle{definition}
\newtheorem{example}{Example}[section]
\newtheorem{defn}{Definition}[section]

%\theoremstyle{theorem}
\newtheorem{thm}{Theorem}[section]

\pgfkeys{/tutorial,
	name={Tutorial 4},
	author={Jason Siefken \& Bernardo Galv\~ao-Sousa},
	course={MAT 244},
	date={},
	term={},
	title={Linear Algebra \& Differential Equations}
	}

\begin{document}
	\begin{tutorial}
				\begin{objectives}
			In this tutorial you will explore some of the ways that Linear Algebra techniques can be applied to the
			study of differential equations.

				These problems relate to the following course learning objectives:
						\textit{Apply linear algebra techniques to classify solutions of linear systems of ordinary differential
			equations including rigorously classifying the stability of equilibrium solutions and creating
			linear approximations to non-linear systems of ordinary differential equations.}
		\end{objectives}

		\vspace{-.5em}
		\subsection*{Problems}
		\vspace{-.5em}

	%	In MAT223, you studied linear algebra in the context of $\R^n$ where vectors were geometric objects. We can expand our notion of 
	%	vectors to include \emph{functions} from $\R$ to $\R$ (you can think of functions as vectors in $\R^{\infty}$).

\begin{enumerate}
	\item\label{q1}
	Recall from Linear Algebra that for a matrix $M$, the complete solution to the equation $M\vec x=\vec b$ can be expressed as
	$
		\Null(M)+\{\vec p\}
	$
		where $\vec p$ is a particular solution to $M\vec x=\vec b$.
		
	\begin{enumerate}
		\item Suppose $\vec u=\mat{1\\2}$ and $\vec v=\mat{-2\\3}$ are two solutions to $M\vec x=\vec b$. Use this information
		to find at least three vectors in $\Null(M)$.
		\item Based on the information above, find three more solutions to $M\vec x=\vec b$.
		\item Do you expect $\vec u+\vec v$ to be a solution to $M\vec x=\vec b$? Why or why not?
	\end{enumerate}

	\item Consider the differential equation
	\[
		y''+y=2
	\]
	\begin{enumerate}
		\item Show that $u(t)=x^2+\sin t$ and $v(t)=x^2+3\sin t$ are solutions to the differential equation.
		\item Use a process similar to what you did in Question \ref{q1} to guess three additional solutions to the differential equation. Verify
		whether or not your guesses are actually solutions.
		\item The function $u(t)-v(t)=-2\sin t$ can be considered to be in the ``null space'' of some transformation. What is this transformation?
		\item Let $K\neq 1$. Why is $t^2+K\sin t$ a solution to the differential equation but $K(t^2+\sin t)$ not? Explain using linear algebra concepts.
	\end{enumerate}

	\item In this question, we will continue exploring the differential equation $y''+y=2$.
	
	Let $\mathcal C=\{\text{infinitely differentiable functions from $\R$ to $\R$}\}$, let $\Id:\mathcal C\to\mathcal C$
	be the identity transformation, and let $D^2:\mathcal C\to\mathcal C$ be the transformation that sends a function to its second derivative.
	Additionally,define $T=D^2+\Id$.

	\begin{enumerate}
		\item Show that $\Id$ and $D^2$ are linear transformations. (Use the same definition of Linear Transformation as in your Linear Algebra courses.)
		\item Show that $T$ is a linear transformation.
		\item Find $\Null(\Id)$ and $\Null(D^2)$.
		\item Show that $\sin$ and $\cos$ are in $\Null(T)$.
		\item You may take as a fact that $\Null(T)$ is two dimensional. Using this fact, find the complete solution to $y''+y=2$. Justify your steps.
	\end{enumerate}

\end{enumerate}

	\end{tutorial}

	\begin{solutions}
				\begin{enumerate}
			\item The \emph{projection} of $\vec v$ onto $X$ is the closest point in $X$ to $\vec v$.
			\item \begin{enumerate}
					\item $\phantom{x}$

\begin{center}
	\begin{tikzpicture}[scale=.8, >=latex]
    \begin{axis}[scale=1,
		    axis equal image,
		    %axis line style={draw=none},
		    axis lines=middle,
		    tick style={draw=none},
		    yticklabels={,,},
		    xticklabels={,,},
		 xmin=-.5,
		 xmax=1.5,
		 ymin=-.5,
		 ymax=1.5,
		 major grid style={dotted, gray},
                 xtick={-10,-9,...,10},
                 ytick={-10,-9,...,10},
                 grid=both,
		 anchor=origin]

	  \draw[green!50!black, very thick]
	    (0,0)-- (0,1)--
	    (1,1) node[above right, black] {$R$}-- (1,0) -- cycle
	    ;
    \end{axis}
\end{tikzpicture}
	\begin{tikzpicture}[scale=.8, >=latex]
    \begin{axis}[scale=1,
		    axis equal image,
		    %axis line style={draw=none},
		    axis lines=middle,
		    tick style={draw=none},
		    yticklabels={,,},
		    xticklabels={,,},
		 xmin=-.5,
		 xmax=1.5,
		 ymin=-.5,
		 ymax=1.5,
		 major grid style={dotted, gray},
                 xtick={-10,-9,...,10},
                 ytick={-10,-9,...,10},
                 grid=both,
		 anchor=origin]

	  \fill[fill=blue]
	    (0,0) circle[radius=2pt] (0,1) circle[radius=2pt]
	    (1,1) circle[radius=2pt] node[above right] {$C$} (1,0) circle[radius=2pt]
	    ;
    \end{axis}
\end{tikzpicture}
	\begin{tikzpicture}[scale=.8, >=latex]
    \begin{axis}[scale=1,
		    axis equal image,
		    %axis line style={draw=none},
		    axis lines=middle,
		    tick style={draw=none},
		    yticklabels={,,},
		    xticklabels={,,},
		 xmin=-1.5,
		 xmax=1.5,
		 ymin=-1.5,
		 ymax=1.5,
		 major grid style={dotted, gray},
                 xtick={-10,-9,...,10},
                 ytick={-10,-9,...,10},
                 grid=both,
		 anchor=origin]

	  \draw[red, very thick]
	    (0,0) circle[radius=1.41]
	    ;
	   \path (1,1) node[above right] {$S$};
    \end{axis}
\end{tikzpicture}
\end{center}
					\item $\phantom{x}$

\begin{center}
	\begin{tikzpicture}[scale=.8, >=latex]
    \begin{axis}[scale=1,
		    axis equal image,
		    %axis line style={draw=none},
		    axis lines=middle,
		    tick style={draw=none},
		    yticklabels={,,},
		    xticklabels={,,},
		 xmin=-.5,
		 xmax=2.5,
		 ymin=-.5,
		 ymax=2.5,
		 major grid style={dotted, gray},
                 xtick={-10,-9,...,10},
                 ytick={-10,-9,...,10},
                 grid=both,
		 anchor=origin]

	  \draw[green!50!black, very thick]
	    (0,0)-- (0,1)--
	    (1,1) node[above right, black] {$R$}-- (1,0) -- cycle
	    ;
	    \draw[orange, ->, very thick] (0,0) -- (.333,0) node[below] {$\vec v_1$};
	    \draw[orange, ->, very thick] (0,0) -- (2,2) node[below right] {$\vec v_2$};
	    \draw[orange, ->, very thick] (0,0) -- (2,1) node[right] {$\vec v_3$};
	    \draw[orange, ->, very thick] (0,0) -- (.333,2) node[above right] {$\vec v_4$};
    \end{axis}
\end{tikzpicture}
	\begin{tikzpicture}[scale=.8, >=latex]
    \begin{axis}[scale=1,
		    axis equal image,
		    %axis line style={draw=none},
		    axis lines=middle,
		    tick style={draw=none},
		    yticklabels={,,},
		    xticklabels={,,},
		 xmin=-.5,
		 xmax=2.5,
		 ymin=-.5,
		 ymax=2.5,
		 major grid style={dotted, gray},
                 xtick={-10,-9,...,10},
                 ytick={-10,-9,...,10},
                 grid=both,
		 anchor=origin]

	  \fill[fill=blue]
	    (0,0) circle[radius=2pt] (0,1) circle[radius=2pt]
	    (1,1) circle[radius=2pt] node[above right] {$C$} (1,0) circle[radius=2pt]
	    ;
	    \draw[orange, ->, very thick] (0,0) -- (.333,0) node[below] {$\vec v_1$};
	    \draw[orange, ->, very thick] (0,0) -- (2,2) node[below right] {$\vec v_2$};
	    \draw[orange, ->, very thick] (0,0) -- (2,1) node[right] {$\vec v_3$};
	    \draw[orange, ->, very thick] (0,0) -- (.333,2) node[above right] {$\vec v_4$};
    \end{axis}
\end{tikzpicture}
	\begin{tikzpicture}[scale=.8, >=latex]
    \begin{axis}[scale=1,
		    axis equal image,
		    %axis line style={draw=none},
		    axis lines=middle,
		    tick style={draw=none},
		    yticklabels={,,},
		    xticklabels={,,},
		 xmin=-1.5,
		 xmax=2.5,
		 ymin=-1.5,
		 ymax=2.5,
		 major grid style={dotted, gray},
                 xtick={-10,-9,...,10},
                 ytick={-10,-9,...,10},
                 grid=both,
		 anchor=origin]

	  \draw[red, very thick]
	    (0,0) circle[radius=1.41]
	    ;
	   \path (1,1) node[above right] {$S$};
	    \draw[orange, ->, very thick] (0,0) -- (.333,0) node[below] {$\vec v_1$};
	    \draw[orange, ->, very thick] (0,0) -- (2,2) node[below right] {$\vec v_2$};
	    \draw[orange, ->, very thick] (0,0) -- (2,1) node[right] {$\vec v_3$};
	    \draw[orange, ->, very thick] (0,0) -- (.333,2) node[above right] {$\vec v_4$};
    \end{axis}
\end{tikzpicture}
\end{center}

	\item
		\[
			\Proj_R\vec v_1=\vec v_1\qquad
			\Proj_R\vec v_2=\mat{1\\1}\qquad
			\Proj_R\vec v_3=\mat{1\\1}\qquad
			\Proj_R\vec v_4=\mat{1/3\\1}
		\]
		\[
			\Proj_C\vec v_1=\mat{0\\0}\qquad
			\Proj_C\vec v_2=\mat{1\\1}\qquad
			\Proj_C\vec v_3=\mat{1\\1}\qquad
			\Proj_C\vec v_4=\mat{0\\1}
		\]
		\[
			\Proj_S\vec v_1=\mat{\sqrt{2}\\0}\qquad
			\Proj_S\vec v_2=\mat{1\\1}\qquad
			\Proj_S\vec v_3=\frac{\sqrt{2}\vec v_3}{\|\vec v_3\|}\qquad
			\Proj_S\vec v_4=\frac{\sqrt{2}\vec v_4}{\|\vec v_4\|}
		\]

\end{enumerate}

	\item \begin{enumerate}
			\item $f(t) = \sqrt{(t-2)^2+(2t-2)^2} = \sqrt{5t^2-12t+8}$.
			\item $\Proj_\ell \vec r$ must be the multiple of $\vec d$ that is
				closest to $\vec r$. Notice that $f(t)$ is minimized when
				$5t^2-12t+8$ is minimized which happens when $t=6/5$. Therefore
				\[
					\Proj_\ell \vec r = \tfrac{6}{5}\mat{1\\2}.
				\]
			\item $90^\circ$.
			\item They are the same thing! To find the closest point on a line
				to a point, you draw a perpendicular to the line passing through the point.
				The intersection with the line is the closest point and is therefore the projection.
				However, the line in this case is given by
				all multiples of $\vec d$, so geometrically,
	\end{enumerate}
		\item  \begin{enumerate}
				\item $\mat{x\\y}\mapsto \mat{x\\0}$.
			\item Let $\vec d=\mat{2\\1}$ and $\vec x=\mat{x\\y}$.
			\item Let $\vec s=\mat{-1\\2}$ be the direction of the sunlight and let $\vec x$ be as before.
				We are looking for when $\vec x+t\vec s$ has a zero $y$-coordinate, which happens when
				$t=-y/2$. Therefore
				\[
					\mat{x\\y}\mapsto \mat{x\\y}+\tfrac{-y}{2}\mat{-1\\2} = \mat{x+y/2\\0}.
				\]
		\end{enumerate}

		\end{enumerate}
	

	
	\end{solutions}
	\begin{instructions}
		\subsection*{Learning Objectives}
	Students need to be able to\ldots
	\begin{itemize}
		\item Compute projections from the definition without memorizing a formula
		\item Exploit right angles to compute projections that would otherwise be
			difficult
	\end{itemize}

\subsection*{Context}
	Students have covered projections and components in class two weeks ago. Last week they started
		working with matrices/matrix multiplication.

	In this class, $\Proj_X\vec v$ is defined as the closest vector in $X$ to $\vec v$. It is \emph{not}
		defined in terms of a formula. Also, to avoid confusion, we \emph{don't write}
		$\Proj_{\vec u}\vec v$, since this is easily confused with projection onto a singleton.
		is orthogonal to $\vec u$. There is a formula for this operation (unlike projections), and they
		should all know it since they've used it on their homework.

\subsection*{What to Do}
	Introduce the learning objectives for the day's tutorial. Explain that, like most things in life,
		there is no algorithm to compute projections---the only way to do it is to understand
		the idea. Since we're practicing learning and understanding non-algorithmic tasks in this
		class, projections provide the perfect place to practice.

	Have students pair up and write the definition of projection. Again, make them
		write it. These definitions will show up on the midterms and many will
		write them wrong. Again, you might have some groups come up to the board and
		write their definitions and then have a short class discussion on whether they are right.
		This definition will be easier for them since there aren't any quantifiers.

		After everyone is on the same page with the definition, have them continue on \#2.
		The point of \#2 is to have them cement in their minds the link between projections
		and orthogonality. Don't give this point away too readily--it's best if they discover it themselves, so
		they \emph{own} it.

		After most of finished \#2, have a class discussion and repeat with \#3. Remember, your goal
		is not to get through problems. This tutorial has \emph{way} more problems than the students
		will be able to get through.

		7 minutes before the end of class, pick a problem that most students have started working on
		to do as a wrap-up.

\subsection*{Notes}
	\begin{itemize}
		\item The definition of $\Proj_X\vec v$ is written mostly with words instead of with
			set notation. This might throw some students off---they won't be able to tell that it's precise.
		\item Projections relate to orthogonality, but not always. \#2 is designed to tease this out---that
			for non-smooth shapes projections might not relate to orthogonality.
		\item With our definition, $\Proj_X\vec v$ might not be unique and so might not be well defined. We will
			never try to trick a student by giving them a non-unique projection. If a student asks about
			this, you can tell them that on tests the projection will always be unique.
		\item For \#3, student \emph{should} know how to minimize a parabola, but many won't. However, almost all
			have taken calculus in high school, so feel free to use basic calculus to answer that question.
		\item You won't get to \#4, but if students are working on it, make sure they draw a picture. It's very
			hard to do in one's head.
	\end{itemize}

	\end{instructions}

\end{document}
