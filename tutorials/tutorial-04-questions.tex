		\begin{objectives}
			Projections are used in mathematics, physics, computer science, and
			statistics.
	In this tutorial you will obtain a deeper understanding of projections.

	These problems relate to the following course learning objectives:
			\textit{Translate between algebraic and geometric viewpoints to solve problems},
			and
			\textit{work independently to understand concepts and procedures that have not been previously
			explained to you}.
		\end{objectives}

		\vspace{-.5em}
		\subsection*{Problems}
		\vspace{-.5em}

		Let $R$ be the square with side-length 1 and lower-left corner at $(0,0)$;
		let $C$ be the corners of $R$; and, let $S$ be the circle of radius $\sqrt{2}$ centered
		at $(0,0)$.

\begin{enumerate}
	\item Write down a mathematically precise definition of the projection of the vector $\vec v$ onto the set $X$.
	\item Let $\vec v_1=\mat{1/3\\0}$, $\vec v_2=\mat{2\\2}$, $\vec v_3=\mat{2\\1}$ and $\vec v_4=\mat{1/3\\2}$.
	\begin{enumerate}
		\item Draw $R$, $C$, and $S$ on separate grids.
		\item Using your drawings, estimate the projections of $\vec v_1$, \ldots, $\vec v_4$ onto $R$, $C$, and $S$.
		\item Compute exactly the projections of $\vec v_1$, \ldots, $\vec v_4$ onto $R$, $C$, and $S$.
	\end{enumerate}


	\item Let $\ell$ be the line given in vector form by $\vec x=t\vec d$ where $\vec d=\mat{1\\2}$, and let $\vec r=\mat{2\\2}$.
		\begin{enumerate}
			\item Find a formula for the distance between $t\vec d$ and $\vec r$ in terms of $t$.
			\item Compute $\Proj_\ell \vec r$.
			\item What is the angle between $\vec d$ and $\vec r-\Proj_\ell \vec r$?
			\item Explain the link between $\Proj_\ell \vec r$ and $\Comp_{\vec d}\vec r$.
		\end{enumerate}

	\item In $\R^3$, the projection of a vector onto the $xy$-plane can be thought of as the shadow
		of the vector at high noon. But where would this shadow be when it's not high noon?\

		To simplify things, let's work in $\R^2$. Assume the $x$-axis is the ``ground'' and the $y$-axis is
		pointing straight up. If the sun is located far off in the distance in the direction of $\vec e_2$,
		the shadow of a vector will be its projection onto the $x$-axis.
		\begin{enumerate}
			\item Find a formula for where the shadow of the vector $\vec x=\mat{x\\y}$ will be at high noon.
			\item Suppose the ground is modeled with the equation $y=\tfrac{1}{2}x$ and that at 4:00pm, the sun shines
				perpendicularly to the ground. That is, the sun is far away in the direction $\mat{-1\\2}$.
				Find a formula for the shadow of the vector $\vec x=\mat{x\\y}$ on the slanted ground at 4:00pm.
			\item The vector $\vec y=\mat{a\\b}$ is above flat ground (i.e., the ground is modeled by the $x$-axis).
				Find a formula for the position of the shadow of $\vec y$ at 4:00pm.
		\end{enumerate}

\end{enumerate}