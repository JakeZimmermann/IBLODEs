		\begin{objectives}
			In this tutorial you will explore some of the ways that Linear Algebra techniques can be applied to the
			study of differential equations.

				These problems relate to the following course learning objectives:
						\textit{Apply linear algebra techniques to classify solutions of linear systems of ordinary differential
			equations including rigorously classifying the stability of equilibrium solutions and creating
			linear approximations to non-linear systems of ordinary differential equations.}
		\end{objectives}

		\vspace{-.5em}
		\subsection*{Problems}
		\vspace{-.5em}

	%	In MAT223, you studied linear algebra in the context of $\R^n$ where vectors were geometric objects. We can expand our notion of 
	%	vectors to include \emph{functions} from $\R$ to $\R$ (you can think of functions as vectors in $\R^{\infty}$).

\begin{enumerate}
	\item\label{q1}
	Recall from Linear Algebra that for a matrix $M$, the complete solution to the equation $M\vec x=\vec b$ can be expressed as
	$
		\Null(M)+\{\vec p\}
	$
		where $\vec p$ is a particular solution to $M\vec x=\vec b$.
		
	\begin{enumerate}
		\item Suppose $\vec u=\mat{1\\2}$ and $\vec v=\mat{-2\\3}$ are two solutions to $M\vec x=\vec b$. Use this information
		to find at least three vectors in $\Null(M)$.
		\item Based on the information above, find three more solutions to $M\vec x=\vec b$.
		\item Do you expect $\vec u+\vec v$ to be a solution to $M\vec x=\vec b$? Why or why not?
	\end{enumerate}

	\item Consider the differential equation
	\[
		y''+y=2
	\]
	\begin{enumerate}
		\item Show that $u(t)=x^2+\sin t$ and $v(t)=x^2+3\sin t$ are solutions to the differential equation.
		\item Use what you learned in Question \ref{q1} to find three additional solutions to the differential equation.
		\item The function $u(t)-v(t)=-2\sin t$ can be considered to be in the ``null space'' of some transformation. What is this transformation?
		\item Let $K\neq 1$. Why is $x^2+K\sin t$ a solution to the differential equation but $K(x^2+\sin t)$ not? Explain using linear algebra concepts.
	\end{enumerate}

	\item In this question, we will continue exploring the differential equation $y''+y=2$.
	
	Let $\mathcal C=\{\text{infinitely differentiable functions from $\R$ to $\R$}\}$, let $\Id:\mathcal C\to\mathcal C$
	be the identity transformation and let $D^2:\mathcal C\to\mathcal C$ be the transformation that sends a function $f(t)$ to its second derivative $f''(t)$.

	Define $T=D^2+\Id$.

	\begin{enumerate}
		\item Show that $\Id$ and $D^2$ are linear transformations.
		\item Show that $T$ is a linear transformation.
		\item Find $\Null(\Id)$ and $\Null(D^2)$.
		\item Show that $\sin$ and $\cos$ are in $\Null(T)$.
	\end{enumerate}


	\item \begin{enumerate}
		\item Write down the definition of a \emph{subspace} of a vector space $X$.
		\item Let $\mathcal F=\{\text{functions from $\R$ to $\R$}\}$. Show that $\mathcal F$ satisfies the properties of a subspace.
	\end{enumerate}


	\item Let $\ell$ be the line given in vector form by $\vec x=t\vec d$ where $\vec d=\mat{1\\2}$, and let $\vec r=\mat{2\\2}$.
		\begin{enumerate}
			\item Find a formula for the distance between $t\vec d$ and $\vec r$ in terms of $t$.
			\item What is the angle between $\vec d$ and $\vec r-\Proj_\ell \vec r$?
		\end{enumerate}

	\item In $\R^3$, the projection of a vector onto the $xy$-plane can be thought of as the shadow
		of the vector at high noon. But where would this shadow be when it's not high noon?\

		To simplify things, let's work in $\R^2$. Assume the $x$-axis is the ``ground'' and the $y$-axis is
		pointing straight up. If the sun is located far off in the distance in the direction of $\vec e_2$,
		the shadow of a vector will be its projection onto the $x$-axis.
		\begin{enumerate}
			\item Find a formula for where the shadow of the vector $\vec x=\mat{x\\y}$ will be at high noon.
			\item Suppose the ground is modeled with the equation $y=\tfrac{1}{2}x$ and that at 4:00pm, the sun shines
				perpendicularly to the ground. That is, the sun is far away in the direction $\mat{-1\\2}$.
				Find a formula for the shadow of the vector $\vec x=\mat{x\\y}$ on the slanted ground at 4:00pm.
			\item The vector $\vec y=\mat{a\\b}$ is above flat ground (i.e., the ground is modeled by the $x$-axis).
				Find a formula for the position of the shadow of $\vec y$ at 4:00pm.
		\end{enumerate}

\end{enumerate}
