		\begin{objectives}
			In this tutorial you will explore some of the ways that Linear Algebra techniques can be applied to the
			study of differential equations.

				These problems relate to the following course learning objectives:
						\textit{Apply linear algebra techniques to classify solutions of linear systems of ordinary differential
			equations including rigorously classifying the stability of equilibrium solutions and creating
			linear approximations to non-linear systems of ordinary differential equations.}
		\end{objectives}

		\vspace{-.5em}
		\subsection*{Problems}
		\vspace{-.5em}

	%	In MAT223, you studied linear algebra in the context of $\R^n$ where vectors were geometric objects. We can expand our notion of 
	%	vectors to include \emph{functions} from $\R$ to $\R$ (you can think of functions as vectors in $\R^{\infty}$).

\begin{enumerate}
	\item\label{q1}
	Recall from Linear Algebra that for a matrix $M$, the complete solution to the equation $M\vec x=\vec b$ can be expressed as
	$
		\Null(M)+\{\vec p\}
	$
		where $\vec p$ is a particular solution to $M\vec x=\vec b$.
		
	\begin{enumerate}
		\item Suppose $\vec u=\mat{1\\2}$ and $\vec v=\mat{-2\\3}$ are two solutions to $M\vec x=\vec b$. Use this information
		to find at least three vectors in $\Null(M)$.
		\item Based on the information above, find three more solutions to $M\vec x=\vec b$.
		\item Do you expect $\vec u+\vec v$ to be a solution to $M\vec x=\vec b$? Why or why not?
	\end{enumerate}

	\item Consider the differential equation
	\[
		y''+y=2
	\]
	\begin{enumerate}
		\item Show that $u(t)=x^2+\sin t$ and $v(t)=x^2+3\sin t$ are solutions to the differential equation.
		\item Use a process similar to what you did in Question \ref{q1} to guess three additional solutions to the differential equation. Verify
		whether or not your guesses are actually solutions.
		\item The function $u(t)-v(t)=-2\sin t$ can be considered to be in the ``null space'' of some transformation. What is this transformation?
		\item Let $K\neq 1$. Why is $t^2+K\sin t$ a solution to the differential equation but $K(t^2+\sin t)$ not? Explain using linear algebra concepts.
	\end{enumerate}

	\item In this question, we will continue exploring the differential equation $y''+y=2$.
	
	Let $\mathcal C=\{\text{infinitely differentiable functions from $\R$ to $\R$}\}$, let $\Id:\mathcal C\to\mathcal C$
	be the identity transformation, and let $D^2:\mathcal C\to\mathcal C$ be the transformation that sends a function to its second derivative.
	Additionally,define $T=D^2+\Id$.

	\begin{enumerate}
		\item Show that $\Id$ and $D^2$ are linear transformations. (Use the same definition of Linear Transformation as in your Linear Algebra courses.)
		\item Show that $T$ is a linear transformation.
		\item Find $\Null(\Id)$ and $\Null(D^2)$.
		\item Show that $\sin$ and $\cos$ are in $\Null(T)$.
		\item You may take as a fact that $\Null(T)$ is two dimensional. Using this fact, find the complete solution to $y''+y=2$. Justify your steps.
	\end{enumerate}

\end{enumerate}
