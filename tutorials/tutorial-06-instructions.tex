\subsection*{Learning Objectives}
	Students need to be able to\ldots
	\begin{itemize}
		\item Use Euler's formula to find the real and imaginary parts of complex solutions.
		\item Use complex solutions to find real solutions to real differential equations.
	\end{itemize}

\vspace*{-.5cm}
\subsection*{Context}

In class we have seen matrix differential equations with complex eigen solutions. We have found real eigen solutions
based on these, but we haven't done the ``hard part'' of exploring why exactly taking the real/imaginary parts of complex
solutions provides us with real solutions (of course this doesn't always work, but it often does).

This tutorial is chance for students to work through the details of these complex-number computations.

\subsection*{What to Do}
	Introduce the learning objectives for the day's tutorial. Explain that using complex numbers is a useful 
	intermediate step for solving differential equations, even if the solution in the end should be real. Remind the
	students what Euler's formula is (it's written at the top of their worksheet) and briefly go over what the real/imaginary 
	part of a complex number is. Stress that the \emph{imaginary} part of a complex number is actually a \emph{real} number (i.e., you
	drop the $i$).
	
	Have students get into groups and start on \#1. This question has many parts and may take most students the whole tutorial.
	Some students may confuse Euler's formula with Euler's method. Let these students know that these are different things, and Euler's
	formula is provided at the top of the worksheet.
	
	1e and 1g are good questions for wrapping up.

\subsection*{Notes}
	\begin{enumerate}
		\item Students may think that the imaginary part of $a+bi$ is $bi$. Correct them and let them know that it is just $b$.
		\item This question looks a lot like what we do in class. It should be straight forward after they did \#1, but it may be time consuming.
	\end{enumerate}