\begin{enumerate}
		\item \begin{enumerate}
			\item Differentiating $f'=ie^{it}$ and so $f''=i^2e^{it}=-e^{it}=-f$.
			\item Yes. The equation is a real equation and so it should have a real solution.
			\item By Euler's formula, $e^{it}=\cos t + i\sin t$, so the real part of $f$ is $\cos t$. 

			Differentiating, we see $\cos '' = -\cos$, so it is still a solution.
			\item By Euler's formula, $e^{it}=\cos t+i\sin t$, so the imaginary part of $f$ is $\sin t$.

			Differentiating, we see $\sin '' = -\sin$, so it is still a solution.
			\item Again, we can apply Euler's formula. We see
			\[
				e^{it}=\cos t+i\sin t \qquad\text{and}\qquad e^{-it}=\cos(-t) + i\sin(-t)=\cos t-i\sin t,
			\]
			so
			\[
				\tfrac{1}{2}e^{it}+\tfrac{1}{2}e^{-it}=\cos t.
			\]
			Similarly,
			\[
				\tfrac{-i}{2}e^{it}+\tfrac{i}{2}e^{-it}=\sin t.
			\]
			
			\item Again, we can apply Euler's formula.
			\[
				e^{(a+bi)t}=e^{at}e^{ibt}=e^{at}(\cos (bt)+i\sin (bt))
				\qquad\text{and}\qquad
				e^{(a-bi)t}=e^{at}e^{-ibt}=e^{at}(\cos (bt)-i\sin (bt))
			\]
			and so
			\[
				\tfrac{1}{2}e^{(a+bi)t}+\tfrac{1}{2}e^{(a-bi)t}=e^{at}\cos(bt)
			\]
				and 
			\[
				\tfrac{-i}{2}e^{(a+bi)t}+\tfrac{i}{2}e^{(a-bi)t}=e^{at}\sin(bt).
			\]
			Notice that $e^{at}\cos(bt)$ and $e^{at}\sin(bt)$ are linearly independent functions that are real.

			\item A solution is $g(t)=e^{it}$. The real part of $g(t)$ is $\cos t$ and the imaginary part is $\sin t$.
			Neither of these are solutions to $y'=iy$. Since $y'=iy$ is not a real equation, we don't expect real solutions,
			so we wouldn't expect the real and imaginary parts of $g$ to be a solution.

		\end{enumerate}

		\item
		\begin{enumerate}
			\item The phase portrait shows arrows going in a circle. There should be real solutions to this equation,
			because Euler's method will simulate a solution and there is no reason to expect that Euler's method will
			not converge.
			\item The eigenvalues of $M$ are $\pm i$ and the eigenvectors are $\mat{1\\-i}$ with eigenvalue $i$ and $\mat{1\\i}$
			with eigenvalue $-i$.
			\item The eigen solutions are $e^{it}\mat{1\\-i}$ and $e^{-it}\mat{1\\i}$. They are linearly independent and so form a basis.
			\item We can find a real basis either by finding a linear combination of the complex eigen solutions that is real or by 
			guessing and checking. We can guess that the real and imaginary parts of the complex eigen solutions are real solutions.
			
			Expanding with Euler's formula, we see
			\[
				e^{it}\mat{1\\-i}=\mat{e^{it}\\-ie^{it}}=\mat{\cos t+i\sin t\\\sin t-i\cos t},
			\]
			and so we guess that $\vec s_1(t) = \mat{\cos t\\\sin t}$ and $\vec s_2(t)=\mat{\sin t\\-\cos t}$ are solutions. Differentiating,
			we indeed see that these are solutions. Since they are real, linearly independent, and we have two of them, we have found a basis
			of real solutions.
			\item There is no real eigen solution to $\vec r\,'=M\vec r$. We have already found a basis of eigen solutions to the equation.
			All other eigen solutions must be multiples of the eigen solutions we found. However, there is no complex eigen solution that will
			make $e^{it}\mat{1\\-i}$ or $e^{it}\mat{1\\i}$ real.
		\end{enumerate}

\end{enumerate}