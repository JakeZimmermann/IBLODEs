\begin{objectives}
	In this tutorial you will learn how to leverage Euler's method to approximate the solution of boundary-value problems.

	This tutorial relates to the following course learning objectives:
	\textit{
		Use a computer to approximate the solutions to differential equations and systems of differential
		equations and explain the advantages and draw-backs of computer-based approximations
	}.
\end{objectives}


\vspace{-.5em}
\subsection*{Problems}
\vspace{-.5em}




%%%%%%%%%%%%%%%%%%%%%%%%%%


\begin{enumerate}
	\item\label{QIVP} Consider the initial-value problem:
	      \begin{equation}\tag{IVP}\label{IVP}
		      \begin{cases}
			      w''(t) = -w(t) \\
			      w(0) = 0       \\
			      w'(0) = 5
		      \end{cases}
	      \end{equation}


	      \begin{enumerate}
		      \item Let $\vec{r}(t) = \mat{w \\ w'}$. 
			  Write the system of differential equations above in matrix form (i.e., as $\vec{r}'=M \vec{r}$). 
			  What is the initial condition for $\vec{r}$?
		      \item Use a spreadsheet to approximate the value of $\vec{r}(1)$ (use a step size $\leq 0.01$).
		      \item Find an approximate value for $w(1)$.
	      \end{enumerate}



	\item Consider the boundary-value problem:
	      \begin{equation}\tag{BVP}\label{BVP}
		      \begin{cases}
			      w''(t) = -w(t) \\
			      w(0) = 0       \\
			      w(1) = 1
		      \end{cases}
	      \end{equation}


	      %	We want to approximate the solution of this differential equation with these two boundary conditions \eqref{BVP}.

	      In Question \ref{QIVP}, we approximated
		   a second-order differential equation with initial conditions by applying Euler's method to a corresponding system.
		   We will modify this approach to approximate the solution to \eqref{BVP}.

	      This method is called the \textbf{Shooting Method}.

	      Notation: \textit{In this question, $w_a(t)$ will denote the Euler approximation to Equations 
		  \eqref{IVP} with $w'(0)=a$.}


	      \begin{enumerate}
		      \item Find a number $a$ such that if $w'(0)=a$, then the Euler approximation satisfies $w_{a}(1)<1$.

		      \item Find a number $b$ such that If $w'(0)=b$, then the Euler approximation satisfies $w_{b}(1)>1$.

		      \item Let $c = \frac{a+b}{2}$. Use Euler's Method again to approximate $w_c(1)$.

			  \item If the boundary problem has a solution, there is a value $k$ so that $w_k(1)=1$.
			  Based on your previous results, is $k\in [a,c]$ or is $k\in [c,b]$? Explain.

		      \item We will try to find $k$ by using \emph{bisection}. That is, when we identify an interval
			  $[a_i, b_i]$ that must contain $k$, we will test $w_{c_i}(1)$ where $c_i=\frac{a_i+b_i}{2}$ and then
			  narrow our search to either $[a_i, c_i]$ or $[c_i,b_i]$.

		      Using $a_1=a$, $b_1=b$, and $c_1=c$ (which you calculated already), fill out the following table:

		            \begin{tabular}{|c||c|c||c|c||c|c|}
			            \hline
			            $i$ & \hspace{20pt} $a_i$ \hspace{20pt} & \hspace{10pt} $w_{a_i}(1)$  \hspace{10pt} & $b_i$  \hspace{20pt} & \hspace{10pt} $w_{b_n}(1)$ \hspace{10pt} & \hspace{20pt} $c_i$  \hspace{20pt} & \hspace{10pt} $w_{c_i}(1)$ \hspace{10pt} \\ \hline
			            1   &                                   &                                           &                      &                                          &                                    &                                          \\[10pt]\hline
			            2   &                                   &                                           &                      &                                          &                                    &                                          \\[10pt]\hline
			            3   &                                   &                                           &                      &                                          &                                    &                                          \\[10pt]\hline
			            4   &                                   &                                           &                      &                                          &                                    &                                          \\[10pt]\hline
			            5   &                                   &                                           &                      &                                          &                                    &                                          \\[10pt]\hline
			            6   &                                   &                                           &                      &                                          &                                    &                                          \\[10pt]\hline
			            7   &                                   &                                           &                      &                                          &                                    &                                          \\[10pt]\hline
		            \end{tabular}
		      \item The iterative process you just performed is called the \emph{Shooting Method}\footnote{ In this case,
			  you used bisection to narrow down the ``target''; the shooting method does not require bisection. For example,
			  you could use Newton's method to narrow the target.}.
			  
			  Use the following Desmos plot to compare the solution you arrived at via the Shooting Method 
			  with the exact solution.
			  
			 \url{https://www.desmos.com/calculator/mvjr18tt2x}
	      \end{enumerate}


	\item The BVP below has infinitely many solutions.
	      \begin{equation}\tag{$\star$}\label{BVP2}
		      \begin{cases}
			      w''(t) = -\pi^2 \cdot w(t) + \cos\Big(w(t)\Big) \\
			      w(0) = 0                   \\
			      w(1) = 0
		      \end{cases}
	      \end{equation}

	      Use the Shooting Method to identify at least \emph{two} solutions to this boundary value problem.

		  Hint: You may need to pick very different initial values for $w'(0)$ to find intervals that contain different solutions.

\end{enumerate}



%
%
%\vfill
%
%
%
%\paragraph{Shooting Method:}	 To approximate the solution of the boundary-value problem:
%$$
%	\begin{cases}
%		w'' = F\big(t,w,\tfrac{dw}{dt}\big) \\
%		w(t_0)=w_0                          \\
%		w(t_1)=w_1
%	\end{cases}
%$$
%\begin{enumerate}[label={(Step \arabic*)}]
%	\item Find two values $a$ and $b$ such that:
%	      \begin{itemize}
%		      \item If $w'(t_0)=a$, then the Euler approximation satisfies $w_a(t_1)<w_1$
%		      \item If $w'(t_0)=b$, then the Euler approximation satisfies $w_b(t_1)>w_1$
%	      \end{itemize}
%
%	\item \label{step}Let $c = \frac{a+b}{2}$. With $w'(t_0)=c$, use Euler's method to obtain an approximation of $w_c(t_1)$.
%
%	\item Repeat \ref{step} with new values for $(a,c)$ or $(b,c)$ instead of $(a,b)$, making sure that $w_1 \in [w_a(t_1), w_b(t_1)]$.
%\end{enumerate}















%%%%%%%%%%%%%%%%%%%%%%%%%%





