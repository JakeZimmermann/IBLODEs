\subsection*{Learning Objectives}
Students need to be able to\ldots
\begin{itemize}
	\item Create affine approximations to a differential equations
	\item Use affine approximation to determine the stability of equilibrium solutions
	\item Recognize that affine approximations are still approximations even when not centered at an equilibrium solution
\end{itemize}

\subsection*{Context}
In lecture, students have linearized a 1d differential equation and used it to analyze the stability of equilibrium solutions.
We have also done this \emph{one time} for a 2d system. 1d affine approximations are familiar to the students from Calc I,
but 2d affine approximations are new to them (though everyone should be taking or have already taken multivariable calculus at this point).

\subsection*{What to Do}
Start the tutorial by stating the day's learning objectives. There is one focus: understand
how to find affine approximations to differential equations and what those affine approximations tell us.

Have everyone get into groups and start on \#1. This will take most students most of the time. If they
do finish \#1, \#2 will take all the remaining time.

6 minutes before the end of class, pick a suitable problem to do as a wrap-up. Unless everyone has finished \#1,
picking one equilibrium for 1h is probably a good choice.

\subsection*{Notes}
\begin{enumerate}
	\item The first several parts of this question focus on an affine approximation \emph{away} from an equilibrium solution.
	      This is on purpose.
	      \begin{enumerate}
		      \item
		      \item Most of the equilibrium solutions we study in this class are at the origin, but not this time. This question
		            is a sanity check to see if students have any idea about what is going on in the course at this point. Most students
		            should have a quick answer to this question. If they don't force them to think about it---they shouldn't be skipping it!
		      \item If they struggle, prod them to think about ``linear approximations'' from Calculus, or about tangent lines.
		      \item Many will choose the vertical axis as the ``$y$'' axis. It is the horizontal axis in this case.
		      \item If they struggle, remind them it is separable.
		      \item Judgement call for the student. Their estimation will be wrong anyways, but making guesses is always a good thing!
		      \item This should be \emph{very} routine for them at this point. To copy-and-paste into Desmos, you must omit the header
		            columns from the spreadsheet.
		      \item They have to put all their knowledge together in this part. If they get stuck, encourage them to work through the
		            previous steps replacing $y=0$ with an equilibrium solution they found.

	      \end{enumerate}
	\item Only the quick students will make it this far. It basically mimics \#1 but in 2d and starting with an equilibrium point
	      rather than some random point.
\end{enumerate}