\subsection*{Learning Objectives} Students need to be able to\ldots
	\begin{itemize}
		\item Explore how different parameters affect the qualitative
			behaviour of solutions to a system of differential equations.
		\item Translate back and forth between a model's equations and
			the real-world situation that the model describes.
	\end{itemize}


	\subsection*{Context} In class we have studied the Fox and Rabbit
	Lotka-Volterra system with fixed parameters. We have made a phase portrait
	and simulated solutions using multi-dimensional Euler's method with
	a spreadsheet. We also analyzed equilibrium solutions and their
	relationship to phase portraits. However, we have only done this once.
	Students need more practice!


\subsection*{What to Do} This is a groupwork tutorial,
	but students may not be used to working in groups.

\subsection*{Notes}
		\begin{enumerate}
			\item This question should be straight forward and serve
				as a warm-up for the tutorial. Don't spend a lot
				of time on this question.
			\item This question is \emph{hard} and will take most of the
				tutorial. The only way to get qualitatively different
				behaviour is by setting $b$ or $c$ to zero.

				One way to see this is that there is always an
				equilibrium in the first quadrant as long as all
				parameters are strictly positive,
				and populations will cycle around the equilibrium.

				This will not be obvious to students because as they
				change their parameters, the their phase
				portrait won't rescale to show them what's really going
				on. When they get to this point (and guess they have found
				different qualitative behaviour even though they haven't)
				have them explicitly calculate the equilibrium points
				and have them think about when the \emph{equilibrium
				points} exhibit a different qualitative behavriour
				(e.g., they align on the axes instead of one being
				in the first quadrant).
			\item The purpose of this question is to show students that
				even without solving a differential equation, we can
				still make mathematically rigorous claims about the
				behaviour of solutions.

				This question is a ``bonus'' of sorts and you should not
				spend time discussing this question as a group unless
				most of the tutorial students have already spent time on this question.
		\end{enumerate}
