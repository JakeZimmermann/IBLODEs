\documentclass[red]{tutorial}
\usepackage[no-math]{fontspec}
\usepackage{xpatch}
	\renewcommand{\ttdefault}{ul9}
	\xpatchcmd{\ttfamily}{\selectfont}{\fontencoding{T1}\selectfont}{}{}
	\DeclareTextCommand{\nobreakspace}{T1}{\leavevmode\nobreak\ }
\usepackage{polyglossia} % English please
	\setdefaultlanguage[variant=us]{english}
%\usepackage[charter,cal=cmcal]{mathdesign} %different font
%\usepackage{avant}
\usepackage{microtype} % Less badboxes

%\usepackage{enumitem}

\usepackage[charter,cal=cmcal]{mathdesign} %different font
%\usepackage{euler}
 
\usepackage{blindtext}
\usepackage{calc, ifthen, xparse, xspace}
\usepackage{makeidx}
\usepackage[hidelinks, urlcolor=blue]{hyperref}   % Internal hyperlinks
\usepackage{mathtools} % replaces amsmath
\usepackage{bbm} %lower case blackboard font
\usepackage{amsthm, bm}
\usepackage{thmtools} % be able to repeat a theorem
\usepackage{thm-restate}
\usepackage{graphicx}
\usepackage[dvipsnames]{xcolor}
\usepackage{multicol}
\usepackage{fnpct} % fancy footnote spacing
\usepackage{tikz}
\usetikzlibrary{arrows.meta}

\usepackage{pgfplots}
\pgfplotsset{compat=1.18}
%\pgfkeys{/pgf/fpu}

 \usepackage{enumitem}
 
\newcommand{\xh}{{{\mathbf e}_1}}
\newcommand{\yh}{{{\mathbf e}_2}}
\newcommand{\zh}{{{\mathbf e}_3}}
\newcommand{\R}{\mathbb{R}}
\newcommand{\Z}{\mathbb{Z}}
\newcommand{\N}{\mathbb{N}}
\newcommand{\proj}{\mathrm{proj}}
\newcommand{\Proj}{\mathrm{proj}}
\newcommand{\Perp}{\mathrm{perp}}
\renewcommand{\span}{\mathrm{span}\,}
\newcommand{\Span}{\mathrm{span}\,}
\newcommand{\Img}{\mathrm{img}\,}
\newcommand{\Null}{\mathrm{null}\,}
\newcommand{\Range}{\mathrm{range}\,}
\newcommand{\rref}{\mathrm{rref}}
\newcommand{\rank}{\mathrm{rank}}
\newcommand{\Rank}{\mathrm{rank}}
\newcommand{\nnul}{\mathrm{nullity}}
\newcommand{\mat}[1]{\begin{bmatrix}#1\end{bmatrix}}
\newcommand{\chr}{\mathrm{char}}
\renewcommand{\d}{\mathrm{d}}


\theoremstyle{definition}
\newtheorem{example}{Example}[section]
\newtheorem{defn}{Definition}[section]

%\theoremstyle{theorem}
\newtheorem{thm}{Theorem}[section]

\pgfkeys{/tutorial,
	name={Tutorial 9},
	author={Jason Siefken \& Bernardo Galv\~ao-Sousa},
	course={MAT 244},
	date={},
	term={},
	title={Shooting Method}
	}

\begin{document}
	\begin{tutorial}
				\begin{objectives}
			In this tutorial you will learn how to leverage Euler's method to approximate the solution of boundary-value problems.
				
			These tutorial relates to the following course learning objective:
			\textit{Build a spreadsheet that can use Euler's method to approximate the solution to a system of differential equations}, \textit{Rewrite a higher-order ODE as a system of ODEs}, \textit{Use Euler's method to simulate the solution to a higher-order ODE}, \textit{Numerically approximate the solution to a boundary value problem involving higher-order ODEs}.
		\end{objectives}


		\vspace{-.5em}
		\subsection*{Problems}
		\vspace{-.5em}




%%%%%%%%%%%%%%%%%%%%%%%%%%


\begin{enumerate}
	\item Consider the initial-value problem:
		\begin{equation}\tag{IVP}\label{IVP}
		\begin{cases}
			w''(t) = -w(t) \\
			w(0) = 0 \\
			w'(0) = 5
		\end{cases}
	\end{equation}

		
		\begin{enumerate}
			\item Let $\vec{r}(t) = \mat{w \\ w'}$. Write the system of differential equations above as $\vec{r}'=M \vec{r}$. What is the initial condition for $\vec{r}$?
			\item In Excel, approximate the solution $\vec{r}(1)$ using Euler's Method with $\Delta t=0.01$.
			\item Give an approximation for $w(1)$.
		\end{enumerate}
	


	\item Consider the boundary-value problem:
		\begin{equation}\tag{BVP}\label{BVP}
		\begin{cases}
			w''(t) = -w(t) \\
			w(0) = 0 \\
			w(1) = 1
		\end{cases}
		\end{equation}

		
%	We want to approximate the solution of this differential equation with these two boundary conditions \eqref{BVP}.
	
	In the previous exercise, we learned how to approximate a second-order differential equation with initial conditions \eqref{IVP} using Euler's method.

	We will take advantage of this algorithm to approximate the solution to \eqref{BVP}.
	
	The method is called the \textbf{Shooting Method}.
	
	\textit{Notation: In this question, we will denote by $w_a(t)$ the Euler's approximation for the solution of \eqref{IVP} with $w'(0)=a$.}
	
	
	\begin{enumerate}
		\item Find a number $a$ such that If $w'(0)=a_1$, then the Euler approximation satisfies $w_{a_1}(1)<1$.

		\item Find a number $b$ such that If $w'(0)=b_1$, then the Euler approximation satisfies $w_{b_1}(1)<1$.

		\item Let $c_1 = \frac{a_1+b_1}{2}$. Use Euler's Method again to approximate $w_c(1)$.
		
		\item We know that the solution of \eqref{BVP} satisfies $w(1)=1$. Which of the following intervals is best?
		\begin{itemize}
			\item $[a_1,c_1]$
			\item $[c_1,b_1]$
		\end{itemize}
		
		\item Denote the interval you chose by $[a_2,b_2]$. Let $c_2 = \frac{a_2+b_2}{2}$ and use Euler's Method again to approximate $w_{c_2}(1)$.

		\item Repeat this process until you find $c_n$ such that the Euler approximation $w_{c_n}(1)$ is very close to $1$. Fill in the following table:
		
		\begin{tabular}{|c||c|c||c|c||c|c|}
			\hline
			$n$ & \hspace{20pt} $a_n$ \hspace{20pt} &  \hspace{10pt} $w_{a_n}(1)$  \hspace{10pt} & $b_n$  \hspace{20pt} &  \hspace{10pt} $w_{b_n}(1)$ \hspace{10pt} &  \hspace{20pt} $c_n$  \hspace{20pt} & \hspace{10pt} $w_{c_n}(1)$ \hspace{10pt} \\ \hline
			 1 & & & & & & \\[10pt]\hline
  			 2 & & & & & & \\[10pt]\hline
			 3 & & & & & & \\[10pt]\hline
  			 4 & & & & & & \\[10pt]\hline
			 5 & & & & & & \\[10pt]\hline
  			 6 & & & & & & \\[10pt]\hline
  			 7 & & & & & & \\[10pt]\hline
		\end{tabular}
		
		
		
		

		\item Use this Desmos \url{https://www.desmos.com/calculator/mvjr18tt2x} to compare your results with the solution.
	\end{enumerate}


	\item The BVP below has infinitely many solutions.
	\begin{equation}\tag{$\star$}\label{BVP2}
		\begin{cases}
			w''(t) = -\pi^2 \cdot w(t) \\
			w(0) = 0 \\
			w(1) = 0
		\end{cases}
	\end{equation}
	
	Use the Shooting method with different initial guesses $a_1$ and $b_1$ to approximate different solutions.

\end{enumerate}





\vfill



	\paragraph{Shooting Method:}	 To approximate the solution of the boundary-value problem:
	$$
	\begin{cases}
		w'' = F\big(t,w,\tfrac{dw}{dt}\big) \\
		w(t_0)=w_0 \\
		w(t_1)=w_1
	\end{cases}
	$$
	\begin{enumerate}[label={(Step \arabic*)}]
		\item Find two values $a$ and $b$ such that:
		\begin{itemize}
			\item If $w'(t_0)=a$, then the Euler approximation satisfies $w_a(t_1)<w_1$
			\item If $w'(t_0)=b$, then the Euler approximation satisfies $w_b(t_1)>w_1$
		\end{itemize}
	
		\item \label{step}Let $c = \frac{a+b}{2}$. With $w'(t_0)=c$, use Euler's method to obtain an approximation of $w_c(t_1)$.

		\item Repeat \ref{step} with new values for $(a,c)$ or $(b,c)$ instead of $(a,b)$, making sure that $w_1 \in [w_a(t_1), w_b(t_1)]$.
	\end{enumerate}
	
	
	
	
	
	
	
	
	






%%%%%%%%%%%%%%%%%%%%%%%%%%






	\end{tutorial}

	\begin{solutions}
				\begin{enumerate}
			\item
				\begin{enumerate}
					\item
				We can use the determinant to compute the volume of each gem.
				\[
					\text{vol}(X) = |\det([\vec a|\vec b|\vec c])| = 5\qquad
					\text{vol}(Y) = |\det([\vec c|\vec d|\vec e])| = 2\qquad
					\text{vol}(Z) = |\det([\vec a|\vec d|\vec e])| = 1
				\]
			So gem $X$ has the largest volume.
					\item One way to define ``pointiness'' of a parallelepiped
						is as the ratio between the volume of the parallelepiped if
						it were a rectangular prism and its true volume. A sharp ``point''
						of the crystal will cause it to have less volume than a point that is close
						to $90^\circ$.

						Computing,
						\[
							\|\vec a\|=\sqrt{2}\qquad
							\|\vec b\|=\sqrt{6}\qquad
							\|\vec c\|=\sqrt{11}\qquad
							\|\vec d\|=\sqrt{3}\qquad
							\|\vec e\|=1,
						\]
						so the pointiness ratios are
						\[
							\text{rat}(X)=\frac{5}{\sqrt{2}\sqrt{6}\sqrt{11}}\approx 0.435\qquad
							\text{rat}(Y)=\frac{2}{\sqrt{11}\sqrt{3}(1)}\approx 0.348\qquad
							\text{rat}(Z)=\frac{1}{\sqrt{2}\sqrt{3}(1)}\approx 0.408.
						\]
						Using this measure, gem $Y$ would be the pointiest.
				\end{enumerate}
			\item After row reduction we get
				\[
					x=\frac{d}{ad-bc}\qquad y=\frac{-c}{ad-bc}.
				\]
				In both cases, we're dividing by $\det\left(\mat{a&b\\c&d}\right)=0$, which is not allowed.

			\item The largest possible volume occurs when $\vec f$ is orthogonal to $\vec a$ and $\vec d$.
				By inspection we see that $\vec f=t\mat{1\\-1\\0}$. Since $\vec f$ is a unit vector, we know
				$\vec f=\pm\mat{1/\sqrt{2}\\-1/\sqrt{2}\\0}$. Computing,
				\[
					\det([\vec a|\vec d|\vec f]) = \pm \frac{2}{\sqrt{2}},
				\]
				and so $2/\sqrt{2}$ is the largest possible volume.
			\item Let $\vec C=\vec u\times \vec v$. Since $\vec u\perp \vec v$, we know $\|\vec C\|=\|\vec u\|\|\vec v\|=\sqrt{42}$.
				Further, $\vec C$ is orthogonal to $\vec u$ and $\vec v$, so $\vec C\in \text{null}(M)$ where $M$ is the matrix
				with rows $\vec u$ and $\vec v$. Row reducing,
				\[
					\text{rref}(M) = \mat{1&0&-5\\0&1&4},
				\]
				and so $\vec C=t\mat{5\\-4\\1}$. Since $\|\vec C\|=\sqrt{42}$, we in fact see $\vec C=\pm \mat{5\\-4\\1}$. Finally,
				computing
				\[
					\det([\vec u|\vec v|\vec C]) = +42
				\]
				when $\vec C=-\mat{5\\-4\\1}$ and so $\vec u\times \vec v = \mat{-5\\4\\-1}$.
		\end{enumerate}

	
	\end{solutions}
	\begin{instructions}
		\subsection*{Learning Objectives}
Students need to be able to\ldots
\begin{itemize}
	\item Build a spreadsheet that can use Euler's method to approximate the solution to a system of differential equations
	\item Rewrite a higher-order ODE as a system of ODEs.
	\item Use Euler's method to simulate the solution to a higher-order ODE.
	\item Numerically approximate the solution to a boundary value problem involving higher-order ODEs.
\end{itemize}

\subsection*{Context}

In class we used Euler's method extensively.

We also introduced boundary-value problems (BVPs) in class to show that studying existence 
and uniqueness of solution is not a trivial matter.

For this type of differential equation, however, we can't use Euler's method directly, since it requires initial conditions.
Instead, we repeatedly apply Euler's method to find better and better guesses at the initial conditions.


\subsection*{Resources for TAs}

Excel spreadsheet with the Shooting Method.

\begin{itemize}
	\item \url{https://utoronto-my.sharepoint.com/:x:/g/personal/bernardo_galvao_sousa_utoronto_ca/EQO08o4-PdFMr8PMd80LmP0BPilrLuGIcNMpP-oVpfM4ow?e=IR39N4}
\end{itemize}




\subsection*{What to Do}
Introduce the learning objectives for the day's tutorial.

Explain the big picture: if we can rewrite a higher-order ODE in terms of a system of first-order ODEs, 
we can use Euler's method to approximate the solution to the higher-order ODE.

Have students get into small groups and start on \#1. Each group needs to have at least 1 laptop.
This question should be done quickly---it's setting the stage for \#2.




\subsection*{Notes}

	\begin{enumerate}
		\item This should be a quick review.
		\item This question carefully walks students through the Shooting Method. If students get stuck on
		this one, go ahead and give them a big-picture overview of the Shooting Method.
		\item This is here for the quicker students, though they may need a smaller step size than $\Delta=0.01$ to get good results.
	\end{enumerate}






	\end{instructions}

\end{document}
