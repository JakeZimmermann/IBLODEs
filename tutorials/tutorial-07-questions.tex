\begin{objectives}
	In this tutorial you will using linearization to classify equilibrium solutions.

	These problems relate to the following course learning objectives:
	\textit{Apply linear algebra techniques to classify solutions of linear systems of ordinary differential
	equations including rigorously classifying the stability of equilibrium solutions and creating
	linear approximations to non-linear systems of ordinary differential equations}.
\end{objectives}


\subsection*{Problems}

\begin{enumerate}
	\item 	Consider the differential equation
	      \begin{equation}
		      y'=(y-2)(y+1)\tag{A}
	      \end{equation}

	      Define $f(y)=(y-2)(y+1)$ and notice that Equation (A) can be written as $y'=f(y)$.

	      \begin{enumerate}
		      \item Look at the slope field for Equation (A).

		            \url{https://www.desmos.com/calculator/z1c415u4fb}

		            Based on the slope field, describe what the shape of solutions to Equation (A) should look like.

		      \item Is $y=0$ an equilibrium solution to Equation (A)? Explain both analytically (e.g., in terms of equations)
		            and qualitatively (e.g., by referring to the slope field).
		      \item Find an \emph{affine approximation} (i.e. an approximation of the form $A(x) = \alpha x+\beta$)
		            to the function $f$ near $0$.\footnote{ In your calculus class, you might have used the term
			            \emph{linear approximation} instead of affine approximation.} Call this affine approximation $A_0$.

		      \item\label{sketch} Sketch the graphs of $f$ and $A_0$ on the same coordinate plane. For what values of $y$ would you
		            call $A_0(y)$ a ``good'' approximation to $f(y)$?

		            \textbf{Warning}: pay close attention to what your
		            axes represent in your sketch. The $y$-axis may not be the one you think!

		      \item Solve the differential equation $y'=A_0(y)$ with initial condition $y(0)=0$. Call your solution $y_{\text{approx}}$.

		      \item\label{errorest} Based on your answer to \ref{sketch}, for how long do you expect $y_{\text{approx}}$ to
		            be a ``good'' approximation to the solution of Equation (A) with initial condition $y(0)=0$?

		      \item Use a spreadsheet and Euler's method with step size $\Delta=0.01$ to simulate a solution to Equation (A) with initial condition $y(0)=0$.
		            Use Desmos to compare your simulated solution to $y_{\text{approx}}$.
		            When does $y_{\text{approx}}$ stop being a good approximation? Does this match your intuition from \ref{errorest}?

		      \item Find the equilibrium solutions to Equation (A). For each equilibrium solution $e$, find an affine
		            approximation, $A_e$, to $f$ near the equilibrium solution. For each equilibrium solution $e$, (i) solve
		            the differential equation $y'=A_e(y)$ and, (ii) use your solutions to classify the nature of each equilibrium solution
		            as attracting/repelling/etc..
	      \end{enumerate}

	\item

	      Recall that for a function $\vec F(x,y)=\Big(F_1(x,y), F_2(x,y)\Big)$,
	      the \emph{total derivative} of $\vec F$ at $\vec E$ can be expressed
	      as the matrix
	      $
		      D_{\vec F}(\vec E) = \mat{
			      \rule[-0.5cm]{0pt}{.8cm}\displaystyle \frac{\partial F_1}{\partial x} &\displaystyle  \frac{\partial F_1}{\partial y}\\
			      \rule[-0.2cm]{0pt}{.3cm}\displaystyle  \frac{\partial F_2}{\partial x} &\displaystyle  \frac{\partial F_2}{\partial y}
		      }
	      $
	      evaluated at $\vec E$.

	      \bigskip

	      Define $\vec F:\R^2\to\R^2$ by
	      \[
		      \vec F(x,y) = \mat{(y-2)(x+1)\\x^2-y}
	      \]
	      and consider the differential equation
	      \begin{equation}
		      \mat{x\\y}' = \vec F(x,y).\tag{B}
	      \end{equation}
	      Define $\vec r(t) = \mat{x(t)\\y(t)}$. Note that we may express Equation (B) as $\vec r\,'(t) = \vec F\Big(\vec r(t)\Big)$.
	      \begin{enumerate}
		      \item Can Equation (B) be written in matrix or affine form?
		      \item Verify that $\vec r(t) = \mat{-1\\1}$ is an equilibrium solution to Equation (B).
		      \item Make a phase portrait for Equation (B) and use your phase portrait to guess the nature
		            of the equilibrium solution $\vec r(t) = \mat{-1\\1}$.
		      \item Find the \emph{total derivative} of $\vec F$ at $\mat{-1\\1}$.
			  \item Make an affine approximation to $\vec F$ near $\mat{-1\\1}$. Call your affine approximation $\vec A$.
			  \item Use eigenvalue/eigenvector analysis to classify the equilibrium solutions for $\vec r\,' = \vec A(\vec r)$.
			  \item Rigorously classify the equilibrium solution $\vec r(t)=\mat{-1\\1}$ for Equation (B).
			  \item Classify the remaining equilibrium solutions for Equation (B).
		      
	      \end{enumerate}
\end{enumerate}
