\subsection*{Learning Objectives}
Students need to be able to\ldots
\begin{itemize}
	\item Find affine approximations to a differential equation or a system of differential equations centered
	      at an equilibrium solution.
	\item Explain why an equilibrium solution may be unstable even if the equilibrium solution
	      for the corresponding affine approximation is stable.
\end{itemize}

\subsection*{Context}

In class we have only used affine approximations to classify equilibrium solutions when the
nature of the equilibrium solution is the same for the differential equation and the affine approximation.
Since this is not always the case, we need to train students how to tell if their affine approximation
gives the required information.


\subsection*{What to Do}

This tutorial is as usual. Start by stating the learning objectives for the day. Then have students
get into groups and start on the problems. Walk around and help encourage groups who are stuck.
6 minutes before the end of tutorial, pick a suitable problem to do as a wrap-up.

\subsection*{Notes}
\begin{enumerate}
	\item 
	\begin{enumerate}
		\item It should be obvious, but students may need to ``shrink the window'' to see details of
		the slope field near the origin.
		\item This may take them some time. If they are stuck tell them to check their work from last tutorial.
		\item We mostly do this in 2d. They should know how to do it in 1d, though. For some of the students,
		telling them to think of $1\times 1$ matrices will help (for others that will just confuse them).
		\item 
		\item Have them actually write down their test. This will help them clarify their ideas.
	\end{enumerate}
	\item
	\begin{enumerate}
		\item This may take them some time, but they need to be able to do this for the exams.
		\item
		\item 
		\item 
		\item They may need step sizes smaller than $\Delta=0.01$ to get a reasonable approximation.
		The closer to the origin they are, the smaller the needed step size is.
		\item Encourage them to make a table or list to enumerate all the options.
	\end{enumerate}
\end{enumerate}