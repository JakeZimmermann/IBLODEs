\documentclass[red]{tutorial}
\usepackage[no-math]{fontspec}
\usepackage{xpatch}
	\renewcommand{\ttdefault}{ul9}
	\xpatchcmd{\ttfamily}{\selectfont}{\fontencoding{T1}\selectfont}{}{}
	\DeclareTextCommand{\nobreakspace}{T1}{\leavevmode\nobreak\ }
\usepackage{polyglossia} % English please
	\setdefaultlanguage[variant=us]{english}
%\usepackage[charter,cal=cmcal]{mathdesign} %different font
%\usepackage{avant}
\usepackage{microtype} % Less badboxes

%\usepackage{enumitem}

\usepackage[charter,cal=cmcal]{mathdesign} %different font
%\usepackage{euler}
 
\usepackage{blindtext}
\usepackage{calc, ifthen, xparse, xspace}
\usepackage{makeidx}
\usepackage[hidelinks, urlcolor=blue]{hyperref}   % Internal hyperlinks
\usepackage{mathtools} % replaces amsmath
\usepackage{bbm} %lower case blackboard font
\usepackage{amsthm, bm}
\usepackage{thmtools} % be able to repeat a theorem
\usepackage{thm-restate}
\usepackage{graphicx}
\usepackage[dvipsnames]{xcolor}
\usepackage{multicol}
\usepackage{fnpct} % fancy footnote spacing
\usepackage{tikz}
\usetikzlibrary{arrows.meta}

\usepackage{pgfplots}
\pgfplotsset{compat=1.18}
%\pgfkeys{/pgf/fpu}

 
\newcommand{\xh}{{{\mathbf e}_1}}
\newcommand{\yh}{{{\mathbf e}_2}}
\newcommand{\zh}{{{\mathbf e}_3}}
\newcommand{\R}{\mathbb{R}}
\newcommand{\Z}{\mathbb{Z}}
\newcommand{\N}{\mathbb{N}}
\newcommand{\proj}{\mathrm{proj}}
\newcommand{\Proj}{\mathrm{proj}}
\newcommand{\Perp}{\mathrm{perp}}
\renewcommand{\span}{\mathrm{span}\,}
\newcommand{\Span}{\mathrm{span}\,}
\newcommand{\Img}{\mathrm{img}\,}
\newcommand{\Null}{\mathrm{null}\,}
\newcommand{\Range}{\mathrm{range}\,}
\newcommand{\rref}{\mathrm{rref}}
\newcommand{\rank}{\mathrm{rank}}
\newcommand{\Rank}{\mathrm{rank}}
\newcommand{\nnul}{\mathrm{nullity}}
\newcommand{\mat}[1]{\begin{bmatrix}#1\end{bmatrix}}
\newcommand{\chr}{\mathrm{char}}
\renewcommand{\d}{\mathrm{d}}


\theoremstyle{definition}
\newtheorem{example}{Example}[section]
\newtheorem{defn}{Definition}[section]

%\theoremstyle{theorem}
\newtheorem{thm}{Theorem}[section]

\pgfkeys{/tutorial,
	name={Tutorial 8},
	author={Jason Siefken \& Bernardo Galv\~ao-Sousa},
	course={MAT 244},
	date={},
	term={},
	title={Linearization II}
	}

\begin{document}
	\begin{tutorial}
				\begin{objectives}
	In this tutorial you will be constructing matrices and linear transformations that satisfy
			given conditions, or explaining why they don't exist.

	These problems relate to the following course learning objectives:
			\textit{Use matrices to solve problems},
			\textit{translate between algebraic and geometric viewpoints to solve problems}, and
			\textit{clearly and correctly express the mathematical ideas of linear algebra to others}.
		\end{objectives}


\begin{enumerate}
	\item Write mathematically precise definitions of the rank of a matrix $A$ and the rank of a linear transformation $\mathcal{T}$.

	\item Give an example of a $2\times 3$ matrix $A$ with the specified rank, or explain why it cannot exist.
	\begin{enumerate}
		\item $\Rank(A) = 1$
		\item $\Rank(A) = 2$
		\item $\Rank(A) = 3$
		\item $\Rank(A) = 0$
	\end{enumerate}

	\item For a linear transformation $\mathcal L: \R^n\to \R^m$, explain how $\Rank(\mathcal L)$ relates to $m$ or $n$ under the following conditions:
	\begin{enumerate}
		\item $\mathcal L$ is one-to-one.
		\item $\mathcal L$ is {\bf not} one-to-one.
		\item $\mathcal L$ is onto.
		\item $\mathcal L$ is {\bf not} onto.
	\end{enumerate}

	\item Give examples of linear transformations $\mathcal T, \mathcal S: \R^3\to\R^3$ that satisfy the following, or explain why they cannot exist.
	\begin{enumerate}
		\item $\Rank(\mathcal T)=\Rank(\mathcal S)=\Rank(\mathcal S\circ \mathcal T)=2$
		\item $\Rank(\mathcal T)=\Rank(\mathcal S)=2$, and $\Rank(\mathcal S\circ \mathcal T)=1$
		\item $\Rank(\mathcal T)=\Rank(\mathcal S)=2$, and $\Rank(\mathcal S\circ \mathcal T)=3$
		\item $\Rank(\mathcal T)=2$, $\Rank(\mathcal S)=1$ and $\Rank(\mathcal S\circ \mathcal T)=0$
	\end{enumerate}

	\item Tommy has returned once again and is working on similar matrices.
	\begin{enumerate}
		\item Write down a mathematically precise definition for two matrices to be similar.

\item
Tommy began with a $3\times 3$ matrix $A$ and multiplied by a change of basis matrix $X$ to find $B=XAX^{-1}$. His matrix computations gave
			\[A=\mat{
1 & 0 & 0  \\
0 & 2 & 0 \\
0 & 0 & 3
}\quad \text{ and } \quad
			B=\mat{
1 & 1 & 0  \\
2 & -1 & 0 \\
0 & 0 & 0
}.
\]
	Unfortunately, Tommy lost his paper containing $X$.
			Can you help him by finding a change of basis matrix
			$X$ that gives this solution or explaining to Tommy why no such matrix exists?
		\end{enumerate}
\end{enumerate} 
	\end{tutorial}

	\begin{solutions}
		\begin{enumerate}
	\item
	      \begin{enumerate}
		      \item
		            \textbf{Equation (A)}: Stable and attracting.

		            \textbf{Equation (B)}: Stable and attracting.

		            \textbf{Equation (C)}: Unstable and repelling.

				\item Equation (A) has an affine approximation $y'=-y$. Equations (B) and (C) 
					have same affine approximation: $y'=0$.
		      \item Every affine approximation has a stable equilibrium solutions at $0$. The affine approximation for Equation (A) is
			  	also attracting.
		      \item Affine approximations are based off of the first derivative. For Equations (B) and (C), the first
		            derivative at the origin is zero, so it fails to differentiate between cases.
		      \item Given an autonomous equation $y'=f(y)$ with an equilibrium at $y=k$, the equilibrium
		            is attracting if $f'(k) < 0$, repelling if $f'(k) > 0$ and more investigation is needed if $f'(k) = 0$.
	      \end{enumerate}

	\item
	      \begin{enumerate}
		      \item $\vec r\,' = \mat{0&-1\\1&0}\vec r$.
		      \item The equilibrium for the affine approximation is stable (not repelling nor attracting).
		      \item No. Similar to Question 1, there are cubics that the first derivative fails to pick up on. These
		            cubics surely cause problems!
		      \item It looks like solutions circle about the origin, but it is hard to tell much beyond that.
		      \item Based on numerical simulations using Euler's method with $\Delta <0.01$, it appears that solutions
		            very slowly circle inwards.
		      %\item If we consider the differential equations $\vec r\,' =\mat{-y\\x}$ and $\vec r\,'=\mat{-x^3\\-y^3}$
		      %      in isolation, the first one has periodic solutions that circle about the origin and the second one
		      %      has solutions that head straight towards the origin. Equation (D) is a sum of these two equations, so
		      %      it makes sense that solutions circle but also tend towards the origin.

		      %      We can quantify how much solutions move towards the origin. We know $\vec r\,'(x,y)$ is a tangent
		      %      vector to a solution curve at the point $(x,y)$ and that the vector $(x,y)$ points radially out
		      %      from the equilibrium solution to the point $(x,y)$. Thus, computing
		      %      \[
			  %          \vec r\,'\cdot \mat{x\\y}=\mat{-y-x^3\\x-y^3}\cdot \mat{x\\y}=-(x^4+y^4)
		      %      \]
		      %      we see that the angle between $\vec r\,'$ and $(x,y)$ is always greater than $90^\circ$. Thus, solution
		      %      curves tend slightly towards the origin, making the equilibrium attracting.

		      \item If the real part of all eigenvalues is non-zero we can use the eigenvalues to classify the equilibrium
		            as follows. If all real parts are positive, the equilibrium is repelling and unstable. If all real parts are negative, the equilibrium
		            is attracting and stable. If there is a mix of positive and negative real parts, the equilibrium is unstable.

		            Otherwise, if at least one real part is positive, we know the equilibrium is unstable (but cannot determine whether it is repelling).
		            If at least one real part is positive and one real part is negative, we know the equilibrium is unstable
		            and not repelling.

		            In all other cases, we cannot conclude from the eigenvalues the nature of the equilibrium.
	      \end{enumerate}
\end{enumerate}
	
	\end{solutions}
	\begin{instructions}
		\subsection*{Learning Objectives}
Students need to be able to\ldots
\begin{itemize}
	\item Find affine approximations to a differential equation or a system of differential equations centered
	      at an equilibrium solution.
	\item Explain why an equilibrium solution may be unstable even if the equilibrium solution
	      for the corresponding affine approximation is stable.
\end{itemize}

\subsection*{Context}

In class we have only used affine approximations to classify equilibrium solutions when the
nature of the equilibrium solution is the same for the differential equation and the affine approximation.
Since this is not always the case, we need to train students how to tell if their affine approximation
gives the required information.


\subsection*{What to Do}
\begin{itemize}
	\item Give the students a couple of minutes to think about
	      problem 1. Then invite four of them up to the board. Have
	      two define ``rank''
	      of a matrix and two define ``rank'' of a linear
	      transformation. Discuss any issues that their
	      definitions may have. Make sure to end this
	      portion with correct, clear definitions on
	      the board. And make sure that the students
	      understand them.

	\item Next, arrange the students in groups and give them
	      10-12 minutes to work on problem 2 and 4. Be sure to
	      walk around the room and try to get the students engaged
	      and working. Once time is up, spend a few minutes at
	      the board addressing any glaring issues you might've
	      noticed while you were circling the room.

	\item Repeat the same process with either problem 3 or
	      5. Ask the students to see if they prefer one problem
	      over the other. Wrap-up the tutorial by discussing any
	      issues you noticed while circling the room.
\end{itemize}

\subsection*{Notes}
\begin{itemize}
	\item Be mindful of late-comers. During group work, try to approach them and make sure they are aware of what is going on, and that they are engaged and working in a group.
	\item Try to keep an organized board. A disorganized and chaotic board is just as annoying to the students as a jumbled mess on a test paper that you're trying to grade is to you.
\end{itemize}
	\end{instructions}

\end{document}
