\subsection*{Learning Objectives}
	Students need to be able to\ldots
	\begin{itemize}
		\item Compute projections from the definition without memorizing a formula
		\item Exploit right angles to compute projections that would otherwise be
			difficult
	\end{itemize}

\subsection*{Context}
	Students have covered projections and components in class two weeks ago. Last week they started
		working with matrices/matrix multiplication.

	In this class, $\Proj_X\vec v$ is defined as the closest vector in $X$ to $\vec v$. It is \emph{not}
		defined in terms of a formula. Also, to avoid confusion, we \emph{don't write}
		$\Proj_{\vec u}\vec v$, since this is easily confused with projection onto a singleton.
		Instead, the ``component of $\vec v$ in the direction $\vec u$'', written $\Comp_{\vec u}\vec v$
		is defined to be the vector in the direction of $\vec u$ so that $\vec v-\Comp_{\vec u}\vec v$
		is orthogonal to $\vec u$. There is a formula for this operation (unlike projections), and they
		should all know it since they've used it on their homework.

\subsection*{What to Do}
	Introduce the learning objectives for the day's tutorial. Explain that, like most things in life,
		there is no algorithm to compute projections---the only way to do it is to understand
		the idea. Since we're practicing learning and understanding non-algorithmic tasks in this
		class, projections provide the perfect place to practice.

	Have students pair up and write the definition of projection. Again, make them
		write it. These definitions will show up on the midterms and many will
		write them wrong. Again, you might have some groups come up to the board and
		write their definitions and then have a short class discussion on whether they are right.
		This definition will be easier for them since there aren't any quantifiers.

		After everyone is on the same page with the definition, have them continue on \#2.
		The point of \#2 is to have them cement in their minds the link between projections
		and orthogonality. Don't give this point away too readily--it's best if they discover it themselves, so
		they \emph{own} it.

		After most of finished \#2, have a class discussion and repeat with \#3. Remember, your goal
		is not to get through problems. This tutorial has \emph{way} more problems than the students
		will be able to get through.

		7 minutes before the end of class, pick a problem that most students have started working on
		to do as a wrap-up.

\subsection*{Notes}
	\begin{itemize}
		\item The definition of $\Proj_X\vec v$ is written mostly with words instead of with
			set notation. This might throw some students off---they won't be able to tell that it's precise.
		\item Projections relate to orthogonality, but not always. \#2 is designed to tease this out---that
			for non-smooth shapes projections might not relate to orthogonality.
		\item With our definition, $\Proj_X\vec v$ might not be unique and so might not be well defined. We will
			never try to trick a student by giving them a non-unique projection. If a student asks about
			this, you can tell them that on tests the projection will always be unique.
		\item For \#3, student \emph{should} know how to minimize a parabola, but many won't. However, almost all
			have taken calculus in high school, so feel free to use basic calculus to answer that question.
		\item You won't get to \#4, but if students are working on it, make sure they draw a picture. It's very
			hard to do in one's head.
	\end{itemize}