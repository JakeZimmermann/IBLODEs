\begin{objectives}
	In this tutorial you will MAT 244's expectations around mathematical communication as well
	as what makes a good executive summary.
\end{objectives}

\vspace{-.5em}
\subsection*{Problems}
\vspace{-.5em}

%%%%%%%%%%%%%%%%%%%%%%%%%%

\begin{enumerate}
	\item The first step in solving a mathematics question is applying and understanding logic until \emph{you}
	understand the question and its answer. The second step is documenting and communicating your steps/argument. 
	

	Below are questions that might appear on a MAT244 exam paired with possible answers.

	\begin{enumerate}
		\item[(A1)] \emph{Write the complete solution to $y'=2y$. No work necessary.}

		\fbox{
		\begin{minipage}{0.4\textwidth}
			\begin{itemize}
				\item[] $y=Ae^{2t}$
			\end{itemize}
		\end{minipage}
		}

		\item[(A2)] \emph{Let $f$ be a solution to $y'=2y$ satisfying $f(0)=1$. Use Euler's method to approximate $f(2)$. Explain how you arrived at your answer.}
		
		\fbox{
		\begin{minipage}{0.4\textwidth}
			\begin{itemize}
				\item[] $\Delta=0.5$
				\item[] $f_{n+1}=f_{n}+2f_{n}\Delta$
				\item[] $f_0=1$
				\item[] $f(2)\approx 16$
			\end{itemize}
		\end{minipage}
		}

		\item[(A3)] \emph{Let $\vec r\,'(t)=A\vec r(t)$ be a differential equation and let $\vec p(t)$ and $\vec q(t)$
		be solutions. Show that $\vec s = \vec p+\vec q$ is also a solution.}

		
		\fbox{
		\begin{minipage}{0.4\textwidth}
		\begin{itemize}
			\item[] $\vec s\,'=\vec p\,'+\vec q\,'$
			\item[] $\vec p\,'=A\vec p$
			\item[] $\vec q\,'=A\vec q$
			\item[] $\vec p\,'+\vec q\,' = A\vec p+A\vec q$
			\item[] $A(\vec p+\vec q) = A\vec p+A\vec q$
			\item[] $A\vec s=A(\vec p+\vec q)$
			\item[] $\vec s\,'=A\vec s$
		\end{itemize}
		\end{minipage}
		}
		

	\end{enumerate}

	\bigskip
	\begin{enumerate}
		\item Read through each question/answer pair. Are there \emph{logical} errors in the answers? If so, what are they?
		\item On a test, (A1) would be worth 2 points, and (A2) and (A3) would be worth 4 points. How many points do you think
			each answer would be awarded? Why?
		\item The actual scoring for each question was as follows: (A1) 1/2 points, (A2) 0/4 points, and (A3) 1/4 points. What is missing from each
		answer that is prevent it from getting full marks?
		\item Fix each argument so that it would receive full marks.
	\end{enumerate}

	\item As part of your final report and your Homework 3, you will be writing an executive summary. The University of Southern California (USC)
	provides guidelines for what makes a good executive summary at \url{https://libguides.usc.edu/writingguide/executivesummary}
	
	Recall the \emph{bee} question from Homework 2. Imagine that you were commissioned by a farmer to study bees in his field and
	to give him guidance on where to plant his flowers so that they will be pollinated.

			\vspace{-.3cm}
	\begin{enumerate}
		\item Read through the USC guidelines. What are three things you would expect to appear in an executive summary for the bee question?
		\item The following is an executive summary for the bee question created by an AI. 

		\framebox{
		\begin{minipage}{0.8\textwidth}
			\small The summary of the bee's behavior can be described as
			follows:

			\vspace{-.3cm}
			\paragraph{In English:}

			The bee's movement is modeled based on the intensity of flower smell and its
			distance from the hive. The bee travels in a field of flowers, where the smell
			intensity varies on a logarithmic scale. The bee's movement is influenced
			by both the current smell intensity and its distance from the hive. The model
			suggests that bees will only stay still in the hive or on a flower, and
			flowers are located at specific distances from the hive. The bee's behavior
			includes moving towards or away from the hive based on smell intensity,
			potentially overshooting its target before settling on a flower. Odd-numbered
			flowers tend to attract bees, while even-numbered flowers are unstable
			points.

			\vspace{-.3cm}
			\paragraph{In mathematical terms:}
			\[
				\begin{cases}
					S'(t) & = \sin(D(t))   \\
					D'(t) & = S(t) - D(t)
				\end{cases}
			\]
			Where:
			\begin{itemize}
				\item $S(t) =$ intensity of flower smell at time $t$ (logarithmic scale
					from $-\infty$ to $+\infty$)

				\item $D(t) =$ (positive) distance between the bee and its hive at time $t$
			\end{itemize}

			\vspace{-.3cm}
			\paragraph{Key points:}

			Smell intensity is constant when $S' = 0$, which occurs when $\sin(D) = 0$
			or $D = n\pi$, where $n$ is a non-negative integer.

			Flowers are located at distances $k\pi$ from the hive, where $k > 0$ is a
			whole number.

			The system can be simulated using Euler's method for the differential
			equations.

			Phase portraits show that odd-numbered flowers are attracting points, while
			even-numbered ones are unstable.

			There's a critical smell intensity (around 7.1 $\pm$ 0.01) that determines
			whether a bee will fly to the 1st or 3rd flower when starting from the hive.

			\medskip
			This model captures the essence of how bees navigate using smell intensity
			and distance from their hive to locate flowers in a field.
		\end{minipage}
		}

		\begin{enumerate}
			\item What did the AI do well in the executive summary?
			\item What did the AI do poorly?
			\item Executive summaries rarely include mathematical formulas. Do you agree with its choice to include one? What is the alternative?
			\item The AI's executive summary does not directly address the farmer's question about where to plant flowers.

			Write an introductory paragraph to a new executive summary that addresses the farmer's question.
		\end{enumerate}

		\item For Homework 3, you are asked to write an executive summary. What key items should your executive summary for Homework 3 include?
		What would be a good introductory paragraph?

	\end{enumerate}


	%\item We gave the AI$^{0}$ the Executive Summary guidelines from the
	%	University of Southern California. Below is the revised Executive Summary.

	%	\hspace{-1cm}
	%	\framebox{
	%	\begin{minipage}{\textwidth}
	%		\small
	%		\paragraph{Executive Summary:}
	%		Bee Behavior Model Analysis\\

	%		This study examines a mathematical model of bee behavior in a field of
	%		flowers, focusing on the relationship between smell intensity and the bee's
	%		distance from its hive. The model uses differential equations to describe
	%		the bee's movement and decision-making process. \\

	%		\paragraph{Key Findings:}
	%		Smell intensity is constant at specific distances from the hive,
	%		particularly when the distance is a non-negative multiple of $\pi$.

	%		Flowers are located at distances that are multiples of $\pi$ from the hive.

	%		When a bee starts at a distance of 5 units from its hive with neutral
	%		smell intensity, it initially moves towards the hive, then towards the
	%		first flower, eventually settling on the flower at a distance of $\pi$ from
	%		the hive.

	%		Odd-numbered flowers (at distances $\pi, 3\pi, 5\pi$, etc.) attract bees,
	%		while even-numbered flowers are unstable attractors.

	%		The minimum smell intensity required for a bee to venture to the third
	%		flower ($3\pi$ distance) from its hive is approximately $7.1 \pm 0.01$ on
	%		the logarithmic scale used in the model. \\

	%		\paragraph{Methodology:}
	%		The study employed a mix of analytic arguments, computer-based simulations
	%		using Euler's method, and qualitative analysis techniques such as phase portraits.

	%		\paragraph{Implications:}
	%		This model provides insights into bee foraging behavior, suggesting that bees
	%		are more likely to visit certain flowers based on their distance from the hive
	%		and the intensity of their smell. This information could be valuable for
	%		understanding pollination patterns and optimizing flower placement in
	%		agricultural settings. \\

	%		\paragraph{Recommendations:}
	%		Further research to validate the model's predictions in real-world scenarios.
	%		Exploration of how environmental factors might influence the model's
	%		parameters. Investigation of potential applications in agriculture and ecology.
	%		\\

	%		This executive summary provides a concise overview of the bee behavior model,
	%		its key findings, and potential implications, serving as a foundation for
	%		further discussion and research in this area.
	%	\end{minipage}
	%	}

\end{enumerate}

%%%%%%%%%%%%%%%%%%%%%%%%%%