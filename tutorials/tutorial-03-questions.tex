		\begin{objectives}
	In this tutorial you will explore orthogonality in depth.

	These problems relate to the following course learning objectives:
	\textit{Work independently to understand concepts and procedures that have not been previously
		explained to you},
		\textit{translate between algebraic and geometric viewpoints to solve problems}, and
		\textit{understand definitions that have been written by others}.
		\end{objectives}

		\vspace{-.4cm}
\subsection*{Definitions}
		Recall the vectors $\vec a$ and $\vec b$ are \emph{orthogonal} if $\vec a\cdot \vec b=0$.
		We say the \emph{sets $A$ and $B$ are orthogonal} if every vector in $A$ is orthogonal to every
		vector in $B$.

		\vspace{-.4cm}
\subsection*{Problems}
Let $\vec v_1=\mat{1\\1\\1\\1}$,
		$\vec v_2=\mat{-1\\1\\1\\1}$,
		$\vec v_3=\mat{-1\\-1\\1\\1}$,
		$\vec v_4=\mat{1\\0\\0\\1}$,
		$\vec v_5=\mat{3\\-1\\-1\\-1}$, and
		$\vec v_6=\mat{1\\1\\2\\0}$.

\begin{enumerate}
	\item
	\begin{enumerate}
		\item Identify all pairs of orthogonal vectors among $\vec v_1$, \ldots, $\vec v_6$.
		\item Let $A=\{\vec v_1, \vec v_2\}$ and $B=\{\vec v_3,\vec v_4\}$. Are $A$ and $B$ orthogonal sets?
			Why or why not?
		\item Let $P=\{\vec v_1, \vec v_6\}$ and $Q=\{\vec v_3,\vec v_5\}$. Are $P$ and $Q$ orthogonal sets?
			Why or why not?
		\item Can you split the vectors $\vec v_1$, \ldots, $\vec v_6$ into two non-empty sets that are orthogonal to
			each other? Explain.
	\end{enumerate}
	\item
	\begin{enumerate}
		\item Using guess-and-check, find two vectors that are orthogonal to both $\vec v_1$ and $\vec v_2$.
		\item Set up and solve a system of equations to find all vectors orthogonal to $\vec v_1$ and $\vec v_2$.
	\end{enumerate}
	\item The dot product is \emph{commutative} and \emph{distributive}. That is $\vec v\cdot (\alpha\vec a+\vec b)=
		(\alpha\vec a+\vec b)\cdot \vec v=\alpha(\vec a\cdot \vec v)+\vec b\cdot\vec v$. Use this to show that if
		the set $X=\{\vec x\}$ is orthogonal to the set $Y=\{\vec y_1,\vec y_2,\vec y_3,\vec y_4\}$,
		then $X$ is also orthogonal to $\Span Y$.
	
	\item We say that $\vec a$ and $\vec b$ are \emph{close} if $\|\vec a-\vec b\|$ is small. We will see if we can
		extend this concept to lines.

		Let the lines $\ell_1$, $\ell_2$, $\ell_3$, and $\ell_4$ be given by the equations $y=x$, $y=1.001x$,
				$y=2000x$, $y=3000x$.
		\begin{enumerate}
			\item Out of $\ell_1$, \ldots, $\ell_4$, which lines would you call ``close''? Can you come up
				with a mathematical definition to justify your conclusion?
			\item For $\ell_1$, \ldots, $\ell_4$, find unit normal vectors $\vec n_1$, \ldots, $\vec n_4$.
				For consistency, ensure each unit normal vector points towards the upper left (i.e., has negative first coordinate
				and positive second coordinate).
			\item Compute the distances between $\vec n_1$, \ldots, $\vec n_4$. Do these distances coincide with your
				intuition about closeness? Why might comparing normal vectors be preferable to comparing direction vectors?
		\end{enumerate}
	
\end{enumerate}