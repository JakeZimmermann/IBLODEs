		\begin{objectives}
			In this tutorial you will explore the Lotka-Volterra model in more detail.

	These problems relate to the following Course Learning Objectives:
			\textit{
Interpret and analyze models based on differential equations using tools like simulation, phase
portraits, analysis of stability, and linear approximation.
			}
		\end{objectives}

\subsection*{Problems}

Consider the Predator-Prey Lotka-Volterra model (the LV model) with unknown parameters:
\begin{align*}
	F' &= a \cdot R\cdot F - b \cdot F \\
	R' &= c \cdot R - d \cdot R \cdot F
\end{align*}
and $a,b,c,d\geq 0$. 

\begin{enumerate}
	\item In class, we used the parameters $a=0.01, b=1.1, c=1.1, d=0.1$.

	\begin{enumerate}
		\item	Make a phase portrait for the LV model on Desmos 
			using the parameters from class.

		\url{https://www.desmos.com/calculator/vrk0q4espx}

		\item The arrows point in the clockwise direction. What would
			it mean if the arrows pointed in the
			\emph{counter-clockwise} direction? Explain in terms 
			of Rabbits and Foxes. Would this be realistic?
	\end{enumerate}

	\item\label{qual}
		Using the parameters from class, the rabbit and fox populations 
		evolve \emph{periodically}. In this case, all
		solutions share the same \emph{qualitative}
		behaviour (namely they are periodic).

		\begin{enumerate}
			\item Analyze your phase portrait for different values of $a,b,c,d$ until you
				find parameters that look like they give rise to a different
				qualitative behaviour.
			\item Simulate (using Euler's method in a spreadsheet) some
				solutions that exhibit different qualitative
				behaviour.
			\item Tell a story about the fox and rabbit population for
				the different parameters you came up with.
		\end{enumerate}
		
		\item If $a,b,c,d>0$, the maximum rabbit population always
			occurs when there are $c/d$ foxes.
		\begin{enumerate}
			\item Justify this claim. (Make sure your justification references the equations.)
			\item Find a similar condition for when the foxes are at their maximum population.
			\item Find conditions for when neither the rabbit nor the fox populations change.
			\item If we relax the conditions to $a,b,c,d\geq 0$, are your conditions still valid?
				Can your work in this question be used to explain when you should
				expect different qualitative behaviour of solutions (i.e., what
				you worked on in part \ref{qual})?
		\end{enumerate}

%		\item From the phase plane, how can you identify the maximum number of foxes/rabbits? What about the minimum?
%		\item Find a formula for the maximum and minimum number of foxes and rabbits. \\
%		
%
%		\item Using an implicit derivative, calculate $\frac{dF}{dR}$ as a function of $F$ and $R$. Solve the separable differential equation.
%				$$
%				-b \ln(y)+ay +dx-c\ln(x) = {\rm Constant}
%				$$
%		\item This means that the graphs of solutions in the phase plane, are level sets of a function $\Psi(x,y)$. What is this function $\Psi(x,y)$?
%		\item Use this fact to find a formula for the max/min of the number of rabbits/foxes that depend only on the initial conditions $r_0$ and $f_0$ (and on the constants $a,b,c,d$).
		
		

	
	
		
\end{enumerate}
