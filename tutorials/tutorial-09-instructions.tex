\subsection*{Learning Objectives}
Students need to be able to\ldots
\begin{itemize}
	\item Build a spreadsheet that can use Euler's method to approximate the solution to a system of differential equations
	\item Rewrite a higher-order ODE as a system of ODEs.
	\item Use Euler's method to simulate the solution to a higher-order ODE.
	\item Numerically approximate the solution to a boundary value problem involving higher-order ODEs.
\end{itemize}

\subsection*{Context}

In class we used Euler's method extensively.

We also introduced boundary-value problems (BVPs) in class to show that studying existence 
and uniqueness of solution is not a trivial matter.

For this type of differential equation, however, we can't use Euler's method directly, since it requires initial conditions.
Instead, we repeatedly apply Euler's method to find better and better guesses at the initial conditions.


\subsection*{Resources for TAs}

Excel spreadsheet with the Shooting Method.

\begin{itemize}
	\item \url{https://utoronto-my.sharepoint.com/:x:/g/personal/bernardo_galvao_sousa_utoronto_ca/EQO08o4-PdFMr8PMd80LmP0BPilrLuGIcNMpP-oVpfM4ow?e=IR39N4}
\end{itemize}




\subsection*{What to Do}
Introduce the learning objectives for the day's tutorial.

Explain the big picture: if we can rewrite a higher-order ODE in terms of a system of first-order ODEs, 
we can use Euler's method to approximate the solution to the higher-order ODE.

Have students get into small groups and start on \#1. Each group needs to have at least 1 laptop.
This question should be done quickly---it's setting the stage for \#2.




\subsection*{Notes}

	\begin{enumerate}
		\item This should be a quick review.
		\item This question carefully walks students through the Shooting Method. If students get stuck on
		this one, go ahead and give them a big-picture overview of the Shooting Method.
		\item This is here for the quicker students, though they may need a smaller step size than $\Delta=0.01$ to get good results.
	\end{enumerate}





